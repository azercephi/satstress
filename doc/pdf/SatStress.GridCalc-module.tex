%
% API Documentation for SatStress
% Module SatStress.GridCalc
%
% Generated by epydoc 3.0.1
% [Fri Mar 28 12:25:18 2008]
%

%%%%%%%%%%%%%%%%%%%%%%%%%%%%%%%%%%%%%%%%%%%%%%%%%%%%%%%%%%%%%%%%%%%%%%%%%%%
%%                          Module Description                           %%
%%%%%%%%%%%%%%%%%%%%%%%%%%%%%%%%%%%%%%%%%%%%%%%%%%%%%%%%%%%%%%%%%%%%%%%%%%%

    \index{SatStress \textit{(package)}!SatStress.GridCalc \textit{(module)}|(}
\section{Module SatStress.GridCalc}

    \label{SatStress:GridCalc}
Calculate stresses on an icy satellite over a rectangular geographic region
on a regularly spaced lat-lon grid.

The datacube containing the results of the calculation are output as a 
Unidata 
netCDF\footnote{\href{http://www.unidata.ucar.edu/software/netcdf}{http://www.unidata.ucar.edu/software/netcdf}}
(.nc) file, which can be displayed using a wide variety of visualization 
software.


%%%%%%%%%%%%%%%%%%%%%%%%%%%%%%%%%%%%%%%%%%%%%%%%%%%%%%%%%%%%%%%%%%%%%%%%%%%
%%                               Functions                               %%
%%%%%%%%%%%%%%%%%%%%%%%%%%%%%%%%%%%%%%%%%%%%%%%%%%%%%%%%%%%%%%%%%%%%%%%%%%%

  \subsection{Functions}

    \label{SatStress:GridCalc:main}
    \index{SatStress \textit{(package)}!SatStress.GridCalc \textit{(module)}!SatStress.GridCalc.main \textit{(function)}}

    \vspace{0.5ex}

\hspace{.8\funcindent}\begin{boxedminipage}{\funcwidth}

    \raggedright \textbf{main}()

    \vspace{-1.5ex}

    \rule{\textwidth}{0.5\fboxrule}
\setlength{\parskip}{2ex}
    Calculate satellite stresses on a regular grid within a lat-lon window.

\setlength{\parskip}{1ex}
    \end{boxedminipage}


%%%%%%%%%%%%%%%%%%%%%%%%%%%%%%%%%%%%%%%%%%%%%%%%%%%%%%%%%%%%%%%%%%%%%%%%%%%
%%                           Class Description                           %%
%%%%%%%%%%%%%%%%%%%%%%%%%%%%%%%%%%%%%%%%%%%%%%%%%%%%%%%%%%%%%%%%%%%%%%%%%%%

    \index{SatStress \textit{(package)}!SatStress.GridCalc \textit{(module)}!SatStress.GridCalc.Grid \textit{(class)}|(}
\subsection{Class Grid}

    \label{SatStress:GridCalc:Grid}
A container class defining the temporal and geographic range and resolution
of the calculation.

The parameters defining the calculation grid are read in from a name value 
file, parsed into a Python dictionary using \texttt{SatStress.nvf2dict}, 
and used to set the data attributes of the \texttt{Grid} object.

The geographic extent of the calculation is specified by minimum and 
maximum values for latitude and longitude.

The geographic resolution of the calculation is defined by an angular 
separation between calculations.  This angular separation is the same in 
the north-south and the east-west direction.

The temporal range and resolution of the calculation can be specified 
either in terms of actual time units (seconds) or in terms of the 
satellite's orbital position (in degrees).  In both cases, time=0 is taken 
to occur at periapse.

\(\Delta\) is a measure of how viscous or elastic the response of the body 
is.  It's equal to (\(\mu\))/(\(\eta\)\(\omega\)) where \(\mu\) and 
\(\eta\) are the shear modulus and viscosity of the surface layer, 
respectively, and \(\omega\) is the forcing frequency to which the body is 
subjected (see Wahr et al. (2008) for a detailed discussion).  It is a 
logarithmic parameter, so its bounds are specified as powers of 10, e.g. if
the minimum value is -3, the initial \(\Delta\) is 10{\textasciicircum}-3 =
0.001.


%%%%%%%%%%%%%%%%%%%%%%%%%%%%%%%%%%%%%%%%%%%%%%%%%%%%%%%%%%%%%%%%%%%%%%%%%%%
%%                                Methods                                %%
%%%%%%%%%%%%%%%%%%%%%%%%%%%%%%%%%%%%%%%%%%%%%%%%%%%%%%%%%%%%%%%%%%%%%%%%%%%

  \subsubsection{Methods}

    \label{SatStress:GridCalc:Grid:__init__}
    \index{SatStress \textit{(package)}!SatStress.GridCalc \textit{(module)}!SatStress.GridCalc.Grid \textit{(class)}!SatStress.GridCalc.Grid.\_\_init\_\_ \textit{(method)}}

    \vspace{0.5ex}

\hspace{.8\funcindent}\begin{boxedminipage}{\funcwidth}

    \raggedright \textbf{\_\_init\_\_}(\textit{self}, \textit{gridFile}, \textit{satellite}={\tt None})

    \vspace{-1.5ex}

    \rule{\textwidth}{0.5\fboxrule}
\setlength{\parskip}{2ex}
    Initialize the Grid object from a gridFile.

\setlength{\parskip}{1ex}
    \end{boxedminipage}


%%%%%%%%%%%%%%%%%%%%%%%%%%%%%%%%%%%%%%%%%%%%%%%%%%%%%%%%%%%%%%%%%%%%%%%%%%%
%%                          Instance Variables                           %%
%%%%%%%%%%%%%%%%%%%%%%%%%%%%%%%%%%%%%%%%%%%%%%%%%%%%%%%%%%%%%%%%%%%%%%%%%%%

  \subsubsection{Instance Variables}

    \vspace{-1cm}
\hspace{\varindent}\begin{longtable}{|p{\varnamewidth}|p{\vardescrwidth}|l}
\cline{1-2}
\cline{1-2} \centering \textbf{Name} & \centering \textbf{Description}& \\
\cline{1-2}
\endhead\cline{1-2}\multicolumn{3}{r}{\small\textit{continued on next page}}\\\endfoot\cline{1-2}
\endlastfoot\raggedright g\-r\-i\-d\-\_\-i\-d\- & \raggedright A string identifying the grid

            {\it (type=str)}&\\
\cline{1-2}
\raggedright l\-a\-t\-\_\-m\-a\-x\- & \raggedright Northern bound, degrees (north positive).

            {\it (type=float)}&\\
\cline{1-2}
\raggedright l\-a\-t\-\_\-m\-i\-n\- & \raggedright Southern bound, degrees (north positive).

            {\it (type=float)}&\\
\cline{1-2}
\raggedright l\-a\-t\-l\-o\-n\-\_\-s\-t\-e\-p\- & \raggedright Angular separation between calculations.

            {\it (type=float)}&\\
\cline{1-2}
\raggedright l\-o\-n\-\_\-m\-a\-x\- & \raggedright Eastern bound, degrees (east positive).

            {\it (type=float)}&\\
\cline{1-2}
\raggedright l\-o\-n\-\_\-m\-i\-n\- & \raggedright Western bound, degrees (east positive).

            {\it (type=float)}&\\
\cline{1-2}
\raggedright n\-s\-r\-\_\-d\-e\-l\-t\-a\-\_\-m\-a\-x\- & \raggedright Final \(\Delta\) = 10{\textasciicircum}(nsr\_delta\_max)

            {\it (type=float)}&\\
\cline{1-2}
\raggedright n\-s\-r\-\_\-d\-e\-l\-t\-a\-\_\-m\-i\-n\- & \raggedright Initial \(\Delta\) = 10{\textasciicircum}(nsr\_delta\_min)

            {\it (type=float)}&\\
\cline{1-2}
\raggedright n\-s\-r\-\_\-d\-e\-l\-t\-a\-\_\-n\-u\-m\-s\-t\-e\-p\-s\- & \raggedright How many \(\Delta\) values to calculate total

            {\it (type=int)}&\\
\cline{1-2}
\raggedright o\-r\-b\-i\-t\-\_\-m\-a\-x\- & \raggedright Final orbital position in degrees (0 = periapse)

            {\it (type=float)}&\\
\cline{1-2}
\raggedright o\-r\-b\-i\-t\-\_\-m\-i\-n\- & \raggedright Initial orbital position in degrees (0 = periapse)

            {\it (type=float)}&\\
\cline{1-2}
\raggedright o\-r\-b\-i\-t\-\_\-s\-t\-e\-p\- & \raggedright Orbital angular separation between calculations in degrees

            {\it (type=float)}&\\
\cline{1-2}
\raggedright t\-i\-m\-e\-\_\-m\-a\-x\- & \raggedright Final time at which calculation ends.

            {\it (type=float)}&\\
\cline{1-2}
\raggedright t\-i\-m\-e\-\_\-m\-i\-n\- & \raggedright Initial time at which calculation begins (0 = periapse).

            {\it (type=float)}&\\
\cline{1-2}
\raggedright t\-i\-m\-e\-\_\-s\-t\-e\-p\- & \raggedright Seconds between subsequent calculations.

            {\it (type=float)}&\\
\cline{1-2}
\end{longtable}

    \index{SatStress \textit{(package)}!SatStress.GridCalc \textit{(module)}!SatStress.GridCalc.Grid \textit{(class)}|)}
    \index{SatStress \textit{(package)}!SatStress.GridCalc \textit{(module)}|)}
