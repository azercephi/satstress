%
% API Documentation for SatStress
% Module SatStress.SatStress
%
% Generated by epydoc 3.0.1
% [Fri Mar 28 12:25:18 2008]
%

%%%%%%%%%%%%%%%%%%%%%%%%%%%%%%%%%%%%%%%%%%%%%%%%%%%%%%%%%%%%%%%%%%%%%%%%%%%
%%                          Module Description                           %%
%%%%%%%%%%%%%%%%%%%%%%%%%%%%%%%%%%%%%%%%%%%%%%%%%%%%%%%%%%%%%%%%%%%%%%%%%%%

    \index{SatStress \textit{(package)}!SatStress.SatStress \textit{(module)}|(}
\section{Module SatStress.SatStress}

    \label{SatStress:SatStress}
A framework for calculating the surface stresses at a particular place and 
time on a satellite resulting from one or more tidal potentials.

(section) 1 Input and Output

  Because \texttt{SatStress} is a "library" module, it doesn't do a lot of 
  input and output - it's mostly about doing calculations.  It does need to
  read in the specification of a \texttt{Satellite} object though, and it 
  can write the same kind of specification out.  To do this, it uses 
  name-value files, and a function called \texttt{nvf2dict}, which creates 
  a Python dictionary (or "associative array").

  A name-value file is just a file containing a bunch of name-value pairs, 
  like:

\begin{alltt}
 ORBIT\_ECCENTRICITY = 0.0094   \# e must be {\textless} 0.25\end{alltt}

  It can also contain comments to enhance human readability (anything 
  following a '\#' on a line is ignored, as with the note in the line 
  above).

(section) 2 Satellites

  Obviously if we want to calculate the stresses on the surface of a 
  satellite, we need to define the satellite, this is what the 
  \texttt{Satellite} object does.

  (section) 2.1 Specifying a Satellite

    In order to specify a satellite, we need:

    \begin{itemize}
    \setlength{\parskip}{0.6ex}
      \item an ID of some kind for the planet/satellite pair of interest

      \item the charactaristics of the satellite's orbital environment

      \item the satellite's internal structure and material properties

      \item the forcings to which the satellite is subjected

    \end{itemize}

    From a few basic inputs, we can calculate many derived characteristics,
    such as the satellite's orbital period or the surface gravity.

    The internal structure and material properties are specified by a 
    series of concentric spherical shells (layers), each one being 
    homogeneous throughout its extent.  Given the densities and thicknesses
    of the these layers, we can calculate the satellite's overall size, 
    mass, density, etc.

    Specifying a tidal forcing may be simple or complex.  For instance, the
    \texttt{Diurnal} forcing depends only on the orbital eccentricity (and 
    other orbital parameters already supplied), and the \texttt{NSR} 
    forcing requires only the addition of the non-synchronous rotation 
    period of the shell.  Specifying an arbitrary true polar wander 
    trajectory would be much more complex.

    For the moment, becuase we are only including simple forcings, their 
    specifying parameters are read in from the satellite definition file.  
    If more, and more complex forcings are eventually added to the model, 
    their specification will probably be split into a separate input file.

  (section) 2.2 Internal Structure and Love Numbers

    SatStress treats the solid portions of the satellite as viscoelastic 
    Maxwell 
    solids\footnote{\href{http://en.wikipedia.org/wiki/Maxwell_material}{http://en.wikipedia.org/wiki/Maxwell\_material}},
    that respond differently to forcings having different frequencies 
    (\(\omega\)).  Given the a specification of the internal structure and 
    material properties of a satellite as a series of layers, and 
    information about the tidal forcings the body is subject to, it's 
    possible to calculate appropriate Love numbers, which describe how the 
    body responds to a change in the gravitational potential.

    Currently the calculation of Love numbers is done by an external 
    program written in Fortran by John Wahr and others, with roots reaching
    deep into the Dark Ages of computing.  As that code (or another Love 
    number code) is more closely integrated with the model, the internal 
    structure of the satellite will become more flexible, but for the 
    moment, we are limited to assuming a 4-layer structure:

    \begin{itemize}
    \setlength{\parskip}{0.6ex}
      \item \textbf{\texttt{ICE\_UPPER}}: The upper portion of the shell 
        (cooler, stiffer)

      \item \textbf{\texttt{ICE\_LOWER}}: The lower portion of the shell 
        (warmer, softer)

      \item \textbf{\texttt{OCEAN}}: An inviscid fluid decoupling the shell 
        from the core.

      \item \textbf{\texttt{CORE}}: The silicate interior of the body.

    \end{itemize}

(section) 3 Stresses

  \texttt{SatStress} can calculate the following stress fields:

  \begin{enumerate}

  \setlength{\parskip}{0.5ex}
    \item \textbf{\texttt{Diurnal}}: stresses arising from an eccentric orbit, 
      having a forcing frequency equal to the orbital frequency.

    \item \textbf{\texttt{NSR}}: stresses arising due to the 
      faster-than-synchronous rotation of a floating shell that is 
      decoupled from the satellite's interior by a fluid layer (an ocean).

  \end{enumerate}

  The expressions defining these stress fields are derived in "Modeling 
  Stresses on Satellites due to Non-Synchronous Rotation and Orbital 
  Eccentricity Using Gravitational Potential Theory" (preprint, 15MB 
  PDF\footnote{\href{http://icymoons.com/Wahretal2008/stress.paper.pdf}{http://icymoons.com/Wahretal2008/stress.paper.pdf}})
  by Wahr et al. (submitted to \textit{Icarus}, in March, 2008).

  (section) 3.1 Stress Fields Live in \texttt{StressDef} Objects

    Each of the above stress fields is defined by a similarly named 
    \texttt{StressDef} object.  These objects contain the formulae 
    necessary to calculate the surface stress.  The expressions for the 
    stresses depend on many parameters which are defined within the 
    \texttt{Satellite} object, and so to create a \texttt{StressDef} 
    object, you need to provide a \texttt{Satellite} object.

    There are many formulae which are identical for both the \texttt{NSR} 
    and \texttt{Diurnal} stress fields, and so instead of duplicating them 
    in both classes, they reside in the \texttt{StressDef} \textit{base 
    class}, from which all \texttt{StressDef} objects inherit many 
    properties.

    The main requirement for each \texttt{StressDef} object is that it must
    define the three components of the stress tensor \(\tau\):

    \begin{itemize}
    \setlength{\parskip}{0.6ex}
      \item \texttt{Ttt} (\(\tau\)\_\(\theta\)\(\theta\)) the north-south 
        (latitudinal) component

      \item \texttt{Tpt} (\(\tau\)\_\(\phi\)\(\theta\) = 
        \(\tau\)\_\(\theta\)\(\phi\)) the shear component

      \item \texttt{Tpp} (\(\tau\)\_\(\phi\)\(\phi\)) the east-west 
        (longitudinal) component

    \end{itemize}

  (section) 3.2 Stress Calculations are Performed by \texttt{StressCalc} Objects

    Once you've \textit{instantiated} a \texttt{StressDef} object, or 
    several of them (one for each stress you want to include), you can 
    compose them together into a \texttt{StressCalc} object, which will 
    actually do calculations at given points on the surface, and given 
    times, and return a 2x2 matrix containing the resulting stress tensor 
    (each component of which is the sum of all of the corresponding 
    components of the stress fields that were used to instantiated the 
    \texttt{StressCalc} object).

    This is (hopefully) easier than it sounds.  With the following few 
    lines, you can construct a satellite, do a single calculation on its 
    surface, and see what it looks like:

\begin{alltt}
\pysrcprompt{{\textgreater}{\textgreater}{\textgreater} }\pysrckeyword{from} SatStress \pysrckeyword{import} *
\pysrcprompt{{\textgreater}{\textgreater}{\textgreater} }the\_sat = Satellite(open(\pysrcstring{"input/Europa.satellite"}))
\pysrcprompt{{\textgreater}{\textgreater}{\textgreater} }the\_stresses = StressCalc([Diurnal(the\_sat), NSR(the\_sat)])
\pysrcprompt{{\textgreater}{\textgreater}{\textgreater} }Tau = the\_stresses.tensor(theta=pi/4.0, phi=pi/3.0, t=10000)
\pysrcprompt{{\textgreater}{\textgreater}{\textgreater} }\pysrckeyword{print}(Tau)\end{alltt}
    The \texttt{SatStress.test} function shows a slightly more complex 
    example, which should be enough to get you started using the package.

  (section) 3.3 Extending the Model

    Other stress fields can (and hopefully will!), be added easily, so long
    as they use the same mathematical definition of the membrane stress 
    tensor (\(\tau\)), as a function of co-latitude (\(\theta\)) (measured 
    south from the north pole), east-positive longitude (\(\phi\)), 
    measured from the meridian on the satellite which passes through the 
    point on the satellite directly beneath the parent planet (assuming a 
    synchronously rotating satellite), and time (\textbf{\textit{t}}), 
    defined as seconds elapsed since pericenter.

    This module could also potentially be extended to also calculate the 
    surface strain (\(\epsilon\)) and displacement (\textbf{\textit{s}}) 
    fields, or to calculate the stresses at any point within the satellite.

\textbf{Date:} Fri Mar 28 19:25:16 2008



\textbf{Author:} Zane Selvans



\textbf{Contact:} zane.selvans@colorado.edu



\textbf{Copyright:} 2008



\textbf{License:} GNU General Public License version 3 (GPL v3)




%%%%%%%%%%%%%%%%%%%%%%%%%%%%%%%%%%%%%%%%%%%%%%%%%%%%%%%%%%%%%%%%%%%%%%%%%%%
%%                               Functions                               %%
%%%%%%%%%%%%%%%%%%%%%%%%%%%%%%%%%%%%%%%%%%%%%%%%%%%%%%%%%%%%%%%%%%%%%%%%%%%

  \subsection{Functions}

    \label{SatStress:SatStress:nvf2dict}
    \index{SatStress \textit{(package)}!SatStress.SatStress \textit{(module)}!SatStress.SatStress.nvf2dict \textit{(function)}}

    \vspace{0.5ex}

\hspace{.8\funcindent}\begin{boxedminipage}{\funcwidth}

    \raggedright \textbf{nvf2dict}(\textit{nvf}, \textit{comment}={\tt \texttt{'}\texttt{\#}\texttt{'}})

    \vspace{-1.5ex}

    \rule{\textwidth}{0.5\fboxrule}
\setlength{\parskip}{2ex}
    Reads from a file object listing name value pairs, creating and 
    returning a corresponding Python dictionary.

    The file should contain a series of name value pairs, one per line 
    separated by the '=' character, with names on the left and values on 
    the right.  Blank lines are ignored, as are lines beginning with the 
    comment character (assumed to be the pound or hash character '\#', 
    unless otherwise specified).  End of line comments are also allowed.  
    String values should not be quoted in the file.  Names are case 
    sensitive.

    Returns a Python dictionary that uses the names as keys and the values 
    as values, and so all Python limitations on what can be used as a 
    dictionary key apply to the name fields.

    Leading and trailing whitespace is stripped from all names and values, 
    and all values are returned as strings.

\setlength{\parskip}{1ex}
      \textbf{Parameters}
      \vspace{-1ex}

      \begin{quote}
        \begin{Ventry}{xxxxxxx}

          \item[nvf]

          an open file object from which to read the name value pairs

            {\it (type=file)}

          \item[comment]

          character which begins comments

            {\it (type=str)}

        \end{Ventry}

      \end{quote}

      \textbf{Return Value}
    \vspace{-1ex}

      \begin{quote}
      a dictionary containing the name value pairs read in from 
      \texttt{nvf}.

      {\it (type=dict)}

      \end{quote}

      \textbf{Raises}
    \vspace{-1ex}

      \begin{quote}
        \begin{description}

          \item[\texttt{NameValueFileParseError}]

          if a non-comment input line does not contain an '=' character, or
          if a non-comment line has nothing but whitespace preceeding or 
          following the '=' character.

          \item[\texttt{NameValueFileDuplicateNameError}]

          if more than one instance of the same name is found in the input 
          file \texttt{nvf}.

        \end{description}

      \end{quote}

    \end{boxedminipage}

    \label{SatStress:SatStress:test}
    \index{SatStress \textit{(package)}!SatStress.SatStress \textit{(module)}!SatStress.SatStress.test \textit{(function)}}

    \vspace{0.5ex}

\hspace{.8\funcindent}\begin{boxedminipage}{\funcwidth}

    \raggedright \textbf{test}(\textit{argv}={\tt \texttt{[}\texttt{'}\texttt{(imported)}\texttt{'}\texttt{]}})

    \vspace{-1.5ex}

    \rule{\textwidth}{0.5\fboxrule}
\setlength{\parskip}{2ex}
    Check to see that SatStress gives the expected output from a series of 
    known calculations.

    Calculates the stresses due to the \texttt{NSR} and \texttt{Diurnal} 
    forcings at a series of lat lon points on Europa, over the course of 
    most of an orbit, and also at a variety of different amounts of viscous
    relaxation.  Compares the calculated values to those listed in the 
    \texttt{pickle} file passed in via the command line.

    \texttt{test} is called from the \texttt{SatStress Makefile}, when one 
    does \texttt{make test}, with the appropriate \texttt{pickle}d input to
    compare against (it is provided with the source code).

    This function also acts as a short demonstration of how to use the 
    SatStress module.

\setlength{\parskip}{1ex}
      \textbf{Parameters}
      \vspace{-1ex}

      \begin{quote}
        \begin{Ventry}{xxxx}

          \item[argv]

          a list of command line arguments

            {\it (type=list)}

        \end{Ventry}

      \end{quote}

      \textbf{Return Value}
    \vspace{-1ex}

      \begin{quote}
      \texttt{True} if the test fails. \texttt{False} if the test passes.

      {\it (type=bool)}

      \end{quote}

    \end{boxedminipage}


%%%%%%%%%%%%%%%%%%%%%%%%%%%%%%%%%%%%%%%%%%%%%%%%%%%%%%%%%%%%%%%%%%%%%%%%%%%
%%                           Class Description                           %%
%%%%%%%%%%%%%%%%%%%%%%%%%%%%%%%%%%%%%%%%%%%%%%%%%%%%%%%%%%%%%%%%%%%%%%%%%%%

    \index{SatStress \textit{(package)}!SatStress.SatStress \textit{(module)}!SatStress.SatStress.Satellite \textit{(class)}|(}
\subsection{Class Satellite}

    \label{SatStress:SatStress:Satellite}
\begin{tabular}{cccccc}
% Line for object, linespec=[False]
\multicolumn{2}{r}{\settowidth{\BCL}{object}\multirow{2}{\BCL}{object}}
&&
  \\\cline{3-3}
  &&\multicolumn{1}{c|}{}
&&
  \\
&&\multicolumn{2}{l}{\textbf{SatStress.SatStress.Satellite}}
\end{tabular}

An object describing the physical structure and context of a satellite.

Defines a satellite's material properties, internal structure, orbital 
context, and the tidal forcings to which it is subjected.


%%%%%%%%%%%%%%%%%%%%%%%%%%%%%%%%%%%%%%%%%%%%%%%%%%%%%%%%%%%%%%%%%%%%%%%%%%%
%%                                Methods                                %%
%%%%%%%%%%%%%%%%%%%%%%%%%%%%%%%%%%%%%%%%%%%%%%%%%%%%%%%%%%%%%%%%%%%%%%%%%%%

  \subsubsection{Methods}

    \vspace{0.5ex}

\hspace{.8\funcindent}\begin{boxedminipage}{\funcwidth}

    \raggedright \textbf{\_\_init\_\_}(\textit{self}, \textit{satFile})

    \vspace{-1.5ex}

    \rule{\textwidth}{0.5\fboxrule}
\setlength{\parskip}{2ex}
    Construct a Satellite object from a satFile

    (section) Required input file parameters:

      The Satellite is initialized from a name value file (as described 
      under \texttt{nvf2dict}).  The file must define the following 
      parameters, all of which are specified in SI (MKS) units.

      \begin{itemize}
      \setlength{\parskip}{0.6ex}
        \item \textbf{\texttt{SYSTEM\_ID}}:  A string identifying the planetary
          system, e.g. JupiterEuropa.

        \item \textbf{\texttt{PLANET\_MASS}}:  The mass of the planet the 
          satellite orbits [kg].

        \item \textbf{\texttt{ORBIT\_ECCENTRICITY}}:  The eccentricity of the 
          satellite's orbit.  Must not exceed 0.25.

        \item \textbf{\texttt{ORBIT\_SEMIMAJOR\_AXIS}}:  The semimajor axis of 
          the satellite's orbit [m].

        \item \textbf{\texttt{NSR\_PERIOD}}:  The time it takes for the 
          satellite's icy shell to undergo one full rotation [s].  If you 
          don't want to have any NSR stresses, just put INFINITY here.

        \item \textbf{\texttt{LOVE\_PATH}}:  The path to the program used to 
          calculate the frequency-dependent degree-2 Love numbers.  
          Currently this code is one provided by John Wahr.

      \end{itemize}

\setlength{\parskip}{1ex}
      \textbf{Parameters}
      \vspace{-1ex}

      \begin{quote}
        \begin{Ventry}{xxxxxxx}

          \item[satFile]

          Open file object containing name value pairs specifying the 
          satellite's internal structure and orbital context, and the tidal
          forcings to which the satellite is subjected.

            {\it (type=file)}

        \end{Ventry}

      \end{quote}

      \textbf{Return Value}
    \vspace{-1ex}

      \begin{quote}
      a Satellite object corresponding to the proffered input file.

      {\it (type=\texttt{Satellite})}

      \end{quote}

      \textbf{Raises}
    \vspace{-1ex}

      \begin{quote}
        \begin{description}

          \item[\texttt{NameValueFileError}]

          if parsing of the input file fails.

          \item[\texttt{MissingSatelliteParamError}]

          if a required input field is not found within the input file.

          \item[\texttt{NonNumberSatelliteParamError}]

          if a required input which is of a numeric type is found within 
          the file, but its value is not convertable to a float.

          \item[\texttt{LoveLayerNumberError}]

          if the number of layers specified is not exactly 4.

          \item[\texttt{GravitationallyUnstableSatelliteError}]

          if the layers are found not to decrease in density from the core 
          to the surface.

          \item[\texttt{ExcessiveSatelliteMassError}]

          if the mass of the satellite's parent planet is not at least 10 
          times larger than the mass of the satellite.

          \item[\texttt{LargeEccentricityError}]

          if the satellite's orbital eccentricity is greater than 0.25

          \item[\texttt{NegativeNSRPeriodError}]

          if the NSR period of the satellite is less than zero.

          \item[\texttt{IOError}]

          if the file specified in \texttt{LOVE\_PATH} is not openable.

        \end{description}

      \end{quote}

      Overrides: object.\_\_init\_\_

    \end{boxedminipage}

    \label{SatStress:SatStress:Satellite:mass}
    \index{SatStress \textit{(package)}!SatStress.SatStress \textit{(module)}!SatStress.SatStress.Satellite \textit{(class)}!SatStress.SatStress.Satellite.mass \textit{(method)}}

    \vspace{0.5ex}

\hspace{.8\funcindent}\begin{boxedminipage}{\funcwidth}

    \raggedright \textbf{mass}(\textit{self})

    \vspace{-1.5ex}

    \rule{\textwidth}{0.5\fboxrule}
\setlength{\parskip}{2ex}
    Calculate the mass of the satellite. (the sum of the layer masses)

\setlength{\parskip}{1ex}
    \end{boxedminipage}

    \label{SatStress:SatStress:Satellite:radius}
    \index{SatStress \textit{(package)}!SatStress.SatStress \textit{(module)}!SatStress.SatStress.Satellite \textit{(class)}!SatStress.SatStress.Satellite.radius \textit{(method)}}

    \vspace{0.5ex}

\hspace{.8\funcindent}\begin{boxedminipage}{\funcwidth}

    \raggedright \textbf{radius}(\textit{self})

    \vspace{-1.5ex}

    \rule{\textwidth}{0.5\fboxrule}
\setlength{\parskip}{2ex}
    Calculate the radius of the satellite (the sum of the layer 
    thicknesses).

\setlength{\parskip}{1ex}
    \end{boxedminipage}

    \label{SatStress:SatStress:Satellite:density}
    \index{SatStress \textit{(package)}!SatStress.SatStress \textit{(module)}!SatStress.SatStress.Satellite \textit{(class)}!SatStress.SatStress.Satellite.density \textit{(method)}}

    \vspace{0.5ex}

\hspace{.8\funcindent}\begin{boxedminipage}{\funcwidth}

    \raggedright \textbf{density}(\textit{self})

    \vspace{-1.5ex}

    \rule{\textwidth}{0.5\fboxrule}
\setlength{\parskip}{2ex}
    Calculate the mean density of the satellite in [kg 
    m{\textasciicircum}-3].

\setlength{\parskip}{1ex}
    \end{boxedminipage}

    \label{SatStress:SatStress:Satellite:surface_gravity}
    \index{SatStress \textit{(package)}!SatStress.SatStress \textit{(module)}!SatStress.SatStress.Satellite \textit{(class)}!SatStress.SatStress.Satellite.surface\_gravity \textit{(method)}}

    \vspace{0.5ex}

\hspace{.8\funcindent}\begin{boxedminipage}{\funcwidth}

    \raggedright \textbf{surface\_gravity}(\textit{self})

    \vspace{-1.5ex}

    \rule{\textwidth}{0.5\fboxrule}
\setlength{\parskip}{2ex}
    Calculate the satellite's surface gravitational acceleration in [m 
    s{\textasciicircum}-2].

\setlength{\parskip}{1ex}
    \end{boxedminipage}

    \label{SatStress:SatStress:Satellite:orbit_period}
    \index{SatStress \textit{(package)}!SatStress.SatStress \textit{(module)}!SatStress.SatStress.Satellite \textit{(class)}!SatStress.SatStress.Satellite.orbit\_period \textit{(method)}}

    \vspace{0.5ex}

\hspace{.8\funcindent}\begin{boxedminipage}{\funcwidth}

    \raggedright \textbf{orbit\_period}(\textit{self})

    \vspace{-1.5ex}

    \rule{\textwidth}{0.5\fboxrule}
\setlength{\parskip}{2ex}
    Calculate the satellite's Keplerian orbital period in seconds.

\setlength{\parskip}{1ex}
    \end{boxedminipage}

    \label{SatStress:SatStress:Satellite:mean_motion}
    \index{SatStress \textit{(package)}!SatStress.SatStress \textit{(module)}!SatStress.SatStress.Satellite \textit{(class)}!SatStress.SatStress.Satellite.mean\_motion \textit{(method)}}

    \vspace{0.5ex}

\hspace{.8\funcindent}\begin{boxedminipage}{\funcwidth}

    \raggedright \textbf{mean\_motion}(\textit{self})

    \vspace{-1.5ex}

    \rule{\textwidth}{0.5\fboxrule}
\setlength{\parskip}{2ex}
    Calculate the orbital mean motion of the satellite [rad 
    s{\textasciicircum}-1].

\setlength{\parskip}{1ex}
    \end{boxedminipage}

    \vspace{0.5ex}

\hspace{.8\funcindent}\begin{boxedminipage}{\funcwidth}

    \raggedright \textbf{\_\_str\_\_}(\textit{self})

    \vspace{-1.5ex}

    \rule{\textwidth}{0.5\fboxrule}
\setlength{\parskip}{2ex}
    Output a satellite definition file equivalent to the object.

\setlength{\parskip}{1ex}
      Overrides: object.\_\_str\_\_

    \end{boxedminipage}


\large{\textbf{\textit{Inherited from object}}}

\begin{quote}
\_\_delattr\_\_(), \_\_getattribute\_\_(), \_\_hash\_\_(), \_\_new\_\_(), \_\_reduce\_\_(), \_\_reduce\_ex\_\_(), \_\_repr\_\_(), \_\_setattr\_\_()
\end{quote}

%%%%%%%%%%%%%%%%%%%%%%%%%%%%%%%%%%%%%%%%%%%%%%%%%%%%%%%%%%%%%%%%%%%%%%%%%%%
%%                              Properties                               %%
%%%%%%%%%%%%%%%%%%%%%%%%%%%%%%%%%%%%%%%%%%%%%%%%%%%%%%%%%%%%%%%%%%%%%%%%%%%

  \subsubsection{Properties}

    \vspace{-1cm}
\hspace{\varindent}\begin{longtable}{|p{\varnamewidth}|p{\vardescrwidth}|l}
\cline{1-2}
\cline{1-2} \centering \textbf{Name} & \centering \textbf{Description}& \\
\cline{1-2}
\endhead\cline{1-2}\multicolumn{3}{r}{\small\textit{continued on next page}}\\\endfoot\cline{1-2}
\endlastfoot\multicolumn{2}{|l|}{\textit{Inherited from object}}\\
\multicolumn{2}{|p{\varwidth}|}{\raggedright \_\_class\_\_}\\
\cline{1-2}
\end{longtable}


%%%%%%%%%%%%%%%%%%%%%%%%%%%%%%%%%%%%%%%%%%%%%%%%%%%%%%%%%%%%%%%%%%%%%%%%%%%
%%                          Instance Variables                           %%
%%%%%%%%%%%%%%%%%%%%%%%%%%%%%%%%%%%%%%%%%%%%%%%%%%%%%%%%%%%%%%%%%%%%%%%%%%%

  \subsubsection{Instance Variables}

    \vspace{-1cm}
\hspace{\varindent}\begin{longtable}{|p{\varnamewidth}|p{\vardescrwidth}|l}
\cline{1-2}
\cline{1-2} \centering \textbf{Name} & \centering \textbf{Description}& \\
\cline{1-2}
\endhead\cline{1-2}\multicolumn{3}{r}{\small\textit{continued on next page}}\\\endfoot\cline{1-2}
\endlastfoot\raggedright l\-a\-y\-e\-r\-s\- & \raggedright a list of \texttt{SatLayer} objects, describing the layers making
          up the satellite.  The layers are ordered from the center of the 
          satellite outward, with layers[0] corresponding to the core.

            {\it (type=list)}&\\
\cline{1-2}
\raggedright l\-o\-v\-e\-\_\-p\-a\-t\-h\- & \raggedright the path to the program which will be used to calculate the 
          degree-2, complex, frequency dependent Love numbers h2, and k2.  
          Path is relative to the directory in which the program is 
          running.  Corresponds to \texttt{LOVE\_PATH} in the input file.

            {\it (type=str)}&\\
\cline{1-2}
\raggedright n\-s\-r\-\_\-p\-e\-r\-i\-o\-d\- & \raggedright the time it takes for the decoupled ice shell to complete one 
          full rotation [s], corresponds to \texttt{NSR\_PERIOD} in the 
          input file.  May also be set to infinity (inf, infinity, INF, 
          INFINITY).

            {\it (type=float)}&\\
\cline{1-2}
\raggedright n\-u\-m\-\_\-l\-a\-y\-e\-r\-s\- & \raggedright the number of layers making up the satellite, as indicated by the
          number of keys within the satParams dictionary contain the string
          'LAYER\_ID'.  Currently this must equal 4.

            {\it (type=int)}&\\
\cline{1-2}
\raggedright o\-r\-b\-i\-t\-\_\-e\-c\-c\-e\-n\-t\-r\-i\-c\-i\-t\-y\- & \raggedright the satellite's orbital eccentricity, corresponds to 
          \texttt{ORBIT\_ECCENTRICITY} in the input file.

            {\it (type=float)}&\\
\cline{1-2}
\raggedright o\-r\-b\-i\-t\-\_\-s\-e\-m\-i\-m\-a\-j\-o\-r\-\_\-a\-x\-i\-s\- & \raggedright semimajor axis of the satellite's orbit [m], corresponds to 
          \texttt{ORBIT\_SEMIMAJOR\_AXIS} in the input file.

            {\it (type=float)}&\\
\cline{1-2}
\raggedright p\-l\-a\-n\-e\-t\-\_\-m\-a\-s\-s\- & \raggedright the mass of the planet the satellite orbits [kg], corresponds to 
          \texttt{PLANET\_MASS} in the input file.

            {\it (type=float)}&\\
\cline{1-2}
\raggedright s\-a\-t\-P\-a\-r\-a\-m\-s\- & \raggedright dictionary containing the name value pairs read in from the input
          file.

            {\it (type=dict)}&\\
\cline{1-2}
\raggedright s\-o\-u\-r\-c\-e\-f\-i\-l\-e\- & \raggedright the file object which was read in order to create the 
          \texttt{Satellite} instance.

            {\it (type=file)}&\\
\cline{1-2}
\raggedright s\-y\-s\-t\-e\-m\-\_\-i\-d\- & \raggedright string identifying the planet/satellite system, corresponds to 
          \texttt{SYSTEM\_ID} in the input file.

            {\it (type=str)}&\\
\cline{1-2}
\end{longtable}

    \index{SatStress \textit{(package)}!SatStress.SatStress \textit{(module)}!SatStress.SatStress.Satellite \textit{(class)}|)}

%%%%%%%%%%%%%%%%%%%%%%%%%%%%%%%%%%%%%%%%%%%%%%%%%%%%%%%%%%%%%%%%%%%%%%%%%%%
%%                           Class Description                           %%
%%%%%%%%%%%%%%%%%%%%%%%%%%%%%%%%%%%%%%%%%%%%%%%%%%%%%%%%%%%%%%%%%%%%%%%%%%%

    \index{SatStress \textit{(package)}!SatStress.SatStress \textit{(module)}!SatStress.SatStress.SatLayer \textit{(class)}|(}
\subsection{Class SatLayer}

    \label{SatStress:SatStress:SatLayer}
\begin{tabular}{cccccc}
% Line for object, linespec=[False]
\multicolumn{2}{r}{\settowidth{\BCL}{object}\multirow{2}{\BCL}{object}}
&&
  \\\cline{3-3}
  &&\multicolumn{1}{c|}{}
&&
  \\
&&\multicolumn{2}{l}{\textbf{SatStress.SatStress.SatLayer}}
\end{tabular}

An object describing a uniform material layer within a satellite.

Note that a layer by itself has no knowledge of where within the satellite 
it resides.  That information is contained in the ordering of the list of 
layers within the satellite object.


%%%%%%%%%%%%%%%%%%%%%%%%%%%%%%%%%%%%%%%%%%%%%%%%%%%%%%%%%%%%%%%%%%%%%%%%%%%
%%                                Methods                                %%
%%%%%%%%%%%%%%%%%%%%%%%%%%%%%%%%%%%%%%%%%%%%%%%%%%%%%%%%%%%%%%%%%%%%%%%%%%%

  \subsubsection{Methods}

    \vspace{0.5ex}

\hspace{.8\funcindent}\begin{boxedminipage}{\funcwidth}

    \raggedright \textbf{\_\_init\_\_}(\textit{self}, \textit{sat}, \textit{layer\_n}={\tt 0})

    \vspace{-1.5ex}

    \rule{\textwidth}{0.5\fboxrule}
\setlength{\parskip}{2ex}
    Construct an object representing a layer within a \texttt{Satellite}.

    Gets values from the \texttt{Satellite.satParams} dictionary for the 
    layer that corresponds to the value of \texttt{layer\_n}.

    Each layer is defined by seven parameter values, and each layer has a 
    unique numeric identifier, appended to the end of all the names of its 
    parameters.  Layer zero is the core, with the number increasing as the 
    satellite is built up toward the surface.  In the below list the "N" at
    the end of the parameter names should be replaced with the number of 
    the layer (\texttt{layer\_n}).  Currently, because of the constraints 
    of the Love number code that we are using, you must specify 4 layers 
    (\texttt{CORE}, \texttt{OCEAN}, \texttt{ICE\_LOWER}, 
    \texttt{ICE\_UPPER}).

    \begin{itemize}
    \setlength{\parskip}{0.6ex}
      \item \textbf{\texttt{LAYER\_ID\_N}}: A string identifying the layer, 
        e.g. \texttt{OCEAN} or \texttt{ICE\_LOWER}

      \item \textbf{\texttt{DENSITY\_N}}: The density of the layer at zero 
        pressure [m kg{\textasciicircum}-3]

      \item \textbf{\texttt{LAME\_MU\_N}}: The real-valued Lame parameter 
        \(\mu\) (shear modulus) [Pa]

      \item \textbf{\texttt{LAME\_LAMBDA\_N}}: The real-valued Lame parameter 
        \(\lambda\) [Pa]

      \item \textbf{\texttt{THICKNESS\_N}}: The thickness of the layer [m]

      \item \textbf{\texttt{VISCOSITY\_N}}: The viscosity of the layer [Pa s]

      \item \textbf{\texttt{TENSILE\_STRENGTH\_N}}: The tensile strength of the
        layer [Pa]

    \end{itemize}

    Not all layers necessarily require all parameters in order for the 
    calculation to succeed, but it is required that they be provided. 
    Parameters that will currently be ignored:

    \begin{itemize}
    \setlength{\parskip}{0.6ex}
      \item \textbf{\texttt{TENSILE\_STRENGTH}} of all layers except for the 
        surface (which is only used when creating fractures).

      \item \textbf{\texttt{VISCOSITY}} of the ocean and the core.

      \item \textbf{\texttt{LAME\_MU}} of the ocean, assumed to be zero.

    \end{itemize}

\setlength{\parskip}{1ex}
      \textbf{Parameters}
      \vspace{-1ex}

      \begin{quote}
        \begin{Ventry}{xxxxxxx}

          \item[sat]

          the \texttt{Satellite} object to which the layer belongs.

            {\it (type=\texttt{Satellite})}

          \item[layer\_n]

          layer\_n indicates which layer in the satellite is being defined,
          with n=0 indicating the center of the satellite, and increasing 
          outward.  This is needed in order to select the appropriate 
          values from the \texttt{Satellite.satParams} dictionary.

            {\it (type=int)}

        \end{Ventry}

      \end{quote}

      \textbf{Return Value}
    \vspace{-1ex}

      \begin{quote}
      a \texttt{SatLayer} object having the properties specified in the

      {\it (type=\texttt{SatLayer})}

      \end{quote}

      \textbf{Raises}
    \vspace{-1ex}

      \begin{quote}
        \begin{description}

          \item[\texttt{MissingSatelliteParamError}]

          if any of the seven input parameters listed above is not found in
          the \texttt{Satellite.satParams} dictionary belonging to the 
          \texttt{sat} object passed in.

          \item[\texttt{NonNumberSatelliteParamError}]

          if any numeric instance variable list above has a value that 
          cannot be converted to a float.

          \item[\texttt{LowLayerDensityError}]

          if a layer is specififed with a density of less than 100 [kg 
          m{\textasciicircum}-3]

          \item[\texttt{LowLayerThicknessError}]

          if a layer is specified with a thickness of less than 100 [m].

          \item[\texttt{NegativeLayerParamError}]

          if either of the Lame parameters, the viscosity, or the tensile 
          strength of a layer is less than zero.

        \end{description}

      \end{quote}

      Overrides: object.\_\_init\_\_

    \end{boxedminipage}

    \vspace{0.5ex}

\hspace{.8\funcindent}\begin{boxedminipage}{\funcwidth}

    \raggedright \textbf{\_\_str\_\_}(\textit{self})

    \vspace{-1.5ex}

    \rule{\textwidth}{0.5\fboxrule}
\setlength{\parskip}{2ex}
    Output a human and machine readable text description of the layer.

    Note that because the layer object does not know explicitly where it is
    within the stratified Satellite (that information is contained in the 
    ordering of Satellite.layers list), this method cannot be used in the 
    output of a Satellite object.

\setlength{\parskip}{1ex}
      Overrides: object.\_\_str\_\_

    \end{boxedminipage}

    \label{SatStress:SatStress:SatLayer:maxwell_time}
    \index{SatStress \textit{(package)}!SatStress.SatStress \textit{(module)}!SatStress.SatStress.SatLayer \textit{(class)}!SatStress.SatStress.SatLayer.maxwell\_time \textit{(method)}}

    \vspace{0.5ex}

\hspace{.8\funcindent}\begin{boxedminipage}{\funcwidth}

    \raggedright \textbf{maxwell\_time}(\textit{self})

    \vspace{-1.5ex}

    \rule{\textwidth}{0.5\fboxrule}
\setlength{\parskip}{2ex}
    Calculate the Maxwell relaxation time of the layer [s] 
    (viscosity/lame\_mu).

\setlength{\parskip}{1ex}
    \end{boxedminipage}

    \label{SatStress:SatStress:SatLayer:bulk_modulus}
    \index{SatStress \textit{(package)}!SatStress.SatStress \textit{(module)}!SatStress.SatStress.SatLayer \textit{(class)}!SatStress.SatStress.SatLayer.bulk\_modulus \textit{(method)}}

    \vspace{0.5ex}

\hspace{.8\funcindent}\begin{boxedminipage}{\funcwidth}

    \raggedright \textbf{bulk\_modulus}(\textit{self})

    \vspace{-1.5ex}

    \rule{\textwidth}{0.5\fboxrule}
\setlength{\parskip}{2ex}
    Calculate the bulk modulus (\(\kappa\)) of the layer [Pa].

\setlength{\parskip}{1ex}
    \end{boxedminipage}

    \label{SatStress:SatStress:SatLayer:youngs_modulus}
    \index{SatStress \textit{(package)}!SatStress.SatStress \textit{(module)}!SatStress.SatStress.SatLayer \textit{(class)}!SatStress.SatStress.SatLayer.youngs\_modulus \textit{(method)}}

    \vspace{0.5ex}

\hspace{.8\funcindent}\begin{boxedminipage}{\funcwidth}

    \raggedright \textbf{youngs\_modulus}(\textit{self})

    \vspace{-1.5ex}

    \rule{\textwidth}{0.5\fboxrule}
\setlength{\parskip}{2ex}
    Calculate the Young's modulus (E) of the layer [Pa].

\setlength{\parskip}{1ex}
    \end{boxedminipage}

    \label{SatStress:SatStress:SatLayer:poissons_ratio}
    \index{SatStress \textit{(package)}!SatStress.SatStress \textit{(module)}!SatStress.SatStress.SatLayer \textit{(class)}!SatStress.SatStress.SatLayer.poissons\_ratio \textit{(method)}}

    \vspace{0.5ex}

\hspace{.8\funcindent}\begin{boxedminipage}{\funcwidth}

    \raggedright \textbf{poissons\_ratio}(\textit{self})

    \vspace{-1.5ex}

    \rule{\textwidth}{0.5\fboxrule}
\setlength{\parskip}{2ex}
    Calculate poisson's ratio (\(\nu\)) of the layer [Pa].

\setlength{\parskip}{1ex}
    \end{boxedminipage}

    \label{SatStress:SatStress:SatLayer:p_wave_velocity}
    \index{SatStress \textit{(package)}!SatStress.SatStress \textit{(module)}!SatStress.SatStress.SatLayer \textit{(class)}!SatStress.SatStress.SatLayer.p\_wave\_velocity \textit{(method)}}

    \vspace{0.5ex}

\hspace{.8\funcindent}\begin{boxedminipage}{\funcwidth}

    \raggedright \textbf{p\_wave\_velocity}(\textit{self})

    \vspace{-1.5ex}

    \rule{\textwidth}{0.5\fboxrule}
\setlength{\parskip}{2ex}
    Calculate the velocity of a compression wave in the layer [m 
    s{\textasciicircum}-1]

\setlength{\parskip}{1ex}
    \end{boxedminipage}


\large{\textbf{\textit{Inherited from object}}}

\begin{quote}
\_\_delattr\_\_(), \_\_getattribute\_\_(), \_\_hash\_\_(), \_\_new\_\_(), \_\_reduce\_\_(), \_\_reduce\_ex\_\_(), \_\_repr\_\_(), \_\_setattr\_\_()
\end{quote}

%%%%%%%%%%%%%%%%%%%%%%%%%%%%%%%%%%%%%%%%%%%%%%%%%%%%%%%%%%%%%%%%%%%%%%%%%%%
%%                              Properties                               %%
%%%%%%%%%%%%%%%%%%%%%%%%%%%%%%%%%%%%%%%%%%%%%%%%%%%%%%%%%%%%%%%%%%%%%%%%%%%

  \subsubsection{Properties}

    \vspace{-1cm}
\hspace{\varindent}\begin{longtable}{|p{\varnamewidth}|p{\vardescrwidth}|l}
\cline{1-2}
\cline{1-2} \centering \textbf{Name} & \centering \textbf{Description}& \\
\cline{1-2}
\endhead\cline{1-2}\multicolumn{3}{r}{\small\textit{continued on next page}}\\\endfoot\cline{1-2}
\endlastfoot\multicolumn{2}{|l|}{\textit{Inherited from object}}\\
\multicolumn{2}{|p{\varwidth}|}{\raggedright \_\_class\_\_}\\
\cline{1-2}
\end{longtable}


%%%%%%%%%%%%%%%%%%%%%%%%%%%%%%%%%%%%%%%%%%%%%%%%%%%%%%%%%%%%%%%%%%%%%%%%%%%
%%                          Instance Variables                           %%
%%%%%%%%%%%%%%%%%%%%%%%%%%%%%%%%%%%%%%%%%%%%%%%%%%%%%%%%%%%%%%%%%%%%%%%%%%%

  \subsubsection{Instance Variables}

    \vspace{-1cm}
\hspace{\varindent}\begin{longtable}{|p{\varnamewidth}|p{\vardescrwidth}|l}
\cline{1-2}
\cline{1-2} \centering \textbf{Name} & \centering \textbf{Description}& \\
\cline{1-2}
\endhead\cline{1-2}\multicolumn{3}{r}{\small\textit{continued on next page}}\\\endfoot\cline{1-2}
\endlastfoot\raggedright d\-e\-n\-s\-i\-t\-y\- & \raggedright the density of the layer, at zero pressure [kg 
          m{\textasciicircum}-3],

            {\it (type=float)}&\\
\cline{1-2}
\raggedright l\-a\-m\-e\-\_\-l\-a\-m\-b\-d\-a\- & \raggedright the layer's Lame parameter, \(\lambda\) [Pa].

            {\it (type=float)}&\\
\cline{1-2}
\raggedright l\-a\-m\-e\-\_\-m\-u\- & \raggedright the layer's Lame parameter, \(\mu\) (the shear modulus) [Pa].

            {\it (type=float)}&\\
\cline{1-2}
\raggedright l\-a\-y\-e\-r\-\_\-i\-d\- & \raggedright a string identifying the layer, e.g. \texttt{CORE}, or 
          \texttt{ICE\_LOWER}

            {\it (type=str)}&\\
\cline{1-2}
\raggedright t\-e\-n\-s\-i\-l\-e\-\_\-s\-t\-r\- & \raggedright the tensile failure strength of the layer [Pa].

            {\it (type=float)}&\\
\cline{1-2}
\raggedright t\-h\-i\-c\-k\-n\-e\-s\-s\- & \raggedright the radial thickness of the layer [m].

            {\it (type=float)}&\\
\cline{1-2}
\raggedright v\-i\-s\-c\-o\-s\-i\-t\-y\- & \raggedright the viscosity of the layer [Pa s].

            {\it (type=float)}&\\
\cline{1-2}
\end{longtable}

    \index{SatStress \textit{(package)}!SatStress.SatStress \textit{(module)}!SatStress.SatStress.SatLayer \textit{(class)}|)}

%%%%%%%%%%%%%%%%%%%%%%%%%%%%%%%%%%%%%%%%%%%%%%%%%%%%%%%%%%%%%%%%%%%%%%%%%%%
%%                           Class Description                           %%
%%%%%%%%%%%%%%%%%%%%%%%%%%%%%%%%%%%%%%%%%%%%%%%%%%%%%%%%%%%%%%%%%%%%%%%%%%%

    \index{SatStress \textit{(package)}!SatStress.SatStress \textit{(module)}!SatStress.SatStress.LoveNum \textit{(class)}|(}
\subsection{Class LoveNum}

    \label{SatStress:SatStress:LoveNum}
\begin{tabular}{cccccc}
% Line for object, linespec=[False]
\multicolumn{2}{r}{\settowidth{\BCL}{object}\multirow{2}{\BCL}{object}}
&&
  \\\cline{3-3}
  &&\multicolumn{1}{c|}{}
&&
  \\
&&\multicolumn{2}{l}{\textbf{SatStress.SatStress.LoveNum}}
\end{tabular}

A container class for the complex Love numbers: h2, k2, and l2.


%%%%%%%%%%%%%%%%%%%%%%%%%%%%%%%%%%%%%%%%%%%%%%%%%%%%%%%%%%%%%%%%%%%%%%%%%%%
%%                                Methods                                %%
%%%%%%%%%%%%%%%%%%%%%%%%%%%%%%%%%%%%%%%%%%%%%%%%%%%%%%%%%%%%%%%%%%%%%%%%%%%

  \subsubsection{Methods}

    \vspace{0.5ex}

\hspace{.8\funcindent}\begin{boxedminipage}{\funcwidth}

    \raggedright \textbf{\_\_init\_\_}(\textit{self}, \textit{h2\_real}, \textit{h2\_imag}, \textit{k2\_real}, \textit{k2\_imag}, \textit{l2\_real}, \textit{l2\_imag})

    \vspace{-1.5ex}

    \rule{\textwidth}{0.5\fboxrule}
\setlength{\parskip}{2ex}
    Using the real and imaginary parts, create complex values.

\setlength{\parskip}{1ex}
      Overrides: object.\_\_init\_\_

    \end{boxedminipage}

    \vspace{0.5ex}

\hspace{.8\funcindent}\begin{boxedminipage}{\funcwidth}

    \raggedright \textbf{\_\_str\_\_}(\textit{self})

    \vspace{-1.5ex}

    \rule{\textwidth}{0.5\fboxrule}
\setlength{\parskip}{2ex}
    Return a human readable string representation of the Love numbers

\setlength{\parskip}{1ex}
      Overrides: object.\_\_str\_\_

    \end{boxedminipage}


\large{\textbf{\textit{Inherited from object}}}

\begin{quote}
\_\_delattr\_\_(), \_\_getattribute\_\_(), \_\_hash\_\_(), \_\_new\_\_(), \_\_reduce\_\_(), \_\_reduce\_ex\_\_(), \_\_repr\_\_(), \_\_setattr\_\_()
\end{quote}

%%%%%%%%%%%%%%%%%%%%%%%%%%%%%%%%%%%%%%%%%%%%%%%%%%%%%%%%%%%%%%%%%%%%%%%%%%%
%%                              Properties                               %%
%%%%%%%%%%%%%%%%%%%%%%%%%%%%%%%%%%%%%%%%%%%%%%%%%%%%%%%%%%%%%%%%%%%%%%%%%%%

  \subsubsection{Properties}

    \vspace{-1cm}
\hspace{\varindent}\begin{longtable}{|p{\varnamewidth}|p{\vardescrwidth}|l}
\cline{1-2}
\cline{1-2} \centering \textbf{Name} & \centering \textbf{Description}& \\
\cline{1-2}
\endhead\cline{1-2}\multicolumn{3}{r}{\small\textit{continued on next page}}\\\endfoot\cline{1-2}
\endlastfoot\multicolumn{2}{|l|}{\textit{Inherited from object}}\\
\multicolumn{2}{|p{\varwidth}|}{\raggedright \_\_class\_\_}\\
\cline{1-2}
\end{longtable}


%%%%%%%%%%%%%%%%%%%%%%%%%%%%%%%%%%%%%%%%%%%%%%%%%%%%%%%%%%%%%%%%%%%%%%%%%%%
%%                          Instance Variables                           %%
%%%%%%%%%%%%%%%%%%%%%%%%%%%%%%%%%%%%%%%%%%%%%%%%%%%%%%%%%%%%%%%%%%%%%%%%%%%

  \subsubsection{Instance Variables}

    \vspace{-1cm}
\hspace{\varindent}\begin{longtable}{|p{\varnamewidth}|p{\vardescrwidth}|l}
\cline{1-2}
\cline{1-2} \centering \textbf{Name} & \centering \textbf{Description}& \\
\cline{1-2}
\endhead\cline{1-2}\multicolumn{3}{r}{\small\textit{continued on next page}}\\\endfoot\cline{1-2}
\endlastfoot\raggedright h\-2\- & \raggedright the degree 2 complex, frequency dependent Love number h.

            {\it (type=complex)}&\\
\cline{1-2}
\raggedright k\-2\- & \raggedright the degree 2 complex, frequency dependent Love number k.

            {\it (type=complex)}&\\
\cline{1-2}
\raggedright l\-2\- & \raggedright the degree 2 complex, frequency dependent Love number l.

            {\it (type=complex)}&\\
\cline{1-2}
\end{longtable}

    \index{SatStress \textit{(package)}!SatStress.SatStress \textit{(module)}!SatStress.SatStress.LoveNum \textit{(class)}|)}

%%%%%%%%%%%%%%%%%%%%%%%%%%%%%%%%%%%%%%%%%%%%%%%%%%%%%%%%%%%%%%%%%%%%%%%%%%%
%%                           Class Description                           %%
%%%%%%%%%%%%%%%%%%%%%%%%%%%%%%%%%%%%%%%%%%%%%%%%%%%%%%%%%%%%%%%%%%%%%%%%%%%

    \index{SatStress \textit{(package)}!SatStress.SatStress \textit{(module)}!SatStress.SatStress.StressDef \textit{(class)}|(}
\subsection{Class StressDef}

    \label{SatStress:SatStress:StressDef}
\begin{tabular}{cccccc}
% Line for object, linespec=[False]
\multicolumn{2}{r}{\settowidth{\BCL}{object}\multirow{2}{\BCL}{object}}
&&
  \\\cline{3-3}
  &&\multicolumn{1}{c|}{}
&&
  \\
&&\multicolumn{2}{l}{\textbf{SatStress.SatStress.StressDef}}
\end{tabular}

\textbf{Known Subclasses:}
SatStress.SatStress.Diurnal,
    SatStress.SatStress.NSR

A base class from which particular tidal stress field objects descend.

Different tidal forcings are specified as sub-classes of this superclass 
(one for each separate forcing).

In the expressions of the stress fields, the time \textit{t} is specified 
in seconds, with zero occuring at periapse, in order to be compatible with 
the future inclusion of stressing mechanisms which may have explicit time 
dependence instead of being a function of the satellite's orbital position 
(e.g. a true polar wander trajectory).

Location is specified within a polar coordinate system having its origin at
the satellite's center of mass, using the following variables:

\begin{itemize}
\setlength{\parskip}{0.6ex}
  \item co-latitude (\(\theta\)): The arc separating a point on the surface of 
    the satellite from the north pole (0 {\textless} \(\theta\) {\textless}
    \(\pi\)).

  \item longitude (\(\phi\)): The arc separating the meridian of a point and 
    the meridian which passes under the average location of the primary 
    (planet) in the sky over the course of an orbit (0 {\textless} \(\phi\)
    {\textless} 2\(\pi\)). \textbf{East is taken as positive.}

\end{itemize}

Each subclass must define its own version of the three components of the 
membrane stress tensor, \texttt{Ttt}, \texttt{Tpp}, and \texttt{Tpt} (the 
north-south, east-west, and shear stress components) as methods.


%%%%%%%%%%%%%%%%%%%%%%%%%%%%%%%%%%%%%%%%%%%%%%%%%%%%%%%%%%%%%%%%%%%%%%%%%%%
%%                                Methods                                %%
%%%%%%%%%%%%%%%%%%%%%%%%%%%%%%%%%%%%%%%%%%%%%%%%%%%%%%%%%%%%%%%%%%%%%%%%%%%

  \subsubsection{Methods}

    \label{SatStress:SatStress:StressDef:calcLove}
    \index{SatStress \textit{(package)}!SatStress.SatStress \textit{(module)}!SatStress.SatStress.StressDef \textit{(class)}!SatStress.SatStress.StressDef.calcLove \textit{(method)}}

    \vspace{0.5ex}

\hspace{.8\funcindent}\begin{boxedminipage}{\funcwidth}

    \raggedright \textbf{calcLove}(\textit{self})

    \vspace{-1.5ex}

    \rule{\textwidth}{0.5\fboxrule}
\setlength{\parskip}{2ex}
    Calculate the Love numbers for the satellite and the given forcing.

    If an infinite forcing period is given, return zero valued Love 
    numbers.

    This is a wrapper function, which can be used to call different Love 
    number codes in the future.

\setlength{\parskip}{1ex}
      \textbf{Raises}
    \vspace{-1ex}

      \begin{quote}
        \begin{description}

          \item[\texttt{InvalidLoveNumberError}]

          if the magnitude of the imaginary part of any Love number is 
          larger than its real part, if the real part is ever less than 
          zero, or if the real coefficient of the imaginary part is ever 
          positive.

        \end{description}

      \end{quote}

    \end{boxedminipage}

    \label{SatStress:SatStress:StressDef:calcLoveInfinitePeriod}
    \index{SatStress \textit{(package)}!SatStress.SatStress \textit{(module)}!SatStress.SatStress.StressDef \textit{(class)}!SatStress.SatStress.StressDef.calcLoveInfinitePeriod \textit{(method)}}

    \vspace{0.5ex}

\hspace{.8\funcindent}\begin{boxedminipage}{\funcwidth}

    \raggedright \textbf{calcLoveInfinitePeriod}(\textit{self})

    \vspace{-1.5ex}

    \rule{\textwidth}{0.5\fboxrule}
\setlength{\parskip}{2ex}
    Return a set of zero Love numbers constructed statically.

    This method is included so we don't have to worry about whether the 
    Love number code can deal with being given an infinite period.  All 
    stresses will relax to zero with an infinite period (since the shear 
    modulus \(\mu\) goes to zero), so it doesn't really matter what we set 
    the Love numbers to here.

\setlength{\parskip}{1ex}
    \end{boxedminipage}

    \label{SatStress:SatStress:StressDef:calcLoveWahr4LayerExternal}
    \index{SatStress \textit{(package)}!SatStress.SatStress \textit{(module)}!SatStress.SatStress.StressDef \textit{(class)}!SatStress.SatStress.StressDef.calcLoveWahr4LayerExternal \textit{(method)}}

    \vspace{0.5ex}

\hspace{.8\funcindent}\begin{boxedminipage}{\funcwidth}

    \raggedright \textbf{calcLoveWahr4LayerExternal}(\textit{self})

    \vspace{-1.5ex}

    \rule{\textwidth}{0.5\fboxrule}
\setlength{\parskip}{2ex}
    Use John Wahr's Love number code to calculate h, k, and l.

    This is done by an external program, written in Fortran by John Wahr 
    (and others), and called elsewhere on the system.  The path to this 
    external program is specified in the satellite definition file as the 
    parameter \texttt{LOVE\_PATH}.  At the moment, the code is fairly 
    limited in the kind of input it can take.  The specified satellite 
    must:

    \begin{itemize}
    \setlength{\parskip}{0.6ex}
      \item use a Maxwell rheology

      \item have a liquid water ocean underlying the ice shell

      \item have a 4-layer structure (ice\_upper, ice\_lower, ocean, core)

    \end{itemize}

    Eventually the Love number code will be more closely integrated with 
    this package, allowing more flexibility in the interior structure of 
    the satellite.

    A temporary directory named lovetmp-XXXXXXX (where the X's are a random
    hexadecimal number) is created in the current working directory, within
    which the Love number code is run.  The directory is deleted 
    immediately following the calculation.

\setlength{\parskip}{1ex}
      \textbf{Raises}
    \vspace{-1ex}

      \begin{quote}
        \begin{description}

          \item[\texttt{LoveExcessiveDeltaError}]

          if \texttt{StressDef.Delta}() {\textgreater} 
          10{\textasciicircum}9 for either of the ice layers.

        \end{description}

      \end{quote}

    \end{boxedminipage}

    \label{SatStress:SatStress:StressDef:Delta}
    \index{SatStress \textit{(package)}!SatStress.SatStress \textit{(module)}!SatStress.SatStress.StressDef \textit{(class)}!SatStress.SatStress.StressDef.Delta \textit{(method)}}

    \vspace{0.5ex}

\hspace{.8\funcindent}\begin{boxedminipage}{\funcwidth}

    \raggedright \textbf{Delta}(\textit{self}, \textit{layer\_n}={\tt -1})

    \vspace{-1.5ex}

    \rule{\textwidth}{0.5\fboxrule}
\setlength{\parskip}{2ex}
    Calculate \(\Delta\), a measure of how viscous the layer's response is.

\setlength{\parskip}{1ex}
      \textbf{Parameters}
      \vspace{-1ex}

      \begin{quote}
        \begin{Ventry}{xxxxxxx}

          \item[layer\_n]

          indicates which satellite layer Delta should be calculated for, 
          defaulting to the surface (recall that layer 0 is the core)

            {\it (type=int)}

        \end{Ventry}

      \end{quote}

      \textbf{Return Value}
    \vspace{-1ex}

      \begin{quote}
      \(\Delta\)= \(\mu\)/(\(\omega\)*\(\eta\))

      {\it (type=float)}

      \end{quote}

    \end{boxedminipage}

    \label{SatStress:SatStress:StressDef:Z}
    \index{SatStress \textit{(package)}!SatStress.SatStress \textit{(module)}!SatStress.SatStress.StressDef \textit{(class)}!SatStress.SatStress.StressDef.Z \textit{(method)}}

    \vspace{0.5ex}

\hspace{.8\funcindent}\begin{boxedminipage}{\funcwidth}

    \raggedright \textbf{Z}(\textit{self})

    \vspace{-1.5ex}

    \rule{\textwidth}{0.5\fboxrule}
\setlength{\parskip}{2ex}
    Calculate the value of Z, a constant that sits in front of many terms 
    in the potential defined by Wahr et al. (2008).

\setlength{\parskip}{1ex}
      \textbf{Return Value}
    \vspace{-1ex}

      \begin{quote}
      Z, a common constant in many of the Wahr et al. potential terms.

      {\it (type=float)}

      \end{quote}

    \end{boxedminipage}

    \label{SatStress:SatStress:StressDef:mu_twiddle}
    \index{SatStress \textit{(package)}!SatStress.SatStress \textit{(module)}!SatStress.SatStress.StressDef \textit{(class)}!SatStress.SatStress.StressDef.mu\_twiddle \textit{(method)}}

    \vspace{0.5ex}

\hspace{.8\funcindent}\begin{boxedminipage}{\funcwidth}

    \raggedright \textbf{mu\_twiddle}(\textit{self}, \textit{layer\_n}={\tt -1})

    \vspace{-1.5ex}

    \rule{\textwidth}{0.5\fboxrule}
\setlength{\parskip}{2ex}
    Calculate the frequency-dependent Lame parameter \(\mu\) for a Maxwell 
    rheology.

\setlength{\parskip}{1ex}
      \textbf{Parameters}
      \vspace{-1ex}

      \begin{quote}
        \begin{Ventry}{xxxxxxx}

          \item[layer\_n]

          number of layer for which we want to calculate \(\mu\), defaults 
          to the surface (with the core being layer zero).

        \end{Ventry}

      \end{quote}

      \textbf{Return Value}
    \vspace{-1ex}

      \begin{quote}
      the frequency-dependent Lame parameter \(\mu\) for a Maxwell rheology

      {\it (type=complex)}

      \end{quote}

    \end{boxedminipage}

    \label{SatStress:SatStress:StressDef:lambda_twiddle}
    \index{SatStress \textit{(package)}!SatStress.SatStress \textit{(module)}!SatStress.SatStress.StressDef \textit{(class)}!SatStress.SatStress.StressDef.lambda\_twiddle \textit{(method)}}

    \vspace{0.5ex}

\hspace{.8\funcindent}\begin{boxedminipage}{\funcwidth}

    \raggedright \textbf{lambda\_twiddle}(\textit{self}, \textit{layer\_n}={\tt -1})

    \vspace{-1.5ex}

    \rule{\textwidth}{0.5\fboxrule}
\setlength{\parskip}{2ex}
    Calculate the frequency-dependent Lame parameter \(\lambda\) for a 
    Maxwell rheology.

\setlength{\parskip}{1ex}
      \textbf{Parameters}
      \vspace{-1ex}

      \begin{quote}
        \begin{Ventry}{xxxxxxx}

          \item[layer\_n]

          number of layer for which we want to calculate \(\mu\), defaults 
          to the surface (with the core being layer zero).

        \end{Ventry}

      \end{quote}

      \textbf{Return Value}
    \vspace{-1ex}

      \begin{quote}
      the frequency-dependent Lame parameter \(\lambda\) for a Maxwell 
      rheology.

      {\it (type=complex)}

      \end{quote}

    \end{boxedminipage}

    \label{SatStress:SatStress:StressDef:alpha}
    \index{SatStress \textit{(package)}!SatStress.SatStress \textit{(module)}!SatStress.SatStress.StressDef \textit{(class)}!SatStress.SatStress.StressDef.alpha \textit{(method)}}

    \vspace{0.5ex}

\hspace{.8\funcindent}\begin{boxedminipage}{\funcwidth}

    \raggedright \textbf{alpha}(\textit{self})

    \vspace{-1.5ex}

    \rule{\textwidth}{0.5\fboxrule}
\setlength{\parskip}{2ex}
    Calculate the coefficient alpha twiddle for the surface layer (see Wahr
    et al. 2008).

\setlength{\parskip}{1ex}
      \textbf{Return Value}
    \vspace{-1ex}

      \begin{quote}
      Calculate the coefficient alpha twiddle for the surface layer (see 
      Wahr et al. 2008).

      {\it (type=complex)}

      \end{quote}

    \end{boxedminipage}

    \label{SatStress:SatStress:StressDef:Gamma}
    \index{SatStress \textit{(package)}!SatStress.SatStress \textit{(module)}!SatStress.SatStress.StressDef \textit{(class)}!SatStress.SatStress.StressDef.Gamma \textit{(method)}}

    \vspace{0.5ex}

\hspace{.8\funcindent}\begin{boxedminipage}{\funcwidth}

    \raggedright \textbf{Gamma}(\textit{self})

    \vspace{-1.5ex}

    \rule{\textwidth}{0.5\fboxrule}
\setlength{\parskip}{2ex}
    Calculate the coefficient capital Gamma twiddle for the surface layer 
    (see Wahr et al. 2008).

\setlength{\parskip}{1ex}
      \textbf{Return Value}
    \vspace{-1ex}

      \begin{quote}
      the coefficient capital Gamma twiddle for the surface layer (see Wahr
      et al. 2008).

      {\it (type=complex)}

      \end{quote}

    \end{boxedminipage}

    \label{SatStress:SatStress:StressDef:b1}
    \index{SatStress \textit{(package)}!SatStress.SatStress \textit{(module)}!SatStress.SatStress.StressDef \textit{(class)}!SatStress.SatStress.StressDef.b1 \textit{(method)}}

    \vspace{0.5ex}

\hspace{.8\funcindent}\begin{boxedminipage}{\funcwidth}

    \raggedright \textbf{b1}(\textit{self})

    \vspace{-1.5ex}

    \rule{\textwidth}{0.5\fboxrule}
\setlength{\parskip}{2ex}
    Calculate the coefficient beta one twiddle for the surface layer (see 
    Wahr et al. 2008).

\setlength{\parskip}{1ex}
      \textbf{Return Value}
    \vspace{-1ex}

      \begin{quote}
      the coefficient beta one twiddle for the surface layer (see Wahr et 
      al. 2008).

      {\it (type=complex)}

      \end{quote}

    \end{boxedminipage}

    \label{SatStress:SatStress:StressDef:g1}
    \index{SatStress \textit{(package)}!SatStress.SatStress \textit{(module)}!SatStress.SatStress.StressDef \textit{(class)}!SatStress.SatStress.StressDef.g1 \textit{(method)}}

    \vspace{0.5ex}

\hspace{.8\funcindent}\begin{boxedminipage}{\funcwidth}

    \raggedright \textbf{g1}(\textit{self})

    \vspace{-1.5ex}

    \rule{\textwidth}{0.5\fboxrule}
\setlength{\parskip}{2ex}
    Calculate the coefficient gamma one twiddle for the surface layer (see 
    Wahr et al. (2008)).

\setlength{\parskip}{1ex}
      \textbf{Return Value}
    \vspace{-1ex}

      \begin{quote}
      the coefficient gamma one twiddle for the surface layer (see Wahr et 
      al. (2008)).

      {\it (type=complex)}

      \end{quote}

    \end{boxedminipage}

    \label{SatStress:SatStress:StressDef:b2}
    \index{SatStress \textit{(package)}!SatStress.SatStress \textit{(module)}!SatStress.SatStress.StressDef \textit{(class)}!SatStress.SatStress.StressDef.b2 \textit{(method)}}

    \vspace{0.5ex}

\hspace{.8\funcindent}\begin{boxedminipage}{\funcwidth}

    \raggedright \textbf{b2}(\textit{self})

    \vspace{-1.5ex}

    \rule{\textwidth}{0.5\fboxrule}
\setlength{\parskip}{2ex}
    Calculate the coefficient beta two twiddle for the surface layer (see 
    Wahr et al. (2008)).

\setlength{\parskip}{1ex}
      \textbf{Return Value}
    \vspace{-1ex}

      \begin{quote}
      the coefficient beta two twiddle for the surface layer (see Wahr et 
      al. (2008)).

      {\it (type=complex)}

      \end{quote}

    \end{boxedminipage}

    \label{SatStress:SatStress:StressDef:g2}
    \index{SatStress \textit{(package)}!SatStress.SatStress \textit{(module)}!SatStress.SatStress.StressDef \textit{(class)}!SatStress.SatStress.StressDef.g2 \textit{(method)}}

    \vspace{0.5ex}

\hspace{.8\funcindent}\begin{boxedminipage}{\funcwidth}

    \raggedright \textbf{g2}(\textit{self})

    \vspace{-1.5ex}

    \rule{\textwidth}{0.5\fboxrule}
\setlength{\parskip}{2ex}
    Calculate the coefficient gamma two twiddle for the surface layer (see 
    Wahr et al. (2008)).

\setlength{\parskip}{1ex}
      \textbf{Return Value}
    \vspace{-1ex}

      \begin{quote}
      the coefficient gamma two twiddle for the surface layer (see Wahr et 
      al. (2008)).

      {\it (type=complex)}

      \end{quote}

    \end{boxedminipage}

    \label{SatStress:SatStress:StressDef:Ttt}
    \index{SatStress \textit{(package)}!SatStress.SatStress \textit{(module)}!SatStress.SatStress.StressDef \textit{(class)}!SatStress.SatStress.StressDef.Ttt \textit{(method)}}

    \vspace{0.5ex}

\hspace{.8\funcindent}\begin{boxedminipage}{\funcwidth}

    \raggedright \textbf{Ttt}(\textit{self}, \textit{theta}, \textit{phi}, \textit{t})

    \vspace{-1.5ex}

    \rule{\textwidth}{0.5\fboxrule}
\setlength{\parskip}{2ex}
    Calculates the \(\tau\)\_\(\theta\)\(\theta\) (north-south) component 
    of the stress tensor.

    In the base class, this is a purely virtual method - it must be defined
    by the subclasses that describe particular tidal stresses.

\setlength{\parskip}{1ex}
      \textbf{Parameters}
      \vspace{-1ex}

      \begin{quote}
        \begin{Ventry}{xxxxx}

          \item[theta]

          the co-latitude of the point at which to calculate the stress 
          [rad].

            {\it (type=float)}

          \item[phi]

          the east-positive longitude of the point at which to calculate 
          the stress [rad].

            {\it (type=float)}

          \item[t]

          the time, in seconds elapsed since pericenter, at which to 
          perform the stress calculation [s].

            {\it (type=float)}

        \end{Ventry}

      \end{quote}

      \textbf{Return Value}
    \vspace{-1ex}

      \begin{quote}
      the \(\tau\)\_\(\theta\)\(\theta\) component of the 2x2 membrane 
      stress tensor.

      {\it (type=float)}

      \end{quote}

    \end{boxedminipage}

    \label{SatStress:SatStress:StressDef:Tpp}
    \index{SatStress \textit{(package)}!SatStress.SatStress \textit{(module)}!SatStress.SatStress.StressDef \textit{(class)}!SatStress.SatStress.StressDef.Tpp \textit{(method)}}

    \vspace{0.5ex}

\hspace{.8\funcindent}\begin{boxedminipage}{\funcwidth}

    \raggedright \textbf{Tpp}(\textit{self}, \textit{theta}, \textit{phi}, \textit{t})

    \vspace{-1.5ex}

    \rule{\textwidth}{0.5\fboxrule}
\setlength{\parskip}{2ex}
    Calculates the \(\tau\)\_\(\phi\)\(\phi\) (east-west) component of the 
    stress tensor.

    In the base class, this is a purely virtual method - it must be defined
    by the subclasses that describe particular tidal stresses.

\setlength{\parskip}{1ex}
      \textbf{Parameters}
      \vspace{-1ex}

      \begin{quote}
        \begin{Ventry}{xxxxx}

          \item[theta]

          the co-latitude of the point at which to calculate the stress 
          [rad].

            {\it (type=float)}

          \item[phi]

          the east-positive longitude of the point at which to calculate 
          the stress [rad].

            {\it (type=float)}

          \item[t]

          the time, in seconds elapsed since pericenter, at which to 
          perform the stress calculation [s].

            {\it (type=float)}

        \end{Ventry}

      \end{quote}

      \textbf{Return Value}
    \vspace{-1ex}

      \begin{quote}
      the \(\tau\)\_\(\phi\)\(\phi\) component of the 2x2 membrane stress 
      tensor.

      {\it (type=float)}

      \end{quote}

    \end{boxedminipage}

    \label{SatStress:SatStress:StressDef:Tpt}
    \index{SatStress \textit{(package)}!SatStress.SatStress \textit{(module)}!SatStress.SatStress.StressDef \textit{(class)}!SatStress.SatStress.StressDef.Tpt \textit{(method)}}

    \vspace{0.5ex}

\hspace{.8\funcindent}\begin{boxedminipage}{\funcwidth}

    \raggedright \textbf{Tpt}(\textit{self}, \textit{theta}, \textit{phi}, \textit{t})

    \vspace{-1.5ex}

    \rule{\textwidth}{0.5\fboxrule}
\setlength{\parskip}{2ex}
    Calculates the \(\tau\)\_\(\phi\)\(\theta\) (off-diagonal) component of
    the stress tensor.

    In the base class, this is a purely virtual method - it must be defined
    by the subclasses that describe particular tidal stresses.

\setlength{\parskip}{1ex}
      \textbf{Parameters}
      \vspace{-1ex}

      \begin{quote}
        \begin{Ventry}{xxxxx}

          \item[theta]

          the co-latitude of the point at which to calculate the stress 
          [rad].

            {\it (type=float)}

          \item[phi]

          the east-positive longitude of the point at which to calculate 
          the stress [rad].

            {\it (type=float)}

          \item[t]

          the time in seconds elapsed since pericenter, at which to perform
          the stress calculation [s].

            {\it (type=float)}

        \end{Ventry}

      \end{quote}

      \textbf{Return Value}
    \vspace{-1ex}

      \begin{quote}
      the \(\tau\)\_\(\phi\)\(\theta\) component of the 2x2 membrane stress
      tensor.

      {\it (type=float)}

      \end{quote}

    \end{boxedminipage}


\large{\textbf{\textit{Inherited from object}}}

\begin{quote}
\_\_delattr\_\_(), \_\_getattribute\_\_(), \_\_hash\_\_(), \_\_init\_\_(), \_\_new\_\_(), \_\_reduce\_\_(), \_\_reduce\_ex\_\_(), \_\_repr\_\_(), \_\_setattr\_\_(), \_\_str\_\_()
\end{quote}

%%%%%%%%%%%%%%%%%%%%%%%%%%%%%%%%%%%%%%%%%%%%%%%%%%%%%%%%%%%%%%%%%%%%%%%%%%%
%%                              Properties                               %%
%%%%%%%%%%%%%%%%%%%%%%%%%%%%%%%%%%%%%%%%%%%%%%%%%%%%%%%%%%%%%%%%%%%%%%%%%%%

  \subsubsection{Properties}

    \vspace{-1cm}
\hspace{\varindent}\begin{longtable}{|p{\varnamewidth}|p{\vardescrwidth}|l}
\cline{1-2}
\cline{1-2} \centering \textbf{Name} & \centering \textbf{Description}& \\
\cline{1-2}
\endhead\cline{1-2}\multicolumn{3}{r}{\small\textit{continued on next page}}\\\endfoot\cline{1-2}
\endlastfoot\multicolumn{2}{|l|}{\textit{Inherited from object}}\\
\multicolumn{2}{|p{\varwidth}|}{\raggedright \_\_class\_\_}\\
\cline{1-2}
\end{longtable}


%%%%%%%%%%%%%%%%%%%%%%%%%%%%%%%%%%%%%%%%%%%%%%%%%%%%%%%%%%%%%%%%%%%%%%%%%%%
%%                            Class Variables                            %%
%%%%%%%%%%%%%%%%%%%%%%%%%%%%%%%%%%%%%%%%%%%%%%%%%%%%%%%%%%%%%%%%%%%%%%%%%%%

  \subsubsection{Class Variables}

    \vspace{-1cm}
\hspace{\varindent}\begin{longtable}{|p{\varnamewidth}|p{\vardescrwidth}|l}
\cline{1-2}
\cline{1-2} \centering \textbf{Name} & \centering \textbf{Description}& \\
\cline{1-2}
\endhead\cline{1-2}\multicolumn{3}{r}{\small\textit{continued on next page}}\\\endfoot\cline{1-2}
\endlastfoot\raggedright o\-m\-e\-g\-a\- & \raggedright the forcing frequency associated with the stress.

\textbf{Value:} 
{\tt 0.0}            {\it (type=float)}&\\
\cline{1-2}
\raggedright s\-a\-t\-e\-l\-l\-i\-t\-e\- & \raggedright the satellite which the stress is being applied to.

\textbf{Value:} 
{\tt None}            {\it (type=\texttt{Satellite})}&\\
\cline{1-2}
\raggedright l\-o\-v\-e\- & \raggedright the Love numbers which result from the given forcing frequency 
          and the specified satellite structure.

\textbf{Value:} 
{\tt LoveNum(0, 0, 0, 0, 0, 0)}            {\it (type=\texttt{LoveNum})}&\\
\cline{1-2}
\end{longtable}

    \index{SatStress \textit{(package)}!SatStress.SatStress \textit{(module)}!SatStress.SatStress.StressDef \textit{(class)}|)}

%%%%%%%%%%%%%%%%%%%%%%%%%%%%%%%%%%%%%%%%%%%%%%%%%%%%%%%%%%%%%%%%%%%%%%%%%%%
%%                           Class Description                           %%
%%%%%%%%%%%%%%%%%%%%%%%%%%%%%%%%%%%%%%%%%%%%%%%%%%%%%%%%%%%%%%%%%%%%%%%%%%%

    \index{SatStress \textit{(package)}!SatStress.SatStress \textit{(module)}!SatStress.SatStress.NSR \textit{(class)}|(}
\subsection{Class NSR}

    \label{SatStress:SatStress:NSR}
\begin{tabular}{cccccccc}
% Line for object, linespec=[False, False]
\multicolumn{2}{r}{\settowidth{\BCL}{object}\multirow{2}{\BCL}{object}}
&&
&&
  \\\cline{3-3}
  &&\multicolumn{1}{c|}{}
&&
&&
  \\
% Line for SatStress.SatStress.StressDef, linespec=[False]
\multicolumn{4}{r}{\settowidth{\BCL}{SatStress.SatStress.StressDef}\multirow{2}{\BCL}{SatStress.SatStress.StressDef}}
&&
  \\\cline{5-5}
  &&&&\multicolumn{1}{c|}{}
&&
  \\
&&&&\multicolumn{2}{l}{\textbf{SatStress.SatStress.NSR}}
\end{tabular}

An object defining the stress field which arises from the non-synchronous 
rotation (NSR) of a satellite's icy shell.

NSR is a subclass of \texttt{StressDef}.  See the derivation and detailed 
discussion of this stress field in in Wahr et al. (2008).


%%%%%%%%%%%%%%%%%%%%%%%%%%%%%%%%%%%%%%%%%%%%%%%%%%%%%%%%%%%%%%%%%%%%%%%%%%%
%%                                Methods                                %%
%%%%%%%%%%%%%%%%%%%%%%%%%%%%%%%%%%%%%%%%%%%%%%%%%%%%%%%%%%%%%%%%%%%%%%%%%%%

  \subsubsection{Methods}

    \vspace{0.5ex}

\hspace{.8\funcindent}\begin{boxedminipage}{\funcwidth}

    \raggedright \textbf{\_\_init\_\_}(\textit{self}, \textit{satellite})

    \vspace{-1.5ex}

    \rule{\textwidth}{0.5\fboxrule}
\setlength{\parskip}{2ex}
    Initialize the definition of the stresses due to NSR of the ice shell.

    The forcing frequency \(\omega\) is the frequency with which a point on
    the surface passes through a single hemisphere, because the NSR stress 
    field is degree 2 (that is, it's 2x the expected \(\omega\) from a full
    rotation)

    Because the core is not subject to the NSR forcing (it remains tidally 
    locked and synchronously rotating), all stresses within it are presumed
    to relax away, allowing it to deform into a tri-axial ellipsoid, with 
    its long axis pointing toward the parent planet.  In order to allow for
    this relaxation the shear modulus (\(\mu\)) of the core is set to an 
    artificially low value for the purpose of the Love number calculation. 
    This increases the magnitude of the radial deformation (and the Love 
    number h2) significantly.  See Wahr et al. (2008) for complete 
    discussion.

\setlength{\parskip}{1ex}
      \textbf{Parameters}
      \vspace{-1ex}

      \begin{quote}
        \begin{Ventry}{xxxxxxxxx}

          \item[satellite]

          the satellite to which the stress is being applied.

            {\it (type=\texttt{Satellite})}

        \end{Ventry}

      \end{quote}

      \textbf{Return Value}
    \vspace{-1ex}

      \begin{quote}
      an object defining the NSR stresses for a particular satellite.

      {\it (type=\texttt{NSR})}

      \end{quote}

      Overrides: object.\_\_init\_\_

    \end{boxedminipage}

    \vspace{0.5ex}

\hspace{.8\funcindent}\begin{boxedminipage}{\funcwidth}

    \raggedright \textbf{Ttt}(\textit{self}, \textit{theta}, \textit{phi}, \textit{t})

    \vspace{-1.5ex}

    \rule{\textwidth}{0.5\fboxrule}
\setlength{\parskip}{2ex}
    Calculates the \(\tau\)\_\(\theta\)\(\theta\) (north-south) component 
    of the stress tensor.

\setlength{\parskip}{1ex}
      \textbf{Parameters}
      \vspace{-1ex}

      \begin{quote}
        \begin{Ventry}{xxxxx}

          \item[theta]

          the co-latitude of the point at which to calculate the stress 
          [rad].

          \item[phi]

          the east-positive longitude of the point at which to calculate 
          the stress [rad].

          \item[t]

          the time, in seconds elapsed since pericenter, at which to 
          perform the stress calculation [s].

        \end{Ventry}

      \end{quote}

      \textbf{Return Value}
    \vspace{-1ex}

      \begin{quote}
      the \(\tau\)\_\(\theta\)\(\theta\) component of the 2x2 membrane 
      stress tensor.

      {\it (type=float)}

      \end{quote}

      Overrides: SatStress.SatStress.StressDef.Ttt

    \end{boxedminipage}

    \vspace{0.5ex}

\hspace{.8\funcindent}\begin{boxedminipage}{\funcwidth}

    \raggedright \textbf{Tpp}(\textit{self}, \textit{theta}, \textit{phi}, \textit{t})

    \vspace{-1.5ex}

    \rule{\textwidth}{0.5\fboxrule}
\setlength{\parskip}{2ex}
    Calculates the \(\tau\)\_\(\phi\)\(\phi\) (east-west) component of the 
    stress tensor.

\setlength{\parskip}{1ex}
      \textbf{Parameters}
      \vspace{-1ex}

      \begin{quote}
        \begin{Ventry}{xxxxx}

          \item[theta]

          the co-latitude of the point at which to calculate the stress 
          [rad].

          \item[phi]

          the east-positive longitude of the point at which to calculate 
          the stress [rad].

          \item[t]

          the time, in seconds elapsed since pericenter, at which to 
          perform the stress calculation [s].

        \end{Ventry}

      \end{quote}

      \textbf{Return Value}
    \vspace{-1ex}

      \begin{quote}
      the \(\tau\)\_\(\phi\)\(\phi\) component of the 2x2 membrane stress 
      tensor.

      {\it (type=float)}

      \end{quote}

      Overrides: SatStress.SatStress.StressDef.Tpp

    \end{boxedminipage}

    \vspace{0.5ex}

\hspace{.8\funcindent}\begin{boxedminipage}{\funcwidth}

    \raggedright \textbf{Tpt}(\textit{self}, \textit{theta}, \textit{phi}, \textit{t})

    \vspace{-1.5ex}

    \rule{\textwidth}{0.5\fboxrule}
\setlength{\parskip}{2ex}
    Calculates the \(\tau\)\_\(\phi\)\(\theta\) (off-diagonal) component of
    the stress tensor.

\setlength{\parskip}{1ex}
      \textbf{Parameters}
      \vspace{-1ex}

      \begin{quote}
        \begin{Ventry}{xxxxx}

          \item[theta]

          the co-latitude of the point at which to calculate the stress 
          [rad].

          \item[phi]

          the east-positive longitude of the point at which to calculate 
          the stress [rad].

          \item[t]

          the time in seconds elapsed since pericenter, at which to perform
          the stress calculation [s].

        \end{Ventry}

      \end{quote}

      \textbf{Return Value}
    \vspace{-1ex}

      \begin{quote}
      the \(\tau\)\_\(\phi\)\(\theta\) component of the 2x2 membrane stress
      tensor.

      {\it (type=float)}

      \end{quote}

      Overrides: SatStress.SatStress.StressDef.Tpt

    \end{boxedminipage}


\large{\textbf{\textit{Inherited from SatStress.SatStress.StressDef\textit{(Section \ref{SatStress:SatStress:StressDef})}}}}

\begin{quote}
Delta(), Gamma(), Z(), alpha(), b1(), b2(), calcLove(), calcLoveInfinitePeriod(), calcLoveWahr4LayerExternal(), g1(), g2(), lambda\_twiddle(), mu\_twiddle()
\end{quote}

\large{\textbf{\textit{Inherited from object}}}

\begin{quote}
\_\_delattr\_\_(), \_\_getattribute\_\_(), \_\_hash\_\_(), \_\_new\_\_(), \_\_reduce\_\_(), \_\_reduce\_ex\_\_(), \_\_repr\_\_(), \_\_setattr\_\_(), \_\_str\_\_()
\end{quote}

%%%%%%%%%%%%%%%%%%%%%%%%%%%%%%%%%%%%%%%%%%%%%%%%%%%%%%%%%%%%%%%%%%%%%%%%%%%
%%                              Properties                               %%
%%%%%%%%%%%%%%%%%%%%%%%%%%%%%%%%%%%%%%%%%%%%%%%%%%%%%%%%%%%%%%%%%%%%%%%%%%%

  \subsubsection{Properties}

    \vspace{-1cm}
\hspace{\varindent}\begin{longtable}{|p{\varnamewidth}|p{\vardescrwidth}|l}
\cline{1-2}
\cline{1-2} \centering \textbf{Name} & \centering \textbf{Description}& \\
\cline{1-2}
\endhead\cline{1-2}\multicolumn{3}{r}{\small\textit{continued on next page}}\\\endfoot\cline{1-2}
\endlastfoot\multicolumn{2}{|l|}{\textit{Inherited from object}}\\
\multicolumn{2}{|p{\varwidth}|}{\raggedright \_\_class\_\_}\\
\cline{1-2}
\end{longtable}


%%%%%%%%%%%%%%%%%%%%%%%%%%%%%%%%%%%%%%%%%%%%%%%%%%%%%%%%%%%%%%%%%%%%%%%%%%%
%%                            Class Variables                            %%
%%%%%%%%%%%%%%%%%%%%%%%%%%%%%%%%%%%%%%%%%%%%%%%%%%%%%%%%%%%%%%%%%%%%%%%%%%%

  \subsubsection{Class Variables}

    \vspace{-1cm}
\hspace{\varindent}\begin{longtable}{|p{\varnamewidth}|p{\vardescrwidth}|l}
\cline{1-2}
\cline{1-2} \centering \textbf{Name} & \centering \textbf{Description}& \\
\cline{1-2}
\endhead\cline{1-2}\multicolumn{3}{r}{\small\textit{continued on next page}}\\\endfoot\cline{1-2}
\endlastfoot\multicolumn{2}{|l|}{\textit{Inherited from SatStress.SatStress.StressDef \textit{(Section \ref{SatStress:SatStress:StressDef})}}}\\
\multicolumn{2}{|p{\varwidth}|}{\raggedright love, omega, satellite}\\
\cline{1-2}
\end{longtable}

    \index{SatStress \textit{(package)}!SatStress.SatStress \textit{(module)}!SatStress.SatStress.NSR \textit{(class)}|)}

%%%%%%%%%%%%%%%%%%%%%%%%%%%%%%%%%%%%%%%%%%%%%%%%%%%%%%%%%%%%%%%%%%%%%%%%%%%
%%                           Class Description                           %%
%%%%%%%%%%%%%%%%%%%%%%%%%%%%%%%%%%%%%%%%%%%%%%%%%%%%%%%%%%%%%%%%%%%%%%%%%%%

    \index{SatStress \textit{(package)}!SatStress.SatStress \textit{(module)}!SatStress.SatStress.Diurnal \textit{(class)}|(}
\subsection{Class Diurnal}

    \label{SatStress:SatStress:Diurnal}
\begin{tabular}{cccccccc}
% Line for object, linespec=[False, False]
\multicolumn{2}{r}{\settowidth{\BCL}{object}\multirow{2}{\BCL}{object}}
&&
&&
  \\\cline{3-3}
  &&\multicolumn{1}{c|}{}
&&
&&
  \\
% Line for SatStress.SatStress.StressDef, linespec=[False]
\multicolumn{4}{r}{\settowidth{\BCL}{SatStress.SatStress.StressDef}\multirow{2}{\BCL}{SatStress.SatStress.StressDef}}
&&
  \\\cline{5-5}
  &&&&\multicolumn{1}{c|}{}
&&
  \\
&&&&\multicolumn{2}{l}{\textbf{SatStress.SatStress.Diurnal}}
\end{tabular}

An object defining the stress field that arises on a satellite due to an 
eccentric orbit.

Diurnal is a subclass of \texttt{StressDef}.  See the derivation and 
detailed discussion of this stress field in in Wahr et al. (2008).


%%%%%%%%%%%%%%%%%%%%%%%%%%%%%%%%%%%%%%%%%%%%%%%%%%%%%%%%%%%%%%%%%%%%%%%%%%%
%%                                Methods                                %%
%%%%%%%%%%%%%%%%%%%%%%%%%%%%%%%%%%%%%%%%%%%%%%%%%%%%%%%%%%%%%%%%%%%%%%%%%%%

  \subsubsection{Methods}

    \vspace{0.5ex}

\hspace{.8\funcindent}\begin{boxedminipage}{\funcwidth}

    \raggedright \textbf{\_\_init\_\_}(\textit{self}, \textit{satellite})

    \vspace{-1.5ex}

    \rule{\textwidth}{0.5\fboxrule}
\setlength{\parskip}{2ex}
    Sets the object's satellite and omega attributes; calculates Love 
    numbers.

\setlength{\parskip}{1ex}
      \textbf{Parameters}
      \vspace{-1ex}

      \begin{quote}
        \begin{Ventry}{xxxxxxxxx}

          \item[satellite]

          the satellite to which the stress is being applied.

            {\it (type=\texttt{Satellite})}

        \end{Ventry}

      \end{quote}

      \textbf{Return Value}
    \vspace{-1ex}

      \begin{quote}
      an object defining the NSR stresses for a particular satellite.

      {\it (type=\texttt{NSR})}

      \end{quote}

      Overrides: object.\_\_init\_\_

    \end{boxedminipage}

    \vspace{0.5ex}

\hspace{.8\funcindent}\begin{boxedminipage}{\funcwidth}

    \raggedright \textbf{Ttt}(\textit{self}, \textit{theta}, \textit{phi}, \textit{t})

    \vspace{-1.5ex}

    \rule{\textwidth}{0.5\fboxrule}
\setlength{\parskip}{2ex}
    Calculates the \(\tau\)\_\(\theta\)\(\theta\) (north-south) component 
    of the stress tensor.

\setlength{\parskip}{1ex}
      \textbf{Parameters}
      \vspace{-1ex}

      \begin{quote}
        \begin{Ventry}{xxxxx}

          \item[theta]

          the co-latitude of the point at which to calculate the stress 
          [rad].

          \item[phi]

          the east-positive longitude of the point at which to calculate 
          the stress [rad].

          \item[t]

          the time, in seconds elapsed since pericenter, at which to 
          perform the stress calculation [s].

        \end{Ventry}

      \end{quote}

      \textbf{Return Value}
    \vspace{-1ex}

      \begin{quote}
      the \(\tau\)\_\(\theta\)\(\theta\) component of the 2x2 membrane 
      stress tensor.

      {\it (type=float)}

      \end{quote}

      Overrides: SatStress.SatStress.StressDef.Ttt

    \end{boxedminipage}

    \vspace{0.5ex}

\hspace{.8\funcindent}\begin{boxedminipage}{\funcwidth}

    \raggedright \textbf{Tpp}(\textit{self}, \textit{theta}, \textit{phi}, \textit{t})

    \vspace{-1.5ex}

    \rule{\textwidth}{0.5\fboxrule}
\setlength{\parskip}{2ex}
    Calculates the \(\tau\)\_\(\phi\)\(\phi\) (east-west) component of the 
    stress tensor.

\setlength{\parskip}{1ex}
      \textbf{Parameters}
      \vspace{-1ex}

      \begin{quote}
        \begin{Ventry}{xxxxx}

          \item[theta]

          the co-latitude of the point at which to calculate the stress 
          [rad].

          \item[phi]

          the east-positive longitude of the point at which to calculate 
          the stress [rad].

          \item[t]

          the time, in seconds elapsed since pericenter, at which to 
          perform the stress calculation [s].

        \end{Ventry}

      \end{quote}

      \textbf{Return Value}
    \vspace{-1ex}

      \begin{quote}
      the \(\tau\)\_\(\phi\)\(\phi\) component of the 2x2 membrane stress 
      tensor.

      {\it (type=float)}

      \end{quote}

      Overrides: SatStress.SatStress.StressDef.Tpp

    \end{boxedminipage}

    \vspace{0.5ex}

\hspace{.8\funcindent}\begin{boxedminipage}{\funcwidth}

    \raggedright \textbf{Tpt}(\textit{self}, \textit{theta}, \textit{phi}, \textit{t})

    \vspace{-1.5ex}

    \rule{\textwidth}{0.5\fboxrule}
\setlength{\parskip}{2ex}
    Calculates the \(\tau\)\_\(\phi\)\(\theta\) (off-diagonal) component of
    the stress tensor.

\setlength{\parskip}{1ex}
      \textbf{Parameters}
      \vspace{-1ex}

      \begin{quote}
        \begin{Ventry}{xxxxx}

          \item[theta]

          the co-latitude of the point at which to calculate the stress 
          [rad].

          \item[phi]

          the east-positive longitude of the point at which to calculate 
          the stress [rad].

          \item[t]

          the time in seconds elapsed since pericenter, at which to perform
          the stress calculation [s].

        \end{Ventry}

      \end{quote}

      \textbf{Return Value}
    \vspace{-1ex}

      \begin{quote}
      the \(\tau\)\_\(\phi\)\(\theta\) component of the 2x2 membrane stress
      tensor.

      {\it (type=float)}

      \end{quote}

      Overrides: SatStress.SatStress.StressDef.Tpt

    \end{boxedminipage}


\large{\textbf{\textit{Inherited from SatStress.SatStress.StressDef\textit{(Section \ref{SatStress:SatStress:StressDef})}}}}

\begin{quote}
Delta(), Gamma(), Z(), alpha(), b1(), b2(), calcLove(), calcLoveInfinitePeriod(), calcLoveWahr4LayerExternal(), g1(), g2(), lambda\_twiddle(), mu\_twiddle()
\end{quote}

\large{\textbf{\textit{Inherited from object}}}

\begin{quote}
\_\_delattr\_\_(), \_\_getattribute\_\_(), \_\_hash\_\_(), \_\_new\_\_(), \_\_reduce\_\_(), \_\_reduce\_ex\_\_(), \_\_repr\_\_(), \_\_setattr\_\_(), \_\_str\_\_()
\end{quote}

%%%%%%%%%%%%%%%%%%%%%%%%%%%%%%%%%%%%%%%%%%%%%%%%%%%%%%%%%%%%%%%%%%%%%%%%%%%
%%                              Properties                               %%
%%%%%%%%%%%%%%%%%%%%%%%%%%%%%%%%%%%%%%%%%%%%%%%%%%%%%%%%%%%%%%%%%%%%%%%%%%%

  \subsubsection{Properties}

    \vspace{-1cm}
\hspace{\varindent}\begin{longtable}{|p{\varnamewidth}|p{\vardescrwidth}|l}
\cline{1-2}
\cline{1-2} \centering \textbf{Name} & \centering \textbf{Description}& \\
\cline{1-2}
\endhead\cline{1-2}\multicolumn{3}{r}{\small\textit{continued on next page}}\\\endfoot\cline{1-2}
\endlastfoot\multicolumn{2}{|l|}{\textit{Inherited from object}}\\
\multicolumn{2}{|p{\varwidth}|}{\raggedright \_\_class\_\_}\\
\cline{1-2}
\end{longtable}


%%%%%%%%%%%%%%%%%%%%%%%%%%%%%%%%%%%%%%%%%%%%%%%%%%%%%%%%%%%%%%%%%%%%%%%%%%%
%%                            Class Variables                            %%
%%%%%%%%%%%%%%%%%%%%%%%%%%%%%%%%%%%%%%%%%%%%%%%%%%%%%%%%%%%%%%%%%%%%%%%%%%%

  \subsubsection{Class Variables}

    \vspace{-1cm}
\hspace{\varindent}\begin{longtable}{|p{\varnamewidth}|p{\vardescrwidth}|l}
\cline{1-2}
\cline{1-2} \centering \textbf{Name} & \centering \textbf{Description}& \\
\cline{1-2}
\endhead\cline{1-2}\multicolumn{3}{r}{\small\textit{continued on next page}}\\\endfoot\cline{1-2}
\endlastfoot\multicolumn{2}{|l|}{\textit{Inherited from SatStress.SatStress.StressDef \textit{(Section \ref{SatStress:SatStress:StressDef})}}}\\
\multicolumn{2}{|p{\varwidth}|}{\raggedright love, omega, satellite}\\
\cline{1-2}
\end{longtable}

    \index{SatStress \textit{(package)}!SatStress.SatStress \textit{(module)}!SatStress.SatStress.Diurnal \textit{(class)}|)}

%%%%%%%%%%%%%%%%%%%%%%%%%%%%%%%%%%%%%%%%%%%%%%%%%%%%%%%%%%%%%%%%%%%%%%%%%%%
%%                           Class Description                           %%
%%%%%%%%%%%%%%%%%%%%%%%%%%%%%%%%%%%%%%%%%%%%%%%%%%%%%%%%%%%%%%%%%%%%%%%%%%%

    \index{SatStress \textit{(package)}!SatStress.SatStress \textit{(module)}!SatStress.SatStress.StressCalc \textit{(class)}|(}
\subsection{Class StressCalc}

    \label{SatStress:SatStress:StressCalc}
\begin{tabular}{cccccc}
% Line for object, linespec=[False]
\multicolumn{2}{r}{\settowidth{\BCL}{object}\multirow{2}{\BCL}{object}}
&&
  \\\cline{3-3}
  &&\multicolumn{1}{c|}{}
&&
  \\
&&\multicolumn{2}{l}{\textbf{SatStress.SatStress.StressCalc}}
\end{tabular}

An object which calculates the stresses on the surface of a 
\texttt{Satellite} that result from one or more stress fields.


%%%%%%%%%%%%%%%%%%%%%%%%%%%%%%%%%%%%%%%%%%%%%%%%%%%%%%%%%%%%%%%%%%%%%%%%%%%
%%                                Methods                                %%
%%%%%%%%%%%%%%%%%%%%%%%%%%%%%%%%%%%%%%%%%%%%%%%%%%%%%%%%%%%%%%%%%%%%%%%%%%%

  \subsubsection{Methods}

    \vspace{0.5ex}

\hspace{.8\funcindent}\begin{boxedminipage}{\funcwidth}

    \raggedright \textbf{\_\_init\_\_}(\textit{self}, \textit{stressdefs})

    \vspace{-1.5ex}

    \rule{\textwidth}{0.5\fboxrule}
\setlength{\parskip}{2ex}
    Defines the list of stresses which are to be calculated at a given 
    point.

\setlength{\parskip}{1ex}
      \textbf{Parameters}
      \vspace{-1ex}

      \begin{quote}
        \begin{Ventry}{xxxxxxxxxx}

          \item[stressdefs]

          a list of \texttt{StressDef} objects, corresponding to the 
          stresses which are to be included in the calculation.

            {\it (type=list)}

        \end{Ventry}

      \end{quote}

      \textbf{Return Value}
    \vspace{-1ex}

      \begin{quote}
      {\it (type=\texttt{StressCalc})}

      \end{quote}

      Overrides: object.\_\_init\_\_

    \end{boxedminipage}

    \label{SatStress:SatStress:StressCalc:tensor}
    \index{SatStress \textit{(package)}!SatStress.SatStress \textit{(module)}!SatStress.SatStress.StressCalc \textit{(class)}!SatStress.SatStress.StressCalc.tensor \textit{(method)}}

    \vspace{0.5ex}

\hspace{.8\funcindent}\begin{boxedminipage}{\funcwidth}

    \raggedright \textbf{tensor}(\textit{self}, \textit{theta}, \textit{phi}, \textit{t})

    \vspace{-1.5ex}

    \rule{\textwidth}{0.5\fboxrule}
\setlength{\parskip}{2ex}
    Calculates surface stresses and returns them as a 2x2 stress tensor.

\setlength{\parskip}{1ex}
      \textbf{Parameters}
      \vspace{-1ex}

      \begin{quote}
        \begin{Ventry}{xxxxx}

          \item[theta]

          the co-latitude of the point at which to calculate the stress 
          [rad].

            {\it (type=float)}

          \item[phi]

          the east-positive longitude of the point at which to calculate 
          the stress [rad].

            {\it (type=float)}

          \item[t]

          the time in seconds elapsed since pericenter, at which to perform
          the stress calculation [s].

            {\it (type=float)}

        \end{Ventry}

      \end{quote}

      \textbf{Return Value}
    \vspace{-1ex}

      \begin{quote}
      symmetric 2x2 surface (membrane) stress tensor \(\tau\)

      {\it (type=Numpy.array)}

      \end{quote}

    \end{boxedminipage}


\large{\textbf{\textit{Inherited from object}}}

\begin{quote}
\_\_delattr\_\_(), \_\_getattribute\_\_(), \_\_hash\_\_(), \_\_new\_\_(), \_\_reduce\_\_(), \_\_reduce\_ex\_\_(), \_\_repr\_\_(), \_\_setattr\_\_(), \_\_str\_\_()
\end{quote}

%%%%%%%%%%%%%%%%%%%%%%%%%%%%%%%%%%%%%%%%%%%%%%%%%%%%%%%%%%%%%%%%%%%%%%%%%%%
%%                              Properties                               %%
%%%%%%%%%%%%%%%%%%%%%%%%%%%%%%%%%%%%%%%%%%%%%%%%%%%%%%%%%%%%%%%%%%%%%%%%%%%

  \subsubsection{Properties}

    \vspace{-1cm}
\hspace{\varindent}\begin{longtable}{|p{\varnamewidth}|p{\vardescrwidth}|l}
\cline{1-2}
\cline{1-2} \centering \textbf{Name} & \centering \textbf{Description}& \\
\cline{1-2}
\endhead\cline{1-2}\multicolumn{3}{r}{\small\textit{continued on next page}}\\\endfoot\cline{1-2}
\endlastfoot\multicolumn{2}{|l|}{\textit{Inherited from object}}\\
\multicolumn{2}{|p{\varwidth}|}{\raggedright \_\_class\_\_}\\
\cline{1-2}
\end{longtable}


%%%%%%%%%%%%%%%%%%%%%%%%%%%%%%%%%%%%%%%%%%%%%%%%%%%%%%%%%%%%%%%%%%%%%%%%%%%
%%                          Instance Variables                           %%
%%%%%%%%%%%%%%%%%%%%%%%%%%%%%%%%%%%%%%%%%%%%%%%%%%%%%%%%%%%%%%%%%%%%%%%%%%%

  \subsubsection{Instance Variables}

    \vspace{-1cm}
\hspace{\varindent}\begin{longtable}{|p{\varnamewidth}|p{\vardescrwidth}|l}
\cline{1-2}
\cline{1-2} \centering \textbf{Name} & \centering \textbf{Description}& \\
\cline{1-2}
\endhead\cline{1-2}\multicolumn{3}{r}{\small\textit{continued on next page}}\\\endfoot\cline{1-2}
\endlastfoot\raggedright s\-t\-r\-e\-s\-s\-e\-s\- & \raggedright a list of \texttt{StressDef} objects, corresponding to the 
          stresses which are to be included in the calculations done by the
          \texttt{StressCalc} object.

            {\it (type=list)}&\\
\cline{1-2}
\end{longtable}

    \index{SatStress \textit{(package)}!SatStress.SatStress \textit{(module)}!SatStress.SatStress.StressCalc \textit{(class)}|)}

%%%%%%%%%%%%%%%%%%%%%%%%%%%%%%%%%%%%%%%%%%%%%%%%%%%%%%%%%%%%%%%%%%%%%%%%%%%
%%                           Class Description                           %%
%%%%%%%%%%%%%%%%%%%%%%%%%%%%%%%%%%%%%%%%%%%%%%%%%%%%%%%%%%%%%%%%%%%%%%%%%%%

    \index{SatStress \textit{(package)}!SatStress.SatStress \textit{(module)}!SatStress.SatStress.Error \textit{(class)}|(}
\subsection{Class Error}

    \label{SatStress:SatStress:Error}
\begin{tabular}{cccccccccc}
% Line for object, linespec=[False, False, False]
\multicolumn{2}{r}{\settowidth{\BCL}{object}\multirow{2}{\BCL}{object}}
&&
&&
&&
  \\\cline{3-3}
  &&\multicolumn{1}{c|}{}
&&
&&
&&
  \\
% Line for exceptions.BaseException, linespec=[False, False]
\multicolumn{4}{r}{\settowidth{\BCL}{exceptions.BaseException}\multirow{2}{\BCL}{exceptions.BaseException}}
&&
&&
  \\\cline{5-5}
  &&&&\multicolumn{1}{c|}{}
&&
&&
  \\
% Line for exceptions.Exception, linespec=[False]
\multicolumn{6}{r}{\settowidth{\BCL}{exceptions.Exception}\multirow{2}{\BCL}{exceptions.Exception}}
&&
  \\\cline{7-7}
  &&&&&&\multicolumn{1}{c|}{}
&&
  \\
&&&&&&\multicolumn{2}{l}{\textbf{SatStress.SatStress.Error}}
\end{tabular}

\textbf{Known Subclasses:}
SatStress.SatStress.SatelliteParamError,
    SatStress.SatStress.NameValueFileError

Base class for errors within the SatStress module.


%%%%%%%%%%%%%%%%%%%%%%%%%%%%%%%%%%%%%%%%%%%%%%%%%%%%%%%%%%%%%%%%%%%%%%%%%%%
%%                                Methods                                %%
%%%%%%%%%%%%%%%%%%%%%%%%%%%%%%%%%%%%%%%%%%%%%%%%%%%%%%%%%%%%%%%%%%%%%%%%%%%

  \subsubsection{Methods}


\large{\textbf{\textit{Inherited from exceptions.Exception}}}

\begin{quote}
\_\_init\_\_(), \_\_new\_\_()
\end{quote}

\large{\textbf{\textit{Inherited from exceptions.BaseException}}}

\begin{quote}
\_\_delattr\_\_(), \_\_getattribute\_\_(), \_\_getitem\_\_(), \_\_getslice\_\_(), \_\_reduce\_\_(), \_\_repr\_\_(), \_\_setattr\_\_(), \_\_setstate\_\_(), \_\_str\_\_()
\end{quote}

\large{\textbf{\textit{Inherited from object}}}

\begin{quote}
\_\_hash\_\_(), \_\_reduce\_ex\_\_()
\end{quote}

%%%%%%%%%%%%%%%%%%%%%%%%%%%%%%%%%%%%%%%%%%%%%%%%%%%%%%%%%%%%%%%%%%%%%%%%%%%
%%                              Properties                               %%
%%%%%%%%%%%%%%%%%%%%%%%%%%%%%%%%%%%%%%%%%%%%%%%%%%%%%%%%%%%%%%%%%%%%%%%%%%%

  \subsubsection{Properties}

    \vspace{-1cm}
\hspace{\varindent}\begin{longtable}{|p{\varnamewidth}|p{\vardescrwidth}|l}
\cline{1-2}
\cline{1-2} \centering \textbf{Name} & \centering \textbf{Description}& \\
\cline{1-2}
\endhead\cline{1-2}\multicolumn{3}{r}{\small\textit{continued on next page}}\\\endfoot\cline{1-2}
\endlastfoot\multicolumn{2}{|l|}{\textit{Inherited from exceptions.BaseException}}\\
\multicolumn{2}{|p{\varwidth}|}{\raggedright args, message}\\
\cline{1-2}
\multicolumn{2}{|l|}{\textit{Inherited from object}}\\
\multicolumn{2}{|p{\varwidth}|}{\raggedright \_\_class\_\_}\\
\cline{1-2}
\end{longtable}

    \index{SatStress \textit{(package)}!SatStress.SatStress \textit{(module)}!SatStress.SatStress.Error \textit{(class)}|)}

%%%%%%%%%%%%%%%%%%%%%%%%%%%%%%%%%%%%%%%%%%%%%%%%%%%%%%%%%%%%%%%%%%%%%%%%%%%
%%                           Class Description                           %%
%%%%%%%%%%%%%%%%%%%%%%%%%%%%%%%%%%%%%%%%%%%%%%%%%%%%%%%%%%%%%%%%%%%%%%%%%%%

    \index{SatStress \textit{(package)}!SatStress.SatStress \textit{(module)}!SatStress.SatStress.NameValueFileError \textit{(class)}|(}
\subsection{Class NameValueFileError}

    \label{SatStress:SatStress:NameValueFileError}
\begin{tabular}{cccccccccccc}
% Line for object, linespec=[False, False, False, False]
\multicolumn{2}{r}{\settowidth{\BCL}{object}\multirow{2}{\BCL}{object}}
&&
&&
&&
&&
  \\\cline{3-3}
  &&\multicolumn{1}{c|}{}
&&
&&
&&
&&
  \\
% Line for exceptions.BaseException, linespec=[False, False, False]
\multicolumn{4}{r}{\settowidth{\BCL}{exceptions.BaseException}\multirow{2}{\BCL}{exceptions.BaseException}}
&&
&&
&&
  \\\cline{5-5}
  &&&&\multicolumn{1}{c|}{}
&&
&&
&&
  \\
% Line for exceptions.Exception, linespec=[False, False]
\multicolumn{6}{r}{\settowidth{\BCL}{exceptions.Exception}\multirow{2}{\BCL}{exceptions.Exception}}
&&
&&
  \\\cline{7-7}
  &&&&&&\multicolumn{1}{c|}{}
&&
&&
  \\
% Line for SatStress.SatStress.Error, linespec=[False]
\multicolumn{8}{r}{\settowidth{\BCL}{SatStress.SatStress.Error}\multirow{2}{\BCL}{SatStress.SatStress.Error}}
&&
  \\\cline{9-9}
  &&&&&&&&\multicolumn{1}{c|}{}
&&
  \\
&&&&&&&&\multicolumn{2}{l}{\textbf{SatStress.SatStress.NameValueFileError}}
\end{tabular}

\textbf{Known Subclasses:}
SatStress.SatStress.NameValueFileDuplicateNameError,
    SatStress.SatStress.NameValueFileParseError

Base class for errors related to NAME=VALUE style input files.


%%%%%%%%%%%%%%%%%%%%%%%%%%%%%%%%%%%%%%%%%%%%%%%%%%%%%%%%%%%%%%%%%%%%%%%%%%%
%%                                Methods                                %%
%%%%%%%%%%%%%%%%%%%%%%%%%%%%%%%%%%%%%%%%%%%%%%%%%%%%%%%%%%%%%%%%%%%%%%%%%%%

  \subsubsection{Methods}


\large{\textbf{\textit{Inherited from exceptions.Exception}}}

\begin{quote}
\_\_init\_\_(), \_\_new\_\_()
\end{quote}

\large{\textbf{\textit{Inherited from exceptions.BaseException}}}

\begin{quote}
\_\_delattr\_\_(), \_\_getattribute\_\_(), \_\_getitem\_\_(), \_\_getslice\_\_(), \_\_reduce\_\_(), \_\_repr\_\_(), \_\_setattr\_\_(), \_\_setstate\_\_(), \_\_str\_\_()
\end{quote}

\large{\textbf{\textit{Inherited from object}}}

\begin{quote}
\_\_hash\_\_(), \_\_reduce\_ex\_\_()
\end{quote}

%%%%%%%%%%%%%%%%%%%%%%%%%%%%%%%%%%%%%%%%%%%%%%%%%%%%%%%%%%%%%%%%%%%%%%%%%%%
%%                              Properties                               %%
%%%%%%%%%%%%%%%%%%%%%%%%%%%%%%%%%%%%%%%%%%%%%%%%%%%%%%%%%%%%%%%%%%%%%%%%%%%

  \subsubsection{Properties}

    \vspace{-1cm}
\hspace{\varindent}\begin{longtable}{|p{\varnamewidth}|p{\vardescrwidth}|l}
\cline{1-2}
\cline{1-2} \centering \textbf{Name} & \centering \textbf{Description}& \\
\cline{1-2}
\endhead\cline{1-2}\multicolumn{3}{r}{\small\textit{continued on next page}}\\\endfoot\cline{1-2}
\endlastfoot\multicolumn{2}{|l|}{\textit{Inherited from exceptions.BaseException}}\\
\multicolumn{2}{|p{\varwidth}|}{\raggedright args, message}\\
\cline{1-2}
\multicolumn{2}{|l|}{\textit{Inherited from object}}\\
\multicolumn{2}{|p{\varwidth}|}{\raggedright \_\_class\_\_}\\
\cline{1-2}
\end{longtable}

    \index{SatStress \textit{(package)}!SatStress.SatStress \textit{(module)}!SatStress.SatStress.NameValueFileError \textit{(class)}|)}

%%%%%%%%%%%%%%%%%%%%%%%%%%%%%%%%%%%%%%%%%%%%%%%%%%%%%%%%%%%%%%%%%%%%%%%%%%%
%%                           Class Description                           %%
%%%%%%%%%%%%%%%%%%%%%%%%%%%%%%%%%%%%%%%%%%%%%%%%%%%%%%%%%%%%%%%%%%%%%%%%%%%

    \index{SatStress \textit{(package)}!SatStress.SatStress \textit{(module)}!SatStress.SatStress.NameValueFileParseError \textit{(class)}|(}
\subsection{Class NameValueFileParseError}

    \label{SatStress:SatStress:NameValueFileParseError}
\begin{tabular}{cccccccccccccc}
% Line for object, linespec=[False, False, False, False, False]
\multicolumn{2}{r}{\settowidth{\BCL}{object}\multirow{2}{\BCL}{object}}
&&
&&
&&
&&
&&
  \\\cline{3-3}
  &&\multicolumn{1}{c|}{}
&&
&&
&&
&&
&&
  \\
% Line for exceptions.BaseException, linespec=[False, False, False, False]
\multicolumn{4}{r}{\settowidth{\BCL}{exceptions.BaseException}\multirow{2}{\BCL}{exceptions.BaseException}}
&&
&&
&&
&&
  \\\cline{5-5}
  &&&&\multicolumn{1}{c|}{}
&&
&&
&&
&&
  \\
% Line for exceptions.Exception, linespec=[False, False, False]
\multicolumn{6}{r}{\settowidth{\BCL}{exceptions.Exception}\multirow{2}{\BCL}{exceptions.Exception}}
&&
&&
&&
  \\\cline{7-7}
  &&&&&&\multicolumn{1}{c|}{}
&&
&&
&&
  \\
% Line for SatStress.SatStress.Error, linespec=[False, False]
\multicolumn{8}{r}{\settowidth{\BCL}{SatStress.SatStress.Error}\multirow{2}{\BCL}{SatStress.SatStress.Error}}
&&
&&
  \\\cline{9-9}
  &&&&&&&&\multicolumn{1}{c|}{}
&&
&&
  \\
% Line for SatStress.SatStress.NameValueFileError, linespec=[False]
\multicolumn{10}{r}{\settowidth{\BCL}{SatStress.SatStress.NameValueFileError}\multirow{2}{\BCL}{SatStress.SatStress.NameValueFileError}}
&&
  \\\cline{11-11}
  &&&&&&&&&&\multicolumn{1}{c|}{}
&&
  \\
&&&&&&&&&&\multicolumn{2}{l}{\textbf{SatStress.SatStress.NameValueFileParseError}}
\end{tabular}

Indicates a poorly formatted NAME=VALUE files.


%%%%%%%%%%%%%%%%%%%%%%%%%%%%%%%%%%%%%%%%%%%%%%%%%%%%%%%%%%%%%%%%%%%%%%%%%%%
%%                                Methods                                %%
%%%%%%%%%%%%%%%%%%%%%%%%%%%%%%%%%%%%%%%%%%%%%%%%%%%%%%%%%%%%%%%%%%%%%%%%%%%

  \subsubsection{Methods}

    \vspace{0.5ex}

\hspace{.8\funcindent}\begin{boxedminipage}{\funcwidth}

    \raggedright \textbf{\_\_init\_\_}(\textit{self}, \textit{nvf}, \textit{line})

    \vspace{-1.5ex}

    \rule{\textwidth}{0.5\fboxrule}
\setlength{\parskip}{2ex}
    Stores the file and line that generated the parse error.

    The file object (nvf) and the contents of the poorly formed line 
    (badline) are stored within the exception, so we can print an error 
    message with useful debugging information to the user.

\setlength{\parskip}{1ex}
      Overrides: object.\_\_init\_\_

    \end{boxedminipage}

    \vspace{0.5ex}

\hspace{.8\funcindent}\begin{boxedminipage}{\funcwidth}

    \raggedright \textbf{\_\_str\_\_}(\textit{self})

\setlength{\parskip}{2ex}
    str(x)

\setlength{\parskip}{1ex}
      Overrides: object.\_\_str\_\_ 	extit{(inherited documentation)}

    \end{boxedminipage}


\large{\textbf{\textit{Inherited from exceptions.Exception}}}

\begin{quote}
\_\_new\_\_()
\end{quote}

\large{\textbf{\textit{Inherited from exceptions.BaseException}}}

\begin{quote}
\_\_delattr\_\_(), \_\_getattribute\_\_(), \_\_getitem\_\_(), \_\_getslice\_\_(), \_\_reduce\_\_(), \_\_repr\_\_(), \_\_setattr\_\_(), \_\_setstate\_\_()
\end{quote}

\large{\textbf{\textit{Inherited from object}}}

\begin{quote}
\_\_hash\_\_(), \_\_reduce\_ex\_\_()
\end{quote}

%%%%%%%%%%%%%%%%%%%%%%%%%%%%%%%%%%%%%%%%%%%%%%%%%%%%%%%%%%%%%%%%%%%%%%%%%%%
%%                              Properties                               %%
%%%%%%%%%%%%%%%%%%%%%%%%%%%%%%%%%%%%%%%%%%%%%%%%%%%%%%%%%%%%%%%%%%%%%%%%%%%

  \subsubsection{Properties}

    \vspace{-1cm}
\hspace{\varindent}\begin{longtable}{|p{\varnamewidth}|p{\vardescrwidth}|l}
\cline{1-2}
\cline{1-2} \centering \textbf{Name} & \centering \textbf{Description}& \\
\cline{1-2}
\endhead\cline{1-2}\multicolumn{3}{r}{\small\textit{continued on next page}}\\\endfoot\cline{1-2}
\endlastfoot\multicolumn{2}{|l|}{\textit{Inherited from exceptions.BaseException}}\\
\multicolumn{2}{|p{\varwidth}|}{\raggedright args, message}\\
\cline{1-2}
\multicolumn{2}{|l|}{\textit{Inherited from object}}\\
\multicolumn{2}{|p{\varwidth}|}{\raggedright \_\_class\_\_}\\
\cline{1-2}
\end{longtable}

    \index{SatStress \textit{(package)}!SatStress.SatStress \textit{(module)}!SatStress.SatStress.NameValueFileParseError \textit{(class)}|)}

%%%%%%%%%%%%%%%%%%%%%%%%%%%%%%%%%%%%%%%%%%%%%%%%%%%%%%%%%%%%%%%%%%%%%%%%%%%
%%                           Class Description                           %%
%%%%%%%%%%%%%%%%%%%%%%%%%%%%%%%%%%%%%%%%%%%%%%%%%%%%%%%%%%%%%%%%%%%%%%%%%%%

    \index{SatStress \textit{(package)}!SatStress.SatStress \textit{(module)}!SatStress.SatStress.NameValueFileDuplicateNameError \textit{(class)}|(}
\subsection{Class NameValueFileDuplicateNameError}

    \label{SatStress:SatStress:NameValueFileDuplicateNameError}
\begin{tabular}{cccccccccccccc}
% Line for object, linespec=[False, False, False, False, False]
\multicolumn{2}{r}{\settowidth{\BCL}{object}\multirow{2}{\BCL}{object}}
&&
&&
&&
&&
&&
  \\\cline{3-3}
  &&\multicolumn{1}{c|}{}
&&
&&
&&
&&
&&
  \\
% Line for exceptions.BaseException, linespec=[False, False, False, False]
\multicolumn{4}{r}{\settowidth{\BCL}{exceptions.BaseException}\multirow{2}{\BCL}{exceptions.BaseException}}
&&
&&
&&
&&
  \\\cline{5-5}
  &&&&\multicolumn{1}{c|}{}
&&
&&
&&
&&
  \\
% Line for exceptions.Exception, linespec=[False, False, False]
\multicolumn{6}{r}{\settowidth{\BCL}{exceptions.Exception}\multirow{2}{\BCL}{exceptions.Exception}}
&&
&&
&&
  \\\cline{7-7}
  &&&&&&\multicolumn{1}{c|}{}
&&
&&
&&
  \\
% Line for SatStress.SatStress.Error, linespec=[False, False]
\multicolumn{8}{r}{\settowidth{\BCL}{SatStress.SatStress.Error}\multirow{2}{\BCL}{SatStress.SatStress.Error}}
&&
&&
  \\\cline{9-9}
  &&&&&&&&\multicolumn{1}{c|}{}
&&
&&
  \\
% Line for SatStress.SatStress.NameValueFileError, linespec=[False]
\multicolumn{10}{r}{\settowidth{\BCL}{SatStress.SatStress.NameValueFileError}\multirow{2}{\BCL}{SatStress.SatStress.NameValueFileError}}
&&
  \\\cline{11-11}
  &&&&&&&&&&\multicolumn{1}{c|}{}
&&
  \\
&&&&&&&&&&\multicolumn{2}{l}{\textbf{SatStress.SatStress.NameValueFileDuplicateNameError}}
\end{tabular}

Indicates multiple copies of the same name in an input file.


%%%%%%%%%%%%%%%%%%%%%%%%%%%%%%%%%%%%%%%%%%%%%%%%%%%%%%%%%%%%%%%%%%%%%%%%%%%
%%                                Methods                                %%
%%%%%%%%%%%%%%%%%%%%%%%%%%%%%%%%%%%%%%%%%%%%%%%%%%%%%%%%%%%%%%%%%%%%%%%%%%%

  \subsubsection{Methods}

    \vspace{0.5ex}

\hspace{.8\funcindent}\begin{boxedminipage}{\funcwidth}

    \raggedright \textbf{\_\_init\_\_}(\textit{self}, \textit{nvf}, \textit{name})

    \vspace{-1.5ex}

    \rule{\textwidth}{0.5\fboxrule}
\setlength{\parskip}{2ex}
    Stores the file and the NAME that was found to be multiply defined.

    NAME is the key, which has been found to be multiply defined in the 
    input file, nvf.

\setlength{\parskip}{1ex}
      Overrides: object.\_\_init\_\_

    \end{boxedminipage}

    \vspace{0.5ex}

\hspace{.8\funcindent}\begin{boxedminipage}{\funcwidth}

    \raggedright \textbf{\_\_str\_\_}(\textit{self})

\setlength{\parskip}{2ex}
    str(x)

\setlength{\parskip}{1ex}
      Overrides: object.\_\_str\_\_ 	extit{(inherited documentation)}

    \end{boxedminipage}


\large{\textbf{\textit{Inherited from exceptions.Exception}}}

\begin{quote}
\_\_new\_\_()
\end{quote}

\large{\textbf{\textit{Inherited from exceptions.BaseException}}}

\begin{quote}
\_\_delattr\_\_(), \_\_getattribute\_\_(), \_\_getitem\_\_(), \_\_getslice\_\_(), \_\_reduce\_\_(), \_\_repr\_\_(), \_\_setattr\_\_(), \_\_setstate\_\_()
\end{quote}

\large{\textbf{\textit{Inherited from object}}}

\begin{quote}
\_\_hash\_\_(), \_\_reduce\_ex\_\_()
\end{quote}

%%%%%%%%%%%%%%%%%%%%%%%%%%%%%%%%%%%%%%%%%%%%%%%%%%%%%%%%%%%%%%%%%%%%%%%%%%%
%%                              Properties                               %%
%%%%%%%%%%%%%%%%%%%%%%%%%%%%%%%%%%%%%%%%%%%%%%%%%%%%%%%%%%%%%%%%%%%%%%%%%%%

  \subsubsection{Properties}

    \vspace{-1cm}
\hspace{\varindent}\begin{longtable}{|p{\varnamewidth}|p{\vardescrwidth}|l}
\cline{1-2}
\cline{1-2} \centering \textbf{Name} & \centering \textbf{Description}& \\
\cline{1-2}
\endhead\cline{1-2}\multicolumn{3}{r}{\small\textit{continued on next page}}\\\endfoot\cline{1-2}
\endlastfoot\multicolumn{2}{|l|}{\textit{Inherited from exceptions.BaseException}}\\
\multicolumn{2}{|p{\varwidth}|}{\raggedright args, message}\\
\cline{1-2}
\multicolumn{2}{|l|}{\textit{Inherited from object}}\\
\multicolumn{2}{|p{\varwidth}|}{\raggedright \_\_class\_\_}\\
\cline{1-2}
\end{longtable}

    \index{SatStress \textit{(package)}!SatStress.SatStress \textit{(module)}!SatStress.SatStress.NameValueFileDuplicateNameError \textit{(class)}|)}

%%%%%%%%%%%%%%%%%%%%%%%%%%%%%%%%%%%%%%%%%%%%%%%%%%%%%%%%%%%%%%%%%%%%%%%%%%%
%%                           Class Description                           %%
%%%%%%%%%%%%%%%%%%%%%%%%%%%%%%%%%%%%%%%%%%%%%%%%%%%%%%%%%%%%%%%%%%%%%%%%%%%

    \index{SatStress \textit{(package)}!SatStress.SatStress \textit{(module)}!SatStress.SatStress.SatelliteParamError \textit{(class)}|(}
\subsection{Class SatelliteParamError}

    \label{SatStress:SatStress:SatelliteParamError}
\begin{tabular}{cccccccccccc}
% Line for object, linespec=[False, False, False, False]
\multicolumn{2}{r}{\settowidth{\BCL}{object}\multirow{2}{\BCL}{object}}
&&
&&
&&
&&
  \\\cline{3-3}
  &&\multicolumn{1}{c|}{}
&&
&&
&&
&&
  \\
% Line for exceptions.BaseException, linespec=[False, False, False]
\multicolumn{4}{r}{\settowidth{\BCL}{exceptions.BaseException}\multirow{2}{\BCL}{exceptions.BaseException}}
&&
&&
&&
  \\\cline{5-5}
  &&&&\multicolumn{1}{c|}{}
&&
&&
&&
  \\
% Line for exceptions.Exception, linespec=[False, False]
\multicolumn{6}{r}{\settowidth{\BCL}{exceptions.Exception}\multirow{2}{\BCL}{exceptions.Exception}}
&&
&&
  \\\cline{7-7}
  &&&&&&\multicolumn{1}{c|}{}
&&
&&
  \\
% Line for SatStress.SatStress.Error, linespec=[False]
\multicolumn{8}{r}{\settowidth{\BCL}{SatStress.SatStress.Error}\multirow{2}{\BCL}{SatStress.SatStress.Error}}
&&
  \\\cline{9-9}
  &&&&&&&&\multicolumn{1}{c|}{}
&&
  \\
&&&&&&&&\multicolumn{2}{l}{\textbf{SatStress.SatStress.SatelliteParamError}}
\end{tabular}

\textbf{Known Subclasses:}
SatStress.SatStress.InvalidSatelliteParamError,
    SatStress.SatStress.MissingSatelliteParamError

Indicates a problem with the Satellite initialization.


%%%%%%%%%%%%%%%%%%%%%%%%%%%%%%%%%%%%%%%%%%%%%%%%%%%%%%%%%%%%%%%%%%%%%%%%%%%
%%                                Methods                                %%
%%%%%%%%%%%%%%%%%%%%%%%%%%%%%%%%%%%%%%%%%%%%%%%%%%%%%%%%%%%%%%%%%%%%%%%%%%%

  \subsubsection{Methods}


\large{\textbf{\textit{Inherited from exceptions.Exception}}}

\begin{quote}
\_\_init\_\_(), \_\_new\_\_()
\end{quote}

\large{\textbf{\textit{Inherited from exceptions.BaseException}}}

\begin{quote}
\_\_delattr\_\_(), \_\_getattribute\_\_(), \_\_getitem\_\_(), \_\_getslice\_\_(), \_\_reduce\_\_(), \_\_repr\_\_(), \_\_setattr\_\_(), \_\_setstate\_\_(), \_\_str\_\_()
\end{quote}

\large{\textbf{\textit{Inherited from object}}}

\begin{quote}
\_\_hash\_\_(), \_\_reduce\_ex\_\_()
\end{quote}

%%%%%%%%%%%%%%%%%%%%%%%%%%%%%%%%%%%%%%%%%%%%%%%%%%%%%%%%%%%%%%%%%%%%%%%%%%%
%%                              Properties                               %%
%%%%%%%%%%%%%%%%%%%%%%%%%%%%%%%%%%%%%%%%%%%%%%%%%%%%%%%%%%%%%%%%%%%%%%%%%%%

  \subsubsection{Properties}

    \vspace{-1cm}
\hspace{\varindent}\begin{longtable}{|p{\varnamewidth}|p{\vardescrwidth}|l}
\cline{1-2}
\cline{1-2} \centering \textbf{Name} & \centering \textbf{Description}& \\
\cline{1-2}
\endhead\cline{1-2}\multicolumn{3}{r}{\small\textit{continued on next page}}\\\endfoot\cline{1-2}
\endlastfoot\multicolumn{2}{|l|}{\textit{Inherited from exceptions.BaseException}}\\
\multicolumn{2}{|p{\varwidth}|}{\raggedright args, message}\\
\cline{1-2}
\multicolumn{2}{|l|}{\textit{Inherited from object}}\\
\multicolumn{2}{|p{\varwidth}|}{\raggedright \_\_class\_\_}\\
\cline{1-2}
\end{longtable}

    \index{SatStress \textit{(package)}!SatStress.SatStress \textit{(module)}!SatStress.SatStress.SatelliteParamError \textit{(class)}|)}

%%%%%%%%%%%%%%%%%%%%%%%%%%%%%%%%%%%%%%%%%%%%%%%%%%%%%%%%%%%%%%%%%%%%%%%%%%%
%%                           Class Description                           %%
%%%%%%%%%%%%%%%%%%%%%%%%%%%%%%%%%%%%%%%%%%%%%%%%%%%%%%%%%%%%%%%%%%%%%%%%%%%

    \index{SatStress \textit{(package)}!SatStress.SatStress \textit{(module)}!SatStress.SatStress.MissingSatelliteParamError \textit{(class)}|(}
\subsection{Class MissingSatelliteParamError}

    \label{SatStress:SatStress:MissingSatelliteParamError}
\begin{tabular}{cccccccccccccc}
% Line for object, linespec=[False, False, False, False, False]
\multicolumn{2}{r}{\settowidth{\BCL}{object}\multirow{2}{\BCL}{object}}
&&
&&
&&
&&
&&
  \\\cline{3-3}
  &&\multicolumn{1}{c|}{}
&&
&&
&&
&&
&&
  \\
% Line for exceptions.BaseException, linespec=[False, False, False, False]
\multicolumn{4}{r}{\settowidth{\BCL}{exceptions.BaseException}\multirow{2}{\BCL}{exceptions.BaseException}}
&&
&&
&&
&&
  \\\cline{5-5}
  &&&&\multicolumn{1}{c|}{}
&&
&&
&&
&&
  \\
% Line for exceptions.Exception, linespec=[False, False, False]
\multicolumn{6}{r}{\settowidth{\BCL}{exceptions.Exception}\multirow{2}{\BCL}{exceptions.Exception}}
&&
&&
&&
  \\\cline{7-7}
  &&&&&&\multicolumn{1}{c|}{}
&&
&&
&&
  \\
% Line for SatStress.SatStress.Error, linespec=[False, False]
\multicolumn{8}{r}{\settowidth{\BCL}{SatStress.SatStress.Error}\multirow{2}{\BCL}{SatStress.SatStress.Error}}
&&
&&
  \\\cline{9-9}
  &&&&&&&&\multicolumn{1}{c|}{}
&&
&&
  \\
% Line for SatStress.SatStress.SatelliteParamError, linespec=[False]
\multicolumn{10}{r}{\settowidth{\BCL}{SatStress.SatStress.SatelliteParamError}\multirow{2}{\BCL}{SatStress.SatStress.SatelliteParamError}}
&&
  \\\cline{11-11}
  &&&&&&&&&&\multicolumn{1}{c|}{}
&&
  \\
&&&&&&&&&&\multicolumn{2}{l}{\textbf{SatStress.SatStress.MissingSatelliteParamError}}
\end{tabular}

Indicates a required parameter was not found in the input file.


%%%%%%%%%%%%%%%%%%%%%%%%%%%%%%%%%%%%%%%%%%%%%%%%%%%%%%%%%%%%%%%%%%%%%%%%%%%
%%                                Methods                                %%
%%%%%%%%%%%%%%%%%%%%%%%%%%%%%%%%%%%%%%%%%%%%%%%%%%%%%%%%%%%%%%%%%%%%%%%%%%%

  \subsubsection{Methods}

    \vspace{0.5ex}

\hspace{.8\funcindent}\begin{boxedminipage}{\funcwidth}

    \raggedright \textbf{\_\_init\_\_}(\textit{self}, \textit{sat}, \textit{missingname})

\setlength{\parskip}{2ex}
    x.\_\_init\_\_(...) initializes x; see x.\_\_class\_\_.\_\_doc\_\_ for 
    signature

\setlength{\parskip}{1ex}
      Overrides: object.\_\_init\_\_ 	extit{(inherited documentation)}

    \end{boxedminipage}

    \vspace{0.5ex}

\hspace{.8\funcindent}\begin{boxedminipage}{\funcwidth}

    \raggedright \textbf{\_\_str\_\_}(\textit{self})

\setlength{\parskip}{2ex}
    str(x)

\setlength{\parskip}{1ex}
      Overrides: object.\_\_str\_\_ 	extit{(inherited documentation)}

    \end{boxedminipage}


\large{\textbf{\textit{Inherited from exceptions.Exception}}}

\begin{quote}
\_\_new\_\_()
\end{quote}

\large{\textbf{\textit{Inherited from exceptions.BaseException}}}

\begin{quote}
\_\_delattr\_\_(), \_\_getattribute\_\_(), \_\_getitem\_\_(), \_\_getslice\_\_(), \_\_reduce\_\_(), \_\_repr\_\_(), \_\_setattr\_\_(), \_\_setstate\_\_()
\end{quote}

\large{\textbf{\textit{Inherited from object}}}

\begin{quote}
\_\_hash\_\_(), \_\_reduce\_ex\_\_()
\end{quote}

%%%%%%%%%%%%%%%%%%%%%%%%%%%%%%%%%%%%%%%%%%%%%%%%%%%%%%%%%%%%%%%%%%%%%%%%%%%
%%                              Properties                               %%
%%%%%%%%%%%%%%%%%%%%%%%%%%%%%%%%%%%%%%%%%%%%%%%%%%%%%%%%%%%%%%%%%%%%%%%%%%%

  \subsubsection{Properties}

    \vspace{-1cm}
\hspace{\varindent}\begin{longtable}{|p{\varnamewidth}|p{\vardescrwidth}|l}
\cline{1-2}
\cline{1-2} \centering \textbf{Name} & \centering \textbf{Description}& \\
\cline{1-2}
\endhead\cline{1-2}\multicolumn{3}{r}{\small\textit{continued on next page}}\\\endfoot\cline{1-2}
\endlastfoot\multicolumn{2}{|l|}{\textit{Inherited from exceptions.BaseException}}\\
\multicolumn{2}{|p{\varwidth}|}{\raggedright args, message}\\
\cline{1-2}
\multicolumn{2}{|l|}{\textit{Inherited from object}}\\
\multicolumn{2}{|p{\varwidth}|}{\raggedright \_\_class\_\_}\\
\cline{1-2}
\end{longtable}

    \index{SatStress \textit{(package)}!SatStress.SatStress \textit{(module)}!SatStress.SatStress.MissingSatelliteParamError \textit{(class)}|)}

%%%%%%%%%%%%%%%%%%%%%%%%%%%%%%%%%%%%%%%%%%%%%%%%%%%%%%%%%%%%%%%%%%%%%%%%%%%
%%                           Class Description                           %%
%%%%%%%%%%%%%%%%%%%%%%%%%%%%%%%%%%%%%%%%%%%%%%%%%%%%%%%%%%%%%%%%%%%%%%%%%%%

    \index{SatStress \textit{(package)}!SatStress.SatStress \textit{(module)}!SatStress.SatStress.InvalidSatelliteParamError \textit{(class)}|(}
\subsection{Class InvalidSatelliteParamError}

    \label{SatStress:SatStress:InvalidSatelliteParamError}
\begin{tabular}{cccccccccccccc}
% Line for object, linespec=[False, False, False, False, False]
\multicolumn{2}{r}{\settowidth{\BCL}{object}\multirow{2}{\BCL}{object}}
&&
&&
&&
&&
&&
  \\\cline{3-3}
  &&\multicolumn{1}{c|}{}
&&
&&
&&
&&
&&
  \\
% Line for exceptions.BaseException, linespec=[False, False, False, False]
\multicolumn{4}{r}{\settowidth{\BCL}{exceptions.BaseException}\multirow{2}{\BCL}{exceptions.BaseException}}
&&
&&
&&
&&
  \\\cline{5-5}
  &&&&\multicolumn{1}{c|}{}
&&
&&
&&
&&
  \\
% Line for exceptions.Exception, linespec=[False, False, False]
\multicolumn{6}{r}{\settowidth{\BCL}{exceptions.Exception}\multirow{2}{\BCL}{exceptions.Exception}}
&&
&&
&&
  \\\cline{7-7}
  &&&&&&\multicolumn{1}{c|}{}
&&
&&
&&
  \\
% Line for SatStress.SatStress.Error, linespec=[False, False]
\multicolumn{8}{r}{\settowidth{\BCL}{SatStress.SatStress.Error}\multirow{2}{\BCL}{SatStress.SatStress.Error}}
&&
&&
  \\\cline{9-9}
  &&&&&&&&\multicolumn{1}{c|}{}
&&
&&
  \\
% Line for SatStress.SatStress.SatelliteParamError, linespec=[False]
\multicolumn{10}{r}{\settowidth{\BCL}{SatStress.SatStress.SatelliteParamError}\multirow{2}{\BCL}{SatStress.SatStress.SatelliteParamError}}
&&
  \\\cline{11-11}
  &&&&&&&&&&\multicolumn{1}{c|}{}
&&
  \\
&&&&&&&&&&\multicolumn{2}{l}{\textbf{SatStress.SatStress.InvalidSatelliteParamError}}
\end{tabular}

\textbf{Known Subclasses:}
SatStress.SatStress.ExcessiveSatelliteMassError,
    SatStress.SatStress.GravitationallyUnstableSatelliteError,
    SatStress.SatStress.InvalidLoveNumberError,
    SatStress.SatStress.LargeEccentricityError,
    SatStress.SatStress.LoveExcessiveDeltaError,
    SatStress.SatStress.LoveLayerNumberError,
    SatStress.SatStress.LowLayerDensityError,
    SatStress.SatStress.LowLayerThicknessError,
    SatStress.SatStress.NegativeLayerParamError,
    SatStress.SatStress.NegativeNSRPeriodError,
    SatStress.SatStress.NonNumberSatelliteParamError

Raised when a required parameter is not found in the input file.


%%%%%%%%%%%%%%%%%%%%%%%%%%%%%%%%%%%%%%%%%%%%%%%%%%%%%%%%%%%%%%%%%%%%%%%%%%%
%%                                Methods                                %%
%%%%%%%%%%%%%%%%%%%%%%%%%%%%%%%%%%%%%%%%%%%%%%%%%%%%%%%%%%%%%%%%%%%%%%%%%%%

  \subsubsection{Methods}

    \vspace{0.5ex}

\hspace{.8\funcindent}\begin{boxedminipage}{\funcwidth}

    \raggedright \textbf{\_\_init\_\_}(\textit{self}, \textit{sat})

    \vspace{-1.5ex}

    \rule{\textwidth}{0.5\fboxrule}
\setlength{\parskip}{2ex}
    Default initialization of an InvalidSatelliteParamError

    Simply sets self.sat = sat (a Satellite object).  Most errors can be 
    well described using only the parameters stored in the Satellite 
    object.

\setlength{\parskip}{1ex}
      Overrides: object.\_\_init\_\_

    \end{boxedminipage}


\large{\textbf{\textit{Inherited from exceptions.Exception}}}

\begin{quote}
\_\_new\_\_()
\end{quote}

\large{\textbf{\textit{Inherited from exceptions.BaseException}}}

\begin{quote}
\_\_delattr\_\_(), \_\_getattribute\_\_(), \_\_getitem\_\_(), \_\_getslice\_\_(), \_\_reduce\_\_(), \_\_repr\_\_(), \_\_setattr\_\_(), \_\_setstate\_\_(), \_\_str\_\_()
\end{quote}

\large{\textbf{\textit{Inherited from object}}}

\begin{quote}
\_\_hash\_\_(), \_\_reduce\_ex\_\_()
\end{quote}

%%%%%%%%%%%%%%%%%%%%%%%%%%%%%%%%%%%%%%%%%%%%%%%%%%%%%%%%%%%%%%%%%%%%%%%%%%%
%%                              Properties                               %%
%%%%%%%%%%%%%%%%%%%%%%%%%%%%%%%%%%%%%%%%%%%%%%%%%%%%%%%%%%%%%%%%%%%%%%%%%%%

  \subsubsection{Properties}

    \vspace{-1cm}
\hspace{\varindent}\begin{longtable}{|p{\varnamewidth}|p{\vardescrwidth}|l}
\cline{1-2}
\cline{1-2} \centering \textbf{Name} & \centering \textbf{Description}& \\
\cline{1-2}
\endhead\cline{1-2}\multicolumn{3}{r}{\small\textit{continued on next page}}\\\endfoot\cline{1-2}
\endlastfoot\multicolumn{2}{|l|}{\textit{Inherited from exceptions.BaseException}}\\
\multicolumn{2}{|p{\varwidth}|}{\raggedright args, message}\\
\cline{1-2}
\multicolumn{2}{|l|}{\textit{Inherited from object}}\\
\multicolumn{2}{|p{\varwidth}|}{\raggedright \_\_class\_\_}\\
\cline{1-2}
\end{longtable}

    \index{SatStress \textit{(package)}!SatStress.SatStress \textit{(module)}!SatStress.SatStress.InvalidSatelliteParamError \textit{(class)}|)}

%%%%%%%%%%%%%%%%%%%%%%%%%%%%%%%%%%%%%%%%%%%%%%%%%%%%%%%%%%%%%%%%%%%%%%%%%%%
%%                           Class Description                           %%
%%%%%%%%%%%%%%%%%%%%%%%%%%%%%%%%%%%%%%%%%%%%%%%%%%%%%%%%%%%%%%%%%%%%%%%%%%%

    \index{SatStress \textit{(package)}!SatStress.SatStress \textit{(module)}!SatStress.SatStress.LargeEccentricityError \textit{(class)}|(}
\subsection{Class LargeEccentricityError}

    \label{SatStress:SatStress:LargeEccentricityError}
\begin{tabular}{cccccccccccccccc}
% Line for object, linespec=[False, False, False, False, False, False]
\multicolumn{2}{r}{\settowidth{\BCL}{object}\multirow{2}{\BCL}{object}}
&&
&&
&&
&&
&&
&&
  \\\cline{3-3}
  &&\multicolumn{1}{c|}{}
&&
&&
&&
&&
&&
&&
  \\
% Line for exceptions.BaseException, linespec=[False, False, False, False, False]
\multicolumn{4}{r}{\settowidth{\BCL}{exceptions.BaseException}\multirow{2}{\BCL}{exceptions.BaseException}}
&&
&&
&&
&&
&&
  \\\cline{5-5}
  &&&&\multicolumn{1}{c|}{}
&&
&&
&&
&&
&&
  \\
% Line for exceptions.Exception, linespec=[False, False, False, False]
\multicolumn{6}{r}{\settowidth{\BCL}{exceptions.Exception}\multirow{2}{\BCL}{exceptions.Exception}}
&&
&&
&&
&&
  \\\cline{7-7}
  &&&&&&\multicolumn{1}{c|}{}
&&
&&
&&
&&
  \\
% Line for SatStress.SatStress.Error, linespec=[False, False, False]
\multicolumn{8}{r}{\settowidth{\BCL}{SatStress.SatStress.Error}\multirow{2}{\BCL}{SatStress.SatStress.Error}}
&&
&&
&&
  \\\cline{9-9}
  &&&&&&&&\multicolumn{1}{c|}{}
&&
&&
&&
  \\
% Line for SatStress.SatStress.SatelliteParamError, linespec=[False, False]
\multicolumn{10}{r}{\settowidth{\BCL}{SatStress.SatStress.SatelliteParamError}\multirow{2}{\BCL}{SatStress.SatStress.SatelliteParamError}}
&&
&&
  \\\cline{11-11}
  &&&&&&&&&&\multicolumn{1}{c|}{}
&&
&&
  \\
% Line for SatStress.SatStress.InvalidSatelliteParamError, linespec=[False]
\multicolumn{12}{r}{\settowidth{\BCL}{SatStress.SatStress.InvalidSatelliteParamError}\multirow{2}{\BCL}{SatStress.SatStress.InvalidSatelliteParamError}}
&&
  \\\cline{13-13}
  &&&&&&&&&&&&\multicolumn{1}{c|}{}
&&
  \\
&&&&&&&&&&&&\multicolumn{2}{l}{\textbf{SatStress.SatStress.LargeEccentricityError}}
\end{tabular}

Raised when satellite orbital eccentricity is {\textgreater} 0.25


%%%%%%%%%%%%%%%%%%%%%%%%%%%%%%%%%%%%%%%%%%%%%%%%%%%%%%%%%%%%%%%%%%%%%%%%%%%
%%                                Methods                                %%
%%%%%%%%%%%%%%%%%%%%%%%%%%%%%%%%%%%%%%%%%%%%%%%%%%%%%%%%%%%%%%%%%%%%%%%%%%%

  \subsubsection{Methods}

    \vspace{0.5ex}

\hspace{.8\funcindent}\begin{boxedminipage}{\funcwidth}

    \raggedright \textbf{\_\_str\_\_}(\textit{self})

\setlength{\parskip}{2ex}
    str(x)

\setlength{\parskip}{1ex}
      Overrides: object.\_\_str\_\_ 	extit{(inherited documentation)}

    \end{boxedminipage}


\large{\textbf{\textit{Inherited from SatStress.SatStress.InvalidSatelliteParamError\textit{(Section \ref{SatStress:SatStress:InvalidSatelliteParamError})}}}}

\begin{quote}
\_\_init\_\_()
\end{quote}

\large{\textbf{\textit{Inherited from exceptions.Exception}}}

\begin{quote}
\_\_new\_\_()
\end{quote}

\large{\textbf{\textit{Inherited from exceptions.BaseException}}}

\begin{quote}
\_\_delattr\_\_(), \_\_getattribute\_\_(), \_\_getitem\_\_(), \_\_getslice\_\_(), \_\_reduce\_\_(), \_\_repr\_\_(), \_\_setattr\_\_(), \_\_setstate\_\_()
\end{quote}

\large{\textbf{\textit{Inherited from object}}}

\begin{quote}
\_\_hash\_\_(), \_\_reduce\_ex\_\_()
\end{quote}

%%%%%%%%%%%%%%%%%%%%%%%%%%%%%%%%%%%%%%%%%%%%%%%%%%%%%%%%%%%%%%%%%%%%%%%%%%%
%%                              Properties                               %%
%%%%%%%%%%%%%%%%%%%%%%%%%%%%%%%%%%%%%%%%%%%%%%%%%%%%%%%%%%%%%%%%%%%%%%%%%%%

  \subsubsection{Properties}

    \vspace{-1cm}
\hspace{\varindent}\begin{longtable}{|p{\varnamewidth}|p{\vardescrwidth}|l}
\cline{1-2}
\cline{1-2} \centering \textbf{Name} & \centering \textbf{Description}& \\
\cline{1-2}
\endhead\cline{1-2}\multicolumn{3}{r}{\small\textit{continued on next page}}\\\endfoot\cline{1-2}
\endlastfoot\multicolumn{2}{|l|}{\textit{Inherited from exceptions.BaseException}}\\
\multicolumn{2}{|p{\varwidth}|}{\raggedright args, message}\\
\cline{1-2}
\multicolumn{2}{|l|}{\textit{Inherited from object}}\\
\multicolumn{2}{|p{\varwidth}|}{\raggedright \_\_class\_\_}\\
\cline{1-2}
\end{longtable}

    \index{SatStress \textit{(package)}!SatStress.SatStress \textit{(module)}!SatStress.SatStress.LargeEccentricityError \textit{(class)}|)}

%%%%%%%%%%%%%%%%%%%%%%%%%%%%%%%%%%%%%%%%%%%%%%%%%%%%%%%%%%%%%%%%%%%%%%%%%%%
%%                           Class Description                           %%
%%%%%%%%%%%%%%%%%%%%%%%%%%%%%%%%%%%%%%%%%%%%%%%%%%%%%%%%%%%%%%%%%%%%%%%%%%%

    \index{SatStress \textit{(package)}!SatStress.SatStress \textit{(module)}!SatStress.SatStress.NegativeNSRPeriodError \textit{(class)}|(}
\subsection{Class NegativeNSRPeriodError}

    \label{SatStress:SatStress:NegativeNSRPeriodError}
\begin{tabular}{cccccccccccccccc}
% Line for object, linespec=[False, False, False, False, False, False]
\multicolumn{2}{r}{\settowidth{\BCL}{object}\multirow{2}{\BCL}{object}}
&&
&&
&&
&&
&&
&&
  \\\cline{3-3}
  &&\multicolumn{1}{c|}{}
&&
&&
&&
&&
&&
&&
  \\
% Line for exceptions.BaseException, linespec=[False, False, False, False, False]
\multicolumn{4}{r}{\settowidth{\BCL}{exceptions.BaseException}\multirow{2}{\BCL}{exceptions.BaseException}}
&&
&&
&&
&&
&&
  \\\cline{5-5}
  &&&&\multicolumn{1}{c|}{}
&&
&&
&&
&&
&&
  \\
% Line for exceptions.Exception, linespec=[False, False, False, False]
\multicolumn{6}{r}{\settowidth{\BCL}{exceptions.Exception}\multirow{2}{\BCL}{exceptions.Exception}}
&&
&&
&&
&&
  \\\cline{7-7}
  &&&&&&\multicolumn{1}{c|}{}
&&
&&
&&
&&
  \\
% Line for SatStress.SatStress.Error, linespec=[False, False, False]
\multicolumn{8}{r}{\settowidth{\BCL}{SatStress.SatStress.Error}\multirow{2}{\BCL}{SatStress.SatStress.Error}}
&&
&&
&&
  \\\cline{9-9}
  &&&&&&&&\multicolumn{1}{c|}{}
&&
&&
&&
  \\
% Line for SatStress.SatStress.SatelliteParamError, linespec=[False, False]
\multicolumn{10}{r}{\settowidth{\BCL}{SatStress.SatStress.SatelliteParamError}\multirow{2}{\BCL}{SatStress.SatStress.SatelliteParamError}}
&&
&&
  \\\cline{11-11}
  &&&&&&&&&&\multicolumn{1}{c|}{}
&&
&&
  \\
% Line for SatStress.SatStress.InvalidSatelliteParamError, linespec=[False]
\multicolumn{12}{r}{\settowidth{\BCL}{SatStress.SatStress.InvalidSatelliteParamError}\multirow{2}{\BCL}{SatStress.SatStress.InvalidSatelliteParamError}}
&&
  \\\cline{13-13}
  &&&&&&&&&&&&\multicolumn{1}{c|}{}
&&
  \\
&&&&&&&&&&&&\multicolumn{2}{l}{\textbf{SatStress.SatStress.NegativeNSRPeriodError}}
\end{tabular}

Raised if the satellite's NSR period is less than zero


%%%%%%%%%%%%%%%%%%%%%%%%%%%%%%%%%%%%%%%%%%%%%%%%%%%%%%%%%%%%%%%%%%%%%%%%%%%
%%                                Methods                                %%
%%%%%%%%%%%%%%%%%%%%%%%%%%%%%%%%%%%%%%%%%%%%%%%%%%%%%%%%%%%%%%%%%%%%%%%%%%%

  \subsubsection{Methods}

    \vspace{0.5ex}

\hspace{.8\funcindent}\begin{boxedminipage}{\funcwidth}

    \raggedright \textbf{\_\_str\_\_}(\textit{self})

\setlength{\parskip}{2ex}
    str(x)

\setlength{\parskip}{1ex}
      Overrides: object.\_\_str\_\_ 	extit{(inherited documentation)}

    \end{boxedminipage}


\large{\textbf{\textit{Inherited from SatStress.SatStress.InvalidSatelliteParamError\textit{(Section \ref{SatStress:SatStress:InvalidSatelliteParamError})}}}}

\begin{quote}
\_\_init\_\_()
\end{quote}

\large{\textbf{\textit{Inherited from exceptions.Exception}}}

\begin{quote}
\_\_new\_\_()
\end{quote}

\large{\textbf{\textit{Inherited from exceptions.BaseException}}}

\begin{quote}
\_\_delattr\_\_(), \_\_getattribute\_\_(), \_\_getitem\_\_(), \_\_getslice\_\_(), \_\_reduce\_\_(), \_\_repr\_\_(), \_\_setattr\_\_(), \_\_setstate\_\_()
\end{quote}

\large{\textbf{\textit{Inherited from object}}}

\begin{quote}
\_\_hash\_\_(), \_\_reduce\_ex\_\_()
\end{quote}

%%%%%%%%%%%%%%%%%%%%%%%%%%%%%%%%%%%%%%%%%%%%%%%%%%%%%%%%%%%%%%%%%%%%%%%%%%%
%%                              Properties                               %%
%%%%%%%%%%%%%%%%%%%%%%%%%%%%%%%%%%%%%%%%%%%%%%%%%%%%%%%%%%%%%%%%%%%%%%%%%%%

  \subsubsection{Properties}

    \vspace{-1cm}
\hspace{\varindent}\begin{longtable}{|p{\varnamewidth}|p{\vardescrwidth}|l}
\cline{1-2}
\cline{1-2} \centering \textbf{Name} & \centering \textbf{Description}& \\
\cline{1-2}
\endhead\cline{1-2}\multicolumn{3}{r}{\small\textit{continued on next page}}\\\endfoot\cline{1-2}
\endlastfoot\multicolumn{2}{|l|}{\textit{Inherited from exceptions.BaseException}}\\
\multicolumn{2}{|p{\varwidth}|}{\raggedright args, message}\\
\cline{1-2}
\multicolumn{2}{|l|}{\textit{Inherited from object}}\\
\multicolumn{2}{|p{\varwidth}|}{\raggedright \_\_class\_\_}\\
\cline{1-2}
\end{longtable}

    \index{SatStress \textit{(package)}!SatStress.SatStress \textit{(module)}!SatStress.SatStress.NegativeNSRPeriodError \textit{(class)}|)}

%%%%%%%%%%%%%%%%%%%%%%%%%%%%%%%%%%%%%%%%%%%%%%%%%%%%%%%%%%%%%%%%%%%%%%%%%%%
%%                           Class Description                           %%
%%%%%%%%%%%%%%%%%%%%%%%%%%%%%%%%%%%%%%%%%%%%%%%%%%%%%%%%%%%%%%%%%%%%%%%%%%%

    \index{SatStress \textit{(package)}!SatStress.SatStress \textit{(module)}!SatStress.SatStress.ExcessiveSatelliteMassError \textit{(class)}|(}
\subsection{Class ExcessiveSatelliteMassError}

    \label{SatStress:SatStress:ExcessiveSatelliteMassError}
\begin{tabular}{cccccccccccccccc}
% Line for object, linespec=[False, False, False, False, False, False]
\multicolumn{2}{r}{\settowidth{\BCL}{object}\multirow{2}{\BCL}{object}}
&&
&&
&&
&&
&&
&&
  \\\cline{3-3}
  &&\multicolumn{1}{c|}{}
&&
&&
&&
&&
&&
&&
  \\
% Line for exceptions.BaseException, linespec=[False, False, False, False, False]
\multicolumn{4}{r}{\settowidth{\BCL}{exceptions.BaseException}\multirow{2}{\BCL}{exceptions.BaseException}}
&&
&&
&&
&&
&&
  \\\cline{5-5}
  &&&&\multicolumn{1}{c|}{}
&&
&&
&&
&&
&&
  \\
% Line for exceptions.Exception, linespec=[False, False, False, False]
\multicolumn{6}{r}{\settowidth{\BCL}{exceptions.Exception}\multirow{2}{\BCL}{exceptions.Exception}}
&&
&&
&&
&&
  \\\cline{7-7}
  &&&&&&\multicolumn{1}{c|}{}
&&
&&
&&
&&
  \\
% Line for SatStress.SatStress.Error, linespec=[False, False, False]
\multicolumn{8}{r}{\settowidth{\BCL}{SatStress.SatStress.Error}\multirow{2}{\BCL}{SatStress.SatStress.Error}}
&&
&&
&&
  \\\cline{9-9}
  &&&&&&&&\multicolumn{1}{c|}{}
&&
&&
&&
  \\
% Line for SatStress.SatStress.SatelliteParamError, linespec=[False, False]
\multicolumn{10}{r}{\settowidth{\BCL}{SatStress.SatStress.SatelliteParamError}\multirow{2}{\BCL}{SatStress.SatStress.SatelliteParamError}}
&&
&&
  \\\cline{11-11}
  &&&&&&&&&&\multicolumn{1}{c|}{}
&&
&&
  \\
% Line for SatStress.SatStress.InvalidSatelliteParamError, linespec=[False]
\multicolumn{12}{r}{\settowidth{\BCL}{SatStress.SatStress.InvalidSatelliteParamError}\multirow{2}{\BCL}{SatStress.SatStress.InvalidSatelliteParamError}}
&&
  \\\cline{13-13}
  &&&&&&&&&&&&\multicolumn{1}{c|}{}
&&
  \\
&&&&&&&&&&&&\multicolumn{2}{l}{\textbf{SatStress.SatStress.ExcessiveSatelliteMassError}}
\end{tabular}

Raised if the satellite's parent planet is less than 10x as massive as the 
satellite.


%%%%%%%%%%%%%%%%%%%%%%%%%%%%%%%%%%%%%%%%%%%%%%%%%%%%%%%%%%%%%%%%%%%%%%%%%%%
%%                                Methods                                %%
%%%%%%%%%%%%%%%%%%%%%%%%%%%%%%%%%%%%%%%%%%%%%%%%%%%%%%%%%%%%%%%%%%%%%%%%%%%

  \subsubsection{Methods}

    \vspace{0.5ex}

\hspace{.8\funcindent}\begin{boxedminipage}{\funcwidth}

    \raggedright \textbf{\_\_str\_\_}(\textit{self})

\setlength{\parskip}{2ex}
    str(x)

\setlength{\parskip}{1ex}
      Overrides: object.\_\_str\_\_ 	extit{(inherited documentation)}

    \end{boxedminipage}


\large{\textbf{\textit{Inherited from SatStress.SatStress.InvalidSatelliteParamError\textit{(Section \ref{SatStress:SatStress:InvalidSatelliteParamError})}}}}

\begin{quote}
\_\_init\_\_()
\end{quote}

\large{\textbf{\textit{Inherited from exceptions.Exception}}}

\begin{quote}
\_\_new\_\_()
\end{quote}

\large{\textbf{\textit{Inherited from exceptions.BaseException}}}

\begin{quote}
\_\_delattr\_\_(), \_\_getattribute\_\_(), \_\_getitem\_\_(), \_\_getslice\_\_(), \_\_reduce\_\_(), \_\_repr\_\_(), \_\_setattr\_\_(), \_\_setstate\_\_()
\end{quote}

\large{\textbf{\textit{Inherited from object}}}

\begin{quote}
\_\_hash\_\_(), \_\_reduce\_ex\_\_()
\end{quote}

%%%%%%%%%%%%%%%%%%%%%%%%%%%%%%%%%%%%%%%%%%%%%%%%%%%%%%%%%%%%%%%%%%%%%%%%%%%
%%                              Properties                               %%
%%%%%%%%%%%%%%%%%%%%%%%%%%%%%%%%%%%%%%%%%%%%%%%%%%%%%%%%%%%%%%%%%%%%%%%%%%%

  \subsubsection{Properties}

    \vspace{-1cm}
\hspace{\varindent}\begin{longtable}{|p{\varnamewidth}|p{\vardescrwidth}|l}
\cline{1-2}
\cline{1-2} \centering \textbf{Name} & \centering \textbf{Description}& \\
\cline{1-2}
\endhead\cline{1-2}\multicolumn{3}{r}{\small\textit{continued on next page}}\\\endfoot\cline{1-2}
\endlastfoot\multicolumn{2}{|l|}{\textit{Inherited from exceptions.BaseException}}\\
\multicolumn{2}{|p{\varwidth}|}{\raggedright args, message}\\
\cline{1-2}
\multicolumn{2}{|l|}{\textit{Inherited from object}}\\
\multicolumn{2}{|p{\varwidth}|}{\raggedright \_\_class\_\_}\\
\cline{1-2}
\end{longtable}

    \index{SatStress \textit{(package)}!SatStress.SatStress \textit{(module)}!SatStress.SatStress.ExcessiveSatelliteMassError \textit{(class)}|)}

%%%%%%%%%%%%%%%%%%%%%%%%%%%%%%%%%%%%%%%%%%%%%%%%%%%%%%%%%%%%%%%%%%%%%%%%%%%
%%                           Class Description                           %%
%%%%%%%%%%%%%%%%%%%%%%%%%%%%%%%%%%%%%%%%%%%%%%%%%%%%%%%%%%%%%%%%%%%%%%%%%%%

    \index{SatStress \textit{(package)}!SatStress.SatStress \textit{(module)}!SatStress.SatStress.LoveLayerNumberError \textit{(class)}|(}
\subsection{Class LoveLayerNumberError}

    \label{SatStress:SatStress:LoveLayerNumberError}
\begin{tabular}{cccccccccccccccc}
% Line for object, linespec=[False, False, False, False, False, False]
\multicolumn{2}{r}{\settowidth{\BCL}{object}\multirow{2}{\BCL}{object}}
&&
&&
&&
&&
&&
&&
  \\\cline{3-3}
  &&\multicolumn{1}{c|}{}
&&
&&
&&
&&
&&
&&
  \\
% Line for exceptions.BaseException, linespec=[False, False, False, False, False]
\multicolumn{4}{r}{\settowidth{\BCL}{exceptions.BaseException}\multirow{2}{\BCL}{exceptions.BaseException}}
&&
&&
&&
&&
&&
  \\\cline{5-5}
  &&&&\multicolumn{1}{c|}{}
&&
&&
&&
&&
&&
  \\
% Line for exceptions.Exception, linespec=[False, False, False, False]
\multicolumn{6}{r}{\settowidth{\BCL}{exceptions.Exception}\multirow{2}{\BCL}{exceptions.Exception}}
&&
&&
&&
&&
  \\\cline{7-7}
  &&&&&&\multicolumn{1}{c|}{}
&&
&&
&&
&&
  \\
% Line for SatStress.SatStress.Error, linespec=[False, False, False]
\multicolumn{8}{r}{\settowidth{\BCL}{SatStress.SatStress.Error}\multirow{2}{\BCL}{SatStress.SatStress.Error}}
&&
&&
&&
  \\\cline{9-9}
  &&&&&&&&\multicolumn{1}{c|}{}
&&
&&
&&
  \\
% Line for SatStress.SatStress.SatelliteParamError, linespec=[False, False]
\multicolumn{10}{r}{\settowidth{\BCL}{SatStress.SatStress.SatelliteParamError}\multirow{2}{\BCL}{SatStress.SatStress.SatelliteParamError}}
&&
&&
  \\\cline{11-11}
  &&&&&&&&&&\multicolumn{1}{c|}{}
&&
&&
  \\
% Line for SatStress.SatStress.InvalidSatelliteParamError, linespec=[False]
\multicolumn{12}{r}{\settowidth{\BCL}{SatStress.SatStress.InvalidSatelliteParamError}\multirow{2}{\BCL}{SatStress.SatStress.InvalidSatelliteParamError}}
&&
  \\\cline{13-13}
  &&&&&&&&&&&&\multicolumn{1}{c|}{}
&&
  \\
&&&&&&&&&&&&\multicolumn{2}{l}{\textbf{SatStress.SatStress.LoveLayerNumberError}}
\end{tabular}

Raised if the number of layers specified in the satellite definition file 
is incompatible with the Love number code.


%%%%%%%%%%%%%%%%%%%%%%%%%%%%%%%%%%%%%%%%%%%%%%%%%%%%%%%%%%%%%%%%%%%%%%%%%%%
%%                                Methods                                %%
%%%%%%%%%%%%%%%%%%%%%%%%%%%%%%%%%%%%%%%%%%%%%%%%%%%%%%%%%%%%%%%%%%%%%%%%%%%

  \subsubsection{Methods}

    \vspace{0.5ex}

\hspace{.8\funcindent}\begin{boxedminipage}{\funcwidth}

    \raggedright \textbf{\_\_str\_\_}(\textit{self})

\setlength{\parskip}{2ex}
    str(x)

\setlength{\parskip}{1ex}
      Overrides: object.\_\_str\_\_ 	extit{(inherited documentation)}

    \end{boxedminipage}


\large{\textbf{\textit{Inherited from SatStress.SatStress.InvalidSatelliteParamError\textit{(Section \ref{SatStress:SatStress:InvalidSatelliteParamError})}}}}

\begin{quote}
\_\_init\_\_()
\end{quote}

\large{\textbf{\textit{Inherited from exceptions.Exception}}}

\begin{quote}
\_\_new\_\_()
\end{quote}

\large{\textbf{\textit{Inherited from exceptions.BaseException}}}

\begin{quote}
\_\_delattr\_\_(), \_\_getattribute\_\_(), \_\_getitem\_\_(), \_\_getslice\_\_(), \_\_reduce\_\_(), \_\_repr\_\_(), \_\_setattr\_\_(), \_\_setstate\_\_()
\end{quote}

\large{\textbf{\textit{Inherited from object}}}

\begin{quote}
\_\_hash\_\_(), \_\_reduce\_ex\_\_()
\end{quote}

%%%%%%%%%%%%%%%%%%%%%%%%%%%%%%%%%%%%%%%%%%%%%%%%%%%%%%%%%%%%%%%%%%%%%%%%%%%
%%                              Properties                               %%
%%%%%%%%%%%%%%%%%%%%%%%%%%%%%%%%%%%%%%%%%%%%%%%%%%%%%%%%%%%%%%%%%%%%%%%%%%%

  \subsubsection{Properties}

    \vspace{-1cm}
\hspace{\varindent}\begin{longtable}{|p{\varnamewidth}|p{\vardescrwidth}|l}
\cline{1-2}
\cline{1-2} \centering \textbf{Name} & \centering \textbf{Description}& \\
\cline{1-2}
\endhead\cline{1-2}\multicolumn{3}{r}{\small\textit{continued on next page}}\\\endfoot\cline{1-2}
\endlastfoot\multicolumn{2}{|l|}{\textit{Inherited from exceptions.BaseException}}\\
\multicolumn{2}{|p{\varwidth}|}{\raggedright args, message}\\
\cline{1-2}
\multicolumn{2}{|l|}{\textit{Inherited from object}}\\
\multicolumn{2}{|p{\varwidth}|}{\raggedright \_\_class\_\_}\\
\cline{1-2}
\end{longtable}

    \index{SatStress \textit{(package)}!SatStress.SatStress \textit{(module)}!SatStress.SatStress.LoveLayerNumberError \textit{(class)}|)}

%%%%%%%%%%%%%%%%%%%%%%%%%%%%%%%%%%%%%%%%%%%%%%%%%%%%%%%%%%%%%%%%%%%%%%%%%%%
%%                           Class Description                           %%
%%%%%%%%%%%%%%%%%%%%%%%%%%%%%%%%%%%%%%%%%%%%%%%%%%%%%%%%%%%%%%%%%%%%%%%%%%%

    \index{SatStress \textit{(package)}!SatStress.SatStress \textit{(module)}!SatStress.SatStress.InvalidLoveNumberError \textit{(class)}|(}
\subsection{Class InvalidLoveNumberError}

    \label{SatStress:SatStress:InvalidLoveNumberError}
\begin{tabular}{cccccccccccccccc}
% Line for object, linespec=[False, False, False, False, False, False]
\multicolumn{2}{r}{\settowidth{\BCL}{object}\multirow{2}{\BCL}{object}}
&&
&&
&&
&&
&&
&&
  \\\cline{3-3}
  &&\multicolumn{1}{c|}{}
&&
&&
&&
&&
&&
&&
  \\
% Line for exceptions.BaseException, linespec=[False, False, False, False, False]
\multicolumn{4}{r}{\settowidth{\BCL}{exceptions.BaseException}\multirow{2}{\BCL}{exceptions.BaseException}}
&&
&&
&&
&&
&&
  \\\cline{5-5}
  &&&&\multicolumn{1}{c|}{}
&&
&&
&&
&&
&&
  \\
% Line for exceptions.Exception, linespec=[False, False, False, False]
\multicolumn{6}{r}{\settowidth{\BCL}{exceptions.Exception}\multirow{2}{\BCL}{exceptions.Exception}}
&&
&&
&&
&&
  \\\cline{7-7}
  &&&&&&\multicolumn{1}{c|}{}
&&
&&
&&
&&
  \\
% Line for SatStress.SatStress.Error, linespec=[False, False, False]
\multicolumn{8}{r}{\settowidth{\BCL}{SatStress.SatStress.Error}\multirow{2}{\BCL}{SatStress.SatStress.Error}}
&&
&&
&&
  \\\cline{9-9}
  &&&&&&&&\multicolumn{1}{c|}{}
&&
&&
&&
  \\
% Line for SatStress.SatStress.SatelliteParamError, linespec=[False, False]
\multicolumn{10}{r}{\settowidth{\BCL}{SatStress.SatStress.SatelliteParamError}\multirow{2}{\BCL}{SatStress.SatStress.SatelliteParamError}}
&&
&&
  \\\cline{11-11}
  &&&&&&&&&&\multicolumn{1}{c|}{}
&&
&&
  \\
% Line for SatStress.SatStress.InvalidSatelliteParamError, linespec=[False]
\multicolumn{12}{r}{\settowidth{\BCL}{SatStress.SatStress.InvalidSatelliteParamError}\multirow{2}{\BCL}{SatStress.SatStress.InvalidSatelliteParamError}}
&&
  \\\cline{13-13}
  &&&&&&&&&&&&\multicolumn{1}{c|}{}
&&
  \\
&&&&&&&&&&&&\multicolumn{2}{l}{\textbf{SatStress.SatStress.InvalidLoveNumberError}}
\end{tabular}

Raised if the Love numbers are found to be suspicious.


%%%%%%%%%%%%%%%%%%%%%%%%%%%%%%%%%%%%%%%%%%%%%%%%%%%%%%%%%%%%%%%%%%%%%%%%%%%
%%                                Methods                                %%
%%%%%%%%%%%%%%%%%%%%%%%%%%%%%%%%%%%%%%%%%%%%%%%%%%%%%%%%%%%%%%%%%%%%%%%%%%%

  \subsubsection{Methods}

    \vspace{0.5ex}

\hspace{.8\funcindent}\begin{boxedminipage}{\funcwidth}

    \raggedright \textbf{\_\_init\_\_}(\textit{self}, \textit{stress}, \textit{love})

\setlength{\parskip}{2ex}
    Default initialization of an InvalidSatelliteParamError

    Simply sets self.sat = sat (a Satellite object).  Most errors can be 
    well described using only the parameters stored in the Satellite 
    object.

\setlength{\parskip}{1ex}
      Overrides: object.\_\_init\_\_ 	extit{(inherited documentation)}

    \end{boxedminipage}

    \vspace{0.5ex}

\hspace{.8\funcindent}\begin{boxedminipage}{\funcwidth}

    \raggedright \textbf{\_\_str\_\_}(\textit{self})

\setlength{\parskip}{2ex}
    str(x)

\setlength{\parskip}{1ex}
      Overrides: object.\_\_str\_\_ 	extit{(inherited documentation)}

    \end{boxedminipage}


\large{\textbf{\textit{Inherited from exceptions.Exception}}}

\begin{quote}
\_\_new\_\_()
\end{quote}

\large{\textbf{\textit{Inherited from exceptions.BaseException}}}

\begin{quote}
\_\_delattr\_\_(), \_\_getattribute\_\_(), \_\_getitem\_\_(), \_\_getslice\_\_(), \_\_reduce\_\_(), \_\_repr\_\_(), \_\_setattr\_\_(), \_\_setstate\_\_()
\end{quote}

\large{\textbf{\textit{Inherited from object}}}

\begin{quote}
\_\_hash\_\_(), \_\_reduce\_ex\_\_()
\end{quote}

%%%%%%%%%%%%%%%%%%%%%%%%%%%%%%%%%%%%%%%%%%%%%%%%%%%%%%%%%%%%%%%%%%%%%%%%%%%
%%                              Properties                               %%
%%%%%%%%%%%%%%%%%%%%%%%%%%%%%%%%%%%%%%%%%%%%%%%%%%%%%%%%%%%%%%%%%%%%%%%%%%%

  \subsubsection{Properties}

    \vspace{-1cm}
\hspace{\varindent}\begin{longtable}{|p{\varnamewidth}|p{\vardescrwidth}|l}
\cline{1-2}
\cline{1-2} \centering \textbf{Name} & \centering \textbf{Description}& \\
\cline{1-2}
\endhead\cline{1-2}\multicolumn{3}{r}{\small\textit{continued on next page}}\\\endfoot\cline{1-2}
\endlastfoot\multicolumn{2}{|l|}{\textit{Inherited from exceptions.BaseException}}\\
\multicolumn{2}{|p{\varwidth}|}{\raggedright args, message}\\
\cline{1-2}
\multicolumn{2}{|l|}{\textit{Inherited from object}}\\
\multicolumn{2}{|p{\varwidth}|}{\raggedright \_\_class\_\_}\\
\cline{1-2}
\end{longtable}

    \index{SatStress \textit{(package)}!SatStress.SatStress \textit{(module)}!SatStress.SatStress.InvalidLoveNumberError \textit{(class)}|)}

%%%%%%%%%%%%%%%%%%%%%%%%%%%%%%%%%%%%%%%%%%%%%%%%%%%%%%%%%%%%%%%%%%%%%%%%%%%
%%                           Class Description                           %%
%%%%%%%%%%%%%%%%%%%%%%%%%%%%%%%%%%%%%%%%%%%%%%%%%%%%%%%%%%%%%%%%%%%%%%%%%%%

    \index{SatStress \textit{(package)}!SatStress.SatStress \textit{(module)}!SatStress.SatStress.LoveExcessiveDeltaError \textit{(class)}|(}
\subsection{Class LoveExcessiveDeltaError}

    \label{SatStress:SatStress:LoveExcessiveDeltaError}
\begin{tabular}{cccccccccccccccc}
% Line for object, linespec=[False, False, False, False, False, False]
\multicolumn{2}{r}{\settowidth{\BCL}{object}\multirow{2}{\BCL}{object}}
&&
&&
&&
&&
&&
&&
  \\\cline{3-3}
  &&\multicolumn{1}{c|}{}
&&
&&
&&
&&
&&
&&
  \\
% Line for exceptions.BaseException, linespec=[False, False, False, False, False]
\multicolumn{4}{r}{\settowidth{\BCL}{exceptions.BaseException}\multirow{2}{\BCL}{exceptions.BaseException}}
&&
&&
&&
&&
&&
  \\\cline{5-5}
  &&&&\multicolumn{1}{c|}{}
&&
&&
&&
&&
&&
  \\
% Line for exceptions.Exception, linespec=[False, False, False, False]
\multicolumn{6}{r}{\settowidth{\BCL}{exceptions.Exception}\multirow{2}{\BCL}{exceptions.Exception}}
&&
&&
&&
&&
  \\\cline{7-7}
  &&&&&&\multicolumn{1}{c|}{}
&&
&&
&&
&&
  \\
% Line for SatStress.SatStress.Error, linespec=[False, False, False]
\multicolumn{8}{r}{\settowidth{\BCL}{SatStress.SatStress.Error}\multirow{2}{\BCL}{SatStress.SatStress.Error}}
&&
&&
&&
  \\\cline{9-9}
  &&&&&&&&\multicolumn{1}{c|}{}
&&
&&
&&
  \\
% Line for SatStress.SatStress.SatelliteParamError, linespec=[False, False]
\multicolumn{10}{r}{\settowidth{\BCL}{SatStress.SatStress.SatelliteParamError}\multirow{2}{\BCL}{SatStress.SatStress.SatelliteParamError}}
&&
&&
  \\\cline{11-11}
  &&&&&&&&&&\multicolumn{1}{c|}{}
&&
&&
  \\
% Line for SatStress.SatStress.InvalidSatelliteParamError, linespec=[False]
\multicolumn{12}{r}{\settowidth{\BCL}{SatStress.SatStress.InvalidSatelliteParamError}\multirow{2}{\BCL}{SatStress.SatStress.InvalidSatelliteParamError}}
&&
  \\\cline{13-13}
  &&&&&&&&&&&&\multicolumn{1}{c|}{}
&&
  \\
&&&&&&&&&&&&\multicolumn{2}{l}{\textbf{SatStress.SatStress.LoveExcessiveDeltaError}}
\end{tabular}

Raised when \(\Delta\) {\textgreater} 10{\textasciicircum}9 for any of the 
ice layers, at which point the Love number code becomes unreliable.


%%%%%%%%%%%%%%%%%%%%%%%%%%%%%%%%%%%%%%%%%%%%%%%%%%%%%%%%%%%%%%%%%%%%%%%%%%%
%%                                Methods                                %%
%%%%%%%%%%%%%%%%%%%%%%%%%%%%%%%%%%%%%%%%%%%%%%%%%%%%%%%%%%%%%%%%%%%%%%%%%%%

  \subsubsection{Methods}

    \vspace{0.5ex}

\hspace{.8\funcindent}\begin{boxedminipage}{\funcwidth}

    \raggedright \textbf{\_\_init\_\_}(\textit{self}, \textit{stress}, \textit{layer\_n})

\setlength{\parskip}{2ex}
    Default initialization of an InvalidSatelliteParamError

    Simply sets self.sat = sat (a Satellite object).  Most errors can be 
    well described using only the parameters stored in the Satellite 
    object.

\setlength{\parskip}{1ex}
      Overrides: object.\_\_init\_\_ 	extit{(inherited documentation)}

    \end{boxedminipage}

    \vspace{0.5ex}

\hspace{.8\funcindent}\begin{boxedminipage}{\funcwidth}

    \raggedright \textbf{\_\_str\_\_}(\textit{self})

\setlength{\parskip}{2ex}
    str(x)

\setlength{\parskip}{1ex}
      Overrides: object.\_\_str\_\_ 	extit{(inherited documentation)}

    \end{boxedminipage}


\large{\textbf{\textit{Inherited from exceptions.Exception}}}

\begin{quote}
\_\_new\_\_()
\end{quote}

\large{\textbf{\textit{Inherited from exceptions.BaseException}}}

\begin{quote}
\_\_delattr\_\_(), \_\_getattribute\_\_(), \_\_getitem\_\_(), \_\_getslice\_\_(), \_\_reduce\_\_(), \_\_repr\_\_(), \_\_setattr\_\_(), \_\_setstate\_\_()
\end{quote}

\large{\textbf{\textit{Inherited from object}}}

\begin{quote}
\_\_hash\_\_(), \_\_reduce\_ex\_\_()
\end{quote}

%%%%%%%%%%%%%%%%%%%%%%%%%%%%%%%%%%%%%%%%%%%%%%%%%%%%%%%%%%%%%%%%%%%%%%%%%%%
%%                              Properties                               %%
%%%%%%%%%%%%%%%%%%%%%%%%%%%%%%%%%%%%%%%%%%%%%%%%%%%%%%%%%%%%%%%%%%%%%%%%%%%

  \subsubsection{Properties}

    \vspace{-1cm}
\hspace{\varindent}\begin{longtable}{|p{\varnamewidth}|p{\vardescrwidth}|l}
\cline{1-2}
\cline{1-2} \centering \textbf{Name} & \centering \textbf{Description}& \\
\cline{1-2}
\endhead\cline{1-2}\multicolumn{3}{r}{\small\textit{continued on next page}}\\\endfoot\cline{1-2}
\endlastfoot\multicolumn{2}{|l|}{\textit{Inherited from exceptions.BaseException}}\\
\multicolumn{2}{|p{\varwidth}|}{\raggedright args, message}\\
\cline{1-2}
\multicolumn{2}{|l|}{\textit{Inherited from object}}\\
\multicolumn{2}{|p{\varwidth}|}{\raggedright \_\_class\_\_}\\
\cline{1-2}
\end{longtable}

    \index{SatStress \textit{(package)}!SatStress.SatStress \textit{(module)}!SatStress.SatStress.LoveExcessiveDeltaError \textit{(class)}|)}

%%%%%%%%%%%%%%%%%%%%%%%%%%%%%%%%%%%%%%%%%%%%%%%%%%%%%%%%%%%%%%%%%%%%%%%%%%%
%%                           Class Description                           %%
%%%%%%%%%%%%%%%%%%%%%%%%%%%%%%%%%%%%%%%%%%%%%%%%%%%%%%%%%%%%%%%%%%%%%%%%%%%

    \index{SatStress \textit{(package)}!SatStress.SatStress \textit{(module)}!SatStress.SatStress.GravitationallyUnstableSatelliteError \textit{(class)}|(}
\subsection{Class GravitationallyUnstableSatelliteError}

    \label{SatStress:SatStress:GravitationallyUnstableSatelliteError}
\begin{tabular}{cccccccccccccccc}
% Line for object, linespec=[False, False, False, False, False, False]
\multicolumn{2}{r}{\settowidth{\BCL}{object}\multirow{2}{\BCL}{object}}
&&
&&
&&
&&
&&
&&
  \\\cline{3-3}
  &&\multicolumn{1}{c|}{}
&&
&&
&&
&&
&&
&&
  \\
% Line for exceptions.BaseException, linespec=[False, False, False, False, False]
\multicolumn{4}{r}{\settowidth{\BCL}{exceptions.BaseException}\multirow{2}{\BCL}{exceptions.BaseException}}
&&
&&
&&
&&
&&
  \\\cline{5-5}
  &&&&\multicolumn{1}{c|}{}
&&
&&
&&
&&
&&
  \\
% Line for exceptions.Exception, linespec=[False, False, False, False]
\multicolumn{6}{r}{\settowidth{\BCL}{exceptions.Exception}\multirow{2}{\BCL}{exceptions.Exception}}
&&
&&
&&
&&
  \\\cline{7-7}
  &&&&&&\multicolumn{1}{c|}{}
&&
&&
&&
&&
  \\
% Line for SatStress.SatStress.Error, linespec=[False, False, False]
\multicolumn{8}{r}{\settowidth{\BCL}{SatStress.SatStress.Error}\multirow{2}{\BCL}{SatStress.SatStress.Error}}
&&
&&
&&
  \\\cline{9-9}
  &&&&&&&&\multicolumn{1}{c|}{}
&&
&&
&&
  \\
% Line for SatStress.SatStress.SatelliteParamError, linespec=[False, False]
\multicolumn{10}{r}{\settowidth{\BCL}{SatStress.SatStress.SatelliteParamError}\multirow{2}{\BCL}{SatStress.SatStress.SatelliteParamError}}
&&
&&
  \\\cline{11-11}
  &&&&&&&&&&\multicolumn{1}{c|}{}
&&
&&
  \\
% Line for SatStress.SatStress.InvalidSatelliteParamError, linespec=[False]
\multicolumn{12}{r}{\settowidth{\BCL}{SatStress.SatStress.InvalidSatelliteParamError}\multirow{2}{\BCL}{SatStress.SatStress.InvalidSatelliteParamError}}
&&
  \\\cline{13-13}
  &&&&&&&&&&&&\multicolumn{1}{c|}{}
&&
  \\
&&&&&&&&&&&&\multicolumn{2}{l}{\textbf{SatStress.SatStress.GravitationallyUnstableSatelliteError}}
\end{tabular}

Raised if the density of layers is found not to decrease as you move toward
the surface from the center of the satellite.


%%%%%%%%%%%%%%%%%%%%%%%%%%%%%%%%%%%%%%%%%%%%%%%%%%%%%%%%%%%%%%%%%%%%%%%%%%%
%%                                Methods                                %%
%%%%%%%%%%%%%%%%%%%%%%%%%%%%%%%%%%%%%%%%%%%%%%%%%%%%%%%%%%%%%%%%%%%%%%%%%%%

  \subsubsection{Methods}

    \vspace{0.5ex}

\hspace{.8\funcindent}\begin{boxedminipage}{\funcwidth}

    \raggedright \textbf{\_\_init\_\_}(\textit{self}, \textit{sat}, \textit{layer\_n})

    \vspace{-1.5ex}

    \rule{\textwidth}{0.5\fboxrule}
\setlength{\parskip}{2ex}
    Overrides the base InvalidSatelliteParamError.\_\_init\_\_() function.

    We also need to know which layers have a gravitationally unstable 
    arrangement.

\setlength{\parskip}{1ex}
      Overrides: object.\_\_init\_\_

    \end{boxedminipage}

    \vspace{0.5ex}

\hspace{.8\funcindent}\begin{boxedminipage}{\funcwidth}

    \raggedright \textbf{\_\_str\_\_}(\textit{self})

\setlength{\parskip}{2ex}
    str(x)

\setlength{\parskip}{1ex}
      Overrides: object.\_\_str\_\_ 	extit{(inherited documentation)}

    \end{boxedminipage}


\large{\textbf{\textit{Inherited from exceptions.Exception}}}

\begin{quote}
\_\_new\_\_()
\end{quote}

\large{\textbf{\textit{Inherited from exceptions.BaseException}}}

\begin{quote}
\_\_delattr\_\_(), \_\_getattribute\_\_(), \_\_getitem\_\_(), \_\_getslice\_\_(), \_\_reduce\_\_(), \_\_repr\_\_(), \_\_setattr\_\_(), \_\_setstate\_\_()
\end{quote}

\large{\textbf{\textit{Inherited from object}}}

\begin{quote}
\_\_hash\_\_(), \_\_reduce\_ex\_\_()
\end{quote}

%%%%%%%%%%%%%%%%%%%%%%%%%%%%%%%%%%%%%%%%%%%%%%%%%%%%%%%%%%%%%%%%%%%%%%%%%%%
%%                              Properties                               %%
%%%%%%%%%%%%%%%%%%%%%%%%%%%%%%%%%%%%%%%%%%%%%%%%%%%%%%%%%%%%%%%%%%%%%%%%%%%

  \subsubsection{Properties}

    \vspace{-1cm}
\hspace{\varindent}\begin{longtable}{|p{\varnamewidth}|p{\vardescrwidth}|l}
\cline{1-2}
\cline{1-2} \centering \textbf{Name} & \centering \textbf{Description}& \\
\cline{1-2}
\endhead\cline{1-2}\multicolumn{3}{r}{\small\textit{continued on next page}}\\\endfoot\cline{1-2}
\endlastfoot\multicolumn{2}{|l|}{\textit{Inherited from exceptions.BaseException}}\\
\multicolumn{2}{|p{\varwidth}|}{\raggedright args, message}\\
\cline{1-2}
\multicolumn{2}{|l|}{\textit{Inherited from object}}\\
\multicolumn{2}{|p{\varwidth}|}{\raggedright \_\_class\_\_}\\
\cline{1-2}
\end{longtable}

    \index{SatStress \textit{(package)}!SatStress.SatStress \textit{(module)}!SatStress.SatStress.GravitationallyUnstableSatelliteError \textit{(class)}|)}

%%%%%%%%%%%%%%%%%%%%%%%%%%%%%%%%%%%%%%%%%%%%%%%%%%%%%%%%%%%%%%%%%%%%%%%%%%%
%%                           Class Description                           %%
%%%%%%%%%%%%%%%%%%%%%%%%%%%%%%%%%%%%%%%%%%%%%%%%%%%%%%%%%%%%%%%%%%%%%%%%%%%

    \index{SatStress \textit{(package)}!SatStress.SatStress \textit{(module)}!SatStress.SatStress.NonNumberSatelliteParamError \textit{(class)}|(}
\subsection{Class NonNumberSatelliteParamError}

    \label{SatStress:SatStress:NonNumberSatelliteParamError}
\begin{tabular}{cccccccccccccccc}
% Line for object, linespec=[False, False, False, False, False, False]
\multicolumn{2}{r}{\settowidth{\BCL}{object}\multirow{2}{\BCL}{object}}
&&
&&
&&
&&
&&
&&
  \\\cline{3-3}
  &&\multicolumn{1}{c|}{}
&&
&&
&&
&&
&&
&&
  \\
% Line for exceptions.BaseException, linespec=[False, False, False, False, False]
\multicolumn{4}{r}{\settowidth{\BCL}{exceptions.BaseException}\multirow{2}{\BCL}{exceptions.BaseException}}
&&
&&
&&
&&
&&
  \\\cline{5-5}
  &&&&\multicolumn{1}{c|}{}
&&
&&
&&
&&
&&
  \\
% Line for exceptions.Exception, linespec=[False, False, False, False]
\multicolumn{6}{r}{\settowidth{\BCL}{exceptions.Exception}\multirow{2}{\BCL}{exceptions.Exception}}
&&
&&
&&
&&
  \\\cline{7-7}
  &&&&&&\multicolumn{1}{c|}{}
&&
&&
&&
&&
  \\
% Line for SatStress.SatStress.Error, linespec=[False, False, False]
\multicolumn{8}{r}{\settowidth{\BCL}{SatStress.SatStress.Error}\multirow{2}{\BCL}{SatStress.SatStress.Error}}
&&
&&
&&
  \\\cline{9-9}
  &&&&&&&&\multicolumn{1}{c|}{}
&&
&&
&&
  \\
% Line for SatStress.SatStress.SatelliteParamError, linespec=[False, False]
\multicolumn{10}{r}{\settowidth{\BCL}{SatStress.SatStress.SatelliteParamError}\multirow{2}{\BCL}{SatStress.SatStress.SatelliteParamError}}
&&
&&
  \\\cline{11-11}
  &&&&&&&&&&\multicolumn{1}{c|}{}
&&
&&
  \\
% Line for SatStress.SatStress.InvalidSatelliteParamError, linespec=[False]
\multicolumn{12}{r}{\settowidth{\BCL}{SatStress.SatStress.InvalidSatelliteParamError}\multirow{2}{\BCL}{SatStress.SatStress.InvalidSatelliteParamError}}
&&
  \\\cline{13-13}
  &&&&&&&&&&&&\multicolumn{1}{c|}{}
&&
  \\
&&&&&&&&&&&&\multicolumn{2}{l}{\textbf{SatStress.SatStress.NonNumberSatelliteParamError}}
\end{tabular}

Indicates that a non-numeric value was found for a numerical parameter.


%%%%%%%%%%%%%%%%%%%%%%%%%%%%%%%%%%%%%%%%%%%%%%%%%%%%%%%%%%%%%%%%%%%%%%%%%%%
%%                                Methods                                %%
%%%%%%%%%%%%%%%%%%%%%%%%%%%%%%%%%%%%%%%%%%%%%%%%%%%%%%%%%%%%%%%%%%%%%%%%%%%

  \subsubsection{Methods}

    \vspace{0.5ex}

\hspace{.8\funcindent}\begin{boxedminipage}{\funcwidth}

    \raggedright \textbf{\_\_init\_\_}(\textit{self}, \textit{sat}, \textit{badname})

\setlength{\parskip}{2ex}
    Default initialization of an InvalidSatelliteParamError

    Simply sets self.sat = sat (a Satellite object).  Most errors can be 
    well described using only the parameters stored in the Satellite 
    object.

\setlength{\parskip}{1ex}
      Overrides: object.\_\_init\_\_ 	extit{(inherited documentation)}

    \end{boxedminipage}

    \vspace{0.5ex}

\hspace{.8\funcindent}\begin{boxedminipage}{\funcwidth}

    \raggedright \textbf{\_\_str\_\_}(\textit{self})

\setlength{\parskip}{2ex}
    str(x)

\setlength{\parskip}{1ex}
      Overrides: object.\_\_str\_\_ 	extit{(inherited documentation)}

    \end{boxedminipage}


\large{\textbf{\textit{Inherited from exceptions.Exception}}}

\begin{quote}
\_\_new\_\_()
\end{quote}

\large{\textbf{\textit{Inherited from exceptions.BaseException}}}

\begin{quote}
\_\_delattr\_\_(), \_\_getattribute\_\_(), \_\_getitem\_\_(), \_\_getslice\_\_(), \_\_reduce\_\_(), \_\_repr\_\_(), \_\_setattr\_\_(), \_\_setstate\_\_()
\end{quote}

\large{\textbf{\textit{Inherited from object}}}

\begin{quote}
\_\_hash\_\_(), \_\_reduce\_ex\_\_()
\end{quote}

%%%%%%%%%%%%%%%%%%%%%%%%%%%%%%%%%%%%%%%%%%%%%%%%%%%%%%%%%%%%%%%%%%%%%%%%%%%
%%                              Properties                               %%
%%%%%%%%%%%%%%%%%%%%%%%%%%%%%%%%%%%%%%%%%%%%%%%%%%%%%%%%%%%%%%%%%%%%%%%%%%%

  \subsubsection{Properties}

    \vspace{-1cm}
\hspace{\varindent}\begin{longtable}{|p{\varnamewidth}|p{\vardescrwidth}|l}
\cline{1-2}
\cline{1-2} \centering \textbf{Name} & \centering \textbf{Description}& \\
\cline{1-2}
\endhead\cline{1-2}\multicolumn{3}{r}{\small\textit{continued on next page}}\\\endfoot\cline{1-2}
\endlastfoot\multicolumn{2}{|l|}{\textit{Inherited from exceptions.BaseException}}\\
\multicolumn{2}{|p{\varwidth}|}{\raggedright args, message}\\
\cline{1-2}
\multicolumn{2}{|l|}{\textit{Inherited from object}}\\
\multicolumn{2}{|p{\varwidth}|}{\raggedright \_\_class\_\_}\\
\cline{1-2}
\end{longtable}

    \index{SatStress \textit{(package)}!SatStress.SatStress \textit{(module)}!SatStress.SatStress.NonNumberSatelliteParamError \textit{(class)}|)}

%%%%%%%%%%%%%%%%%%%%%%%%%%%%%%%%%%%%%%%%%%%%%%%%%%%%%%%%%%%%%%%%%%%%%%%%%%%
%%                           Class Description                           %%
%%%%%%%%%%%%%%%%%%%%%%%%%%%%%%%%%%%%%%%%%%%%%%%%%%%%%%%%%%%%%%%%%%%%%%%%%%%

    \index{SatStress \textit{(package)}!SatStress.SatStress \textit{(module)}!SatStress.SatStress.LowLayerDensityError \textit{(class)}|(}
\subsection{Class LowLayerDensityError}

    \label{SatStress:SatStress:LowLayerDensityError}
\begin{tabular}{cccccccccccccccc}
% Line for object, linespec=[False, False, False, False, False, False]
\multicolumn{2}{r}{\settowidth{\BCL}{object}\multirow{2}{\BCL}{object}}
&&
&&
&&
&&
&&
&&
  \\\cline{3-3}
  &&\multicolumn{1}{c|}{}
&&
&&
&&
&&
&&
&&
  \\
% Line for exceptions.BaseException, linespec=[False, False, False, False, False]
\multicolumn{4}{r}{\settowidth{\BCL}{exceptions.BaseException}\multirow{2}{\BCL}{exceptions.BaseException}}
&&
&&
&&
&&
&&
  \\\cline{5-5}
  &&&&\multicolumn{1}{c|}{}
&&
&&
&&
&&
&&
  \\
% Line for exceptions.Exception, linespec=[False, False, False, False]
\multicolumn{6}{r}{\settowidth{\BCL}{exceptions.Exception}\multirow{2}{\BCL}{exceptions.Exception}}
&&
&&
&&
&&
  \\\cline{7-7}
  &&&&&&\multicolumn{1}{c|}{}
&&
&&
&&
&&
  \\
% Line for SatStress.SatStress.Error, linespec=[False, False, False]
\multicolumn{8}{r}{\settowidth{\BCL}{SatStress.SatStress.Error}\multirow{2}{\BCL}{SatStress.SatStress.Error}}
&&
&&
&&
  \\\cline{9-9}
  &&&&&&&&\multicolumn{1}{c|}{}
&&
&&
&&
  \\
% Line for SatStress.SatStress.SatelliteParamError, linespec=[False, False]
\multicolumn{10}{r}{\settowidth{\BCL}{SatStress.SatStress.SatelliteParamError}\multirow{2}{\BCL}{SatStress.SatStress.SatelliteParamError}}
&&
&&
  \\\cline{11-11}
  &&&&&&&&&&\multicolumn{1}{c|}{}
&&
&&
  \\
% Line for SatStress.SatStress.InvalidSatelliteParamError, linespec=[False]
\multicolumn{12}{r}{\settowidth{\BCL}{SatStress.SatStress.InvalidSatelliteParamError}\multirow{2}{\BCL}{SatStress.SatStress.InvalidSatelliteParamError}}
&&
  \\\cline{13-13}
  &&&&&&&&&&&&\multicolumn{1}{c|}{}
&&
  \\
&&&&&&&&&&&&\multicolumn{2}{l}{\textbf{SatStress.SatStress.LowLayerDensityError}}
\end{tabular}

Indicates that a layer has been assigned an unrealistically low density.


%%%%%%%%%%%%%%%%%%%%%%%%%%%%%%%%%%%%%%%%%%%%%%%%%%%%%%%%%%%%%%%%%%%%%%%%%%%
%%                                Methods                                %%
%%%%%%%%%%%%%%%%%%%%%%%%%%%%%%%%%%%%%%%%%%%%%%%%%%%%%%%%%%%%%%%%%%%%%%%%%%%

  \subsubsection{Methods}

    \vspace{0.5ex}

\hspace{.8\funcindent}\begin{boxedminipage}{\funcwidth}

    \raggedright \textbf{\_\_init\_\_}(\textit{self}, \textit{sat}, \textit{layer\_n})

\setlength{\parskip}{2ex}
    Default initialization of an InvalidSatelliteParamError

    Simply sets self.sat = sat (a Satellite object).  Most errors can be 
    well described using only the parameters stored in the Satellite 
    object.

\setlength{\parskip}{1ex}
      Overrides: object.\_\_init\_\_ 	extit{(inherited documentation)}

    \end{boxedminipage}

    \vspace{0.5ex}

\hspace{.8\funcindent}\begin{boxedminipage}{\funcwidth}

    \raggedright \textbf{\_\_str\_\_}(\textit{self})

\setlength{\parskip}{2ex}
    str(x)

\setlength{\parskip}{1ex}
      Overrides: object.\_\_str\_\_ 	extit{(inherited documentation)}

    \end{boxedminipage}


\large{\textbf{\textit{Inherited from exceptions.Exception}}}

\begin{quote}
\_\_new\_\_()
\end{quote}

\large{\textbf{\textit{Inherited from exceptions.BaseException}}}

\begin{quote}
\_\_delattr\_\_(), \_\_getattribute\_\_(), \_\_getitem\_\_(), \_\_getslice\_\_(), \_\_reduce\_\_(), \_\_repr\_\_(), \_\_setattr\_\_(), \_\_setstate\_\_()
\end{quote}

\large{\textbf{\textit{Inherited from object}}}

\begin{quote}
\_\_hash\_\_(), \_\_reduce\_ex\_\_()
\end{quote}

%%%%%%%%%%%%%%%%%%%%%%%%%%%%%%%%%%%%%%%%%%%%%%%%%%%%%%%%%%%%%%%%%%%%%%%%%%%
%%                              Properties                               %%
%%%%%%%%%%%%%%%%%%%%%%%%%%%%%%%%%%%%%%%%%%%%%%%%%%%%%%%%%%%%%%%%%%%%%%%%%%%

  \subsubsection{Properties}

    \vspace{-1cm}
\hspace{\varindent}\begin{longtable}{|p{\varnamewidth}|p{\vardescrwidth}|l}
\cline{1-2}
\cline{1-2} \centering \textbf{Name} & \centering \textbf{Description}& \\
\cline{1-2}
\endhead\cline{1-2}\multicolumn{3}{r}{\small\textit{continued on next page}}\\\endfoot\cline{1-2}
\endlastfoot\multicolumn{2}{|l|}{\textit{Inherited from exceptions.BaseException}}\\
\multicolumn{2}{|p{\varwidth}|}{\raggedright args, message}\\
\cline{1-2}
\multicolumn{2}{|l|}{\textit{Inherited from object}}\\
\multicolumn{2}{|p{\varwidth}|}{\raggedright \_\_class\_\_}\\
\cline{1-2}
\end{longtable}

    \index{SatStress \textit{(package)}!SatStress.SatStress \textit{(module)}!SatStress.SatStress.LowLayerDensityError \textit{(class)}|)}

%%%%%%%%%%%%%%%%%%%%%%%%%%%%%%%%%%%%%%%%%%%%%%%%%%%%%%%%%%%%%%%%%%%%%%%%%%%
%%                           Class Description                           %%
%%%%%%%%%%%%%%%%%%%%%%%%%%%%%%%%%%%%%%%%%%%%%%%%%%%%%%%%%%%%%%%%%%%%%%%%%%%

    \index{SatStress \textit{(package)}!SatStress.SatStress \textit{(module)}!SatStress.SatStress.LowLayerThicknessError \textit{(class)}|(}
\subsection{Class LowLayerThicknessError}

    \label{SatStress:SatStress:LowLayerThicknessError}
\begin{tabular}{cccccccccccccccc}
% Line for object, linespec=[False, False, False, False, False, False]
\multicolumn{2}{r}{\settowidth{\BCL}{object}\multirow{2}{\BCL}{object}}
&&
&&
&&
&&
&&
&&
  \\\cline{3-3}
  &&\multicolumn{1}{c|}{}
&&
&&
&&
&&
&&
&&
  \\
% Line for exceptions.BaseException, linespec=[False, False, False, False, False]
\multicolumn{4}{r}{\settowidth{\BCL}{exceptions.BaseException}\multirow{2}{\BCL}{exceptions.BaseException}}
&&
&&
&&
&&
&&
  \\\cline{5-5}
  &&&&\multicolumn{1}{c|}{}
&&
&&
&&
&&
&&
  \\
% Line for exceptions.Exception, linespec=[False, False, False, False]
\multicolumn{6}{r}{\settowidth{\BCL}{exceptions.Exception}\multirow{2}{\BCL}{exceptions.Exception}}
&&
&&
&&
&&
  \\\cline{7-7}
  &&&&&&\multicolumn{1}{c|}{}
&&
&&
&&
&&
  \\
% Line for SatStress.SatStress.Error, linespec=[False, False, False]
\multicolumn{8}{r}{\settowidth{\BCL}{SatStress.SatStress.Error}\multirow{2}{\BCL}{SatStress.SatStress.Error}}
&&
&&
&&
  \\\cline{9-9}
  &&&&&&&&\multicolumn{1}{c|}{}
&&
&&
&&
  \\
% Line for SatStress.SatStress.SatelliteParamError, linespec=[False, False]
\multicolumn{10}{r}{\settowidth{\BCL}{SatStress.SatStress.SatelliteParamError}\multirow{2}{\BCL}{SatStress.SatStress.SatelliteParamError}}
&&
&&
  \\\cline{11-11}
  &&&&&&&&&&\multicolumn{1}{c|}{}
&&
&&
  \\
% Line for SatStress.SatStress.InvalidSatelliteParamError, linespec=[False]
\multicolumn{12}{r}{\settowidth{\BCL}{SatStress.SatStress.InvalidSatelliteParamError}\multirow{2}{\BCL}{SatStress.SatStress.InvalidSatelliteParamError}}
&&
  \\\cline{13-13}
  &&&&&&&&&&&&\multicolumn{1}{c|}{}
&&
  \\
&&&&&&&&&&&&\multicolumn{2}{l}{\textbf{SatStress.SatStress.LowLayerThicknessError}}
\end{tabular}

Indicates that a layer has been given too small of a thickness


%%%%%%%%%%%%%%%%%%%%%%%%%%%%%%%%%%%%%%%%%%%%%%%%%%%%%%%%%%%%%%%%%%%%%%%%%%%
%%                                Methods                                %%
%%%%%%%%%%%%%%%%%%%%%%%%%%%%%%%%%%%%%%%%%%%%%%%%%%%%%%%%%%%%%%%%%%%%%%%%%%%

  \subsubsection{Methods}

    \vspace{0.5ex}

\hspace{.8\funcindent}\begin{boxedminipage}{\funcwidth}

    \raggedright \textbf{\_\_init\_\_}(\textit{self}, \textit{sat}, \textit{layer\_n})

\setlength{\parskip}{2ex}
    Default initialization of an InvalidSatelliteParamError

    Simply sets self.sat = sat (a Satellite object).  Most errors can be 
    well described using only the parameters stored in the Satellite 
    object.

\setlength{\parskip}{1ex}
      Overrides: object.\_\_init\_\_ 	extit{(inherited documentation)}

    \end{boxedminipage}

    \vspace{0.5ex}

\hspace{.8\funcindent}\begin{boxedminipage}{\funcwidth}

    \raggedright \textbf{\_\_str\_\_}(\textit{self})

\setlength{\parskip}{2ex}
    str(x)

\setlength{\parskip}{1ex}
      Overrides: object.\_\_str\_\_ 	extit{(inherited documentation)}

    \end{boxedminipage}


\large{\textbf{\textit{Inherited from exceptions.Exception}}}

\begin{quote}
\_\_new\_\_()
\end{quote}

\large{\textbf{\textit{Inherited from exceptions.BaseException}}}

\begin{quote}
\_\_delattr\_\_(), \_\_getattribute\_\_(), \_\_getitem\_\_(), \_\_getslice\_\_(), \_\_reduce\_\_(), \_\_repr\_\_(), \_\_setattr\_\_(), \_\_setstate\_\_()
\end{quote}

\large{\textbf{\textit{Inherited from object}}}

\begin{quote}
\_\_hash\_\_(), \_\_reduce\_ex\_\_()
\end{quote}

%%%%%%%%%%%%%%%%%%%%%%%%%%%%%%%%%%%%%%%%%%%%%%%%%%%%%%%%%%%%%%%%%%%%%%%%%%%
%%                              Properties                               %%
%%%%%%%%%%%%%%%%%%%%%%%%%%%%%%%%%%%%%%%%%%%%%%%%%%%%%%%%%%%%%%%%%%%%%%%%%%%

  \subsubsection{Properties}

    \vspace{-1cm}
\hspace{\varindent}\begin{longtable}{|p{\varnamewidth}|p{\vardescrwidth}|l}
\cline{1-2}
\cline{1-2} \centering \textbf{Name} & \centering \textbf{Description}& \\
\cline{1-2}
\endhead\cline{1-2}\multicolumn{3}{r}{\small\textit{continued on next page}}\\\endfoot\cline{1-2}
\endlastfoot\multicolumn{2}{|l|}{\textit{Inherited from exceptions.BaseException}}\\
\multicolumn{2}{|p{\varwidth}|}{\raggedright args, message}\\
\cline{1-2}
\multicolumn{2}{|l|}{\textit{Inherited from object}}\\
\multicolumn{2}{|p{\varwidth}|}{\raggedright \_\_class\_\_}\\
\cline{1-2}
\end{longtable}

    \index{SatStress \textit{(package)}!SatStress.SatStress \textit{(module)}!SatStress.SatStress.LowLayerThicknessError \textit{(class)}|)}

%%%%%%%%%%%%%%%%%%%%%%%%%%%%%%%%%%%%%%%%%%%%%%%%%%%%%%%%%%%%%%%%%%%%%%%%%%%
%%                           Class Description                           %%
%%%%%%%%%%%%%%%%%%%%%%%%%%%%%%%%%%%%%%%%%%%%%%%%%%%%%%%%%%%%%%%%%%%%%%%%%%%

    \index{SatStress \textit{(package)}!SatStress.SatStress \textit{(module)}!SatStress.SatStress.NegativeLayerParamError \textit{(class)}|(}
\subsection{Class NegativeLayerParamError}

    \label{SatStress:SatStress:NegativeLayerParamError}
\begin{tabular}{cccccccccccccccc}
% Line for object, linespec=[False, False, False, False, False, False]
\multicolumn{2}{r}{\settowidth{\BCL}{object}\multirow{2}{\BCL}{object}}
&&
&&
&&
&&
&&
&&
  \\\cline{3-3}
  &&\multicolumn{1}{c|}{}
&&
&&
&&
&&
&&
&&
  \\
% Line for exceptions.BaseException, linespec=[False, False, False, False, False]
\multicolumn{4}{r}{\settowidth{\BCL}{exceptions.BaseException}\multirow{2}{\BCL}{exceptions.BaseException}}
&&
&&
&&
&&
&&
  \\\cline{5-5}
  &&&&\multicolumn{1}{c|}{}
&&
&&
&&
&&
&&
  \\
% Line for exceptions.Exception, linespec=[False, False, False, False]
\multicolumn{6}{r}{\settowidth{\BCL}{exceptions.Exception}\multirow{2}{\BCL}{exceptions.Exception}}
&&
&&
&&
&&
  \\\cline{7-7}
  &&&&&&\multicolumn{1}{c|}{}
&&
&&
&&
&&
  \\
% Line for SatStress.SatStress.Error, linespec=[False, False, False]
\multicolumn{8}{r}{\settowidth{\BCL}{SatStress.SatStress.Error}\multirow{2}{\BCL}{SatStress.SatStress.Error}}
&&
&&
&&
  \\\cline{9-9}
  &&&&&&&&\multicolumn{1}{c|}{}
&&
&&
&&
  \\
% Line for SatStress.SatStress.SatelliteParamError, linespec=[False, False]
\multicolumn{10}{r}{\settowidth{\BCL}{SatStress.SatStress.SatelliteParamError}\multirow{2}{\BCL}{SatStress.SatStress.SatelliteParamError}}
&&
&&
  \\\cline{11-11}
  &&&&&&&&&&\multicolumn{1}{c|}{}
&&
&&
  \\
% Line for SatStress.SatStress.InvalidSatelliteParamError, linespec=[False]
\multicolumn{12}{r}{\settowidth{\BCL}{SatStress.SatStress.InvalidSatelliteParamError}\multirow{2}{\BCL}{SatStress.SatStress.InvalidSatelliteParamError}}
&&
  \\\cline{13-13}
  &&&&&&&&&&&&\multicolumn{1}{c|}{}
&&
  \\
&&&&&&&&&&&&\multicolumn{2}{l}{\textbf{SatStress.SatStress.NegativeLayerParamError}}
\end{tabular}

Indicates a layer has been given an unphysical material property.


%%%%%%%%%%%%%%%%%%%%%%%%%%%%%%%%%%%%%%%%%%%%%%%%%%%%%%%%%%%%%%%%%%%%%%%%%%%
%%                                Methods                                %%
%%%%%%%%%%%%%%%%%%%%%%%%%%%%%%%%%%%%%%%%%%%%%%%%%%%%%%%%%%%%%%%%%%%%%%%%%%%

  \subsubsection{Methods}

    \vspace{0.5ex}

\hspace{.8\funcindent}\begin{boxedminipage}{\funcwidth}

    \raggedright \textbf{\_\_init\_\_}(\textit{self}, \textit{sat}, \textit{badparam})

\setlength{\parskip}{2ex}
    Default initialization of an InvalidSatelliteParamError

    Simply sets self.sat = sat (a Satellite object).  Most errors can be 
    well described using only the parameters stored in the Satellite 
    object.

\setlength{\parskip}{1ex}
      Overrides: object.\_\_init\_\_ 	extit{(inherited documentation)}

    \end{boxedminipage}

    \vspace{0.5ex}

\hspace{.8\funcindent}\begin{boxedminipage}{\funcwidth}

    \raggedright \textbf{\_\_str\_\_}(\textit{self})

\setlength{\parskip}{2ex}
    str(x)

\setlength{\parskip}{1ex}
      Overrides: object.\_\_str\_\_ 	extit{(inherited documentation)}

    \end{boxedminipage}


\large{\textbf{\textit{Inherited from exceptions.Exception}}}

\begin{quote}
\_\_new\_\_()
\end{quote}

\large{\textbf{\textit{Inherited from exceptions.BaseException}}}

\begin{quote}
\_\_delattr\_\_(), \_\_getattribute\_\_(), \_\_getitem\_\_(), \_\_getslice\_\_(), \_\_reduce\_\_(), \_\_repr\_\_(), \_\_setattr\_\_(), \_\_setstate\_\_()
\end{quote}

\large{\textbf{\textit{Inherited from object}}}

\begin{quote}
\_\_hash\_\_(), \_\_reduce\_ex\_\_()
\end{quote}

%%%%%%%%%%%%%%%%%%%%%%%%%%%%%%%%%%%%%%%%%%%%%%%%%%%%%%%%%%%%%%%%%%%%%%%%%%%
%%                              Properties                               %%
%%%%%%%%%%%%%%%%%%%%%%%%%%%%%%%%%%%%%%%%%%%%%%%%%%%%%%%%%%%%%%%%%%%%%%%%%%%

  \subsubsection{Properties}

    \vspace{-1cm}
\hspace{\varindent}\begin{longtable}{|p{\varnamewidth}|p{\vardescrwidth}|l}
\cline{1-2}
\cline{1-2} \centering \textbf{Name} & \centering \textbf{Description}& \\
\cline{1-2}
\endhead\cline{1-2}\multicolumn{3}{r}{\small\textit{continued on next page}}\\\endfoot\cline{1-2}
\endlastfoot\multicolumn{2}{|l|}{\textit{Inherited from exceptions.BaseException}}\\
\multicolumn{2}{|p{\varwidth}|}{\raggedright args, message}\\
\cline{1-2}
\multicolumn{2}{|l|}{\textit{Inherited from object}}\\
\multicolumn{2}{|p{\varwidth}|}{\raggedright \_\_class\_\_}\\
\cline{1-2}
\end{longtable}

    \index{SatStress \textit{(package)}!SatStress.SatStress \textit{(module)}!SatStress.SatStress.NegativeLayerParamError \textit{(class)}|)}
    \index{SatStress \textit{(package)}!SatStress.SatStress \textit{(module)}|)}
