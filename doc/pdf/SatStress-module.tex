%
% API Documentation for SatStress
% Package satstress
%
% Generated by epydoc 3.0.1
% [Fri Mar 28 13:14:36 2008]
%

%%%%%%%%%%%%%%%%%%%%%%%%%%%%%%%%%%%%%%%%%%%%%%%%%%%%%%%%%%%%%%%%%%%%%%%%%%%
%%                          Module Description                           %%
%%%%%%%%%%%%%%%%%%%%%%%%%%%%%%%%%%%%%%%%%%%%%%%%%%%%%%%%%%%%%%%%%%%%%%%%%%%

    \index{satstress \textit{(package)}|(}
\section{Package satstress}

    \label{satstress}
Tools for analysing the relationship between tidal stresses and tectonics 
on icy satellites.

Written by Zane 
Selvans\footnote{\href{http://zaneselvans.org}{http://zaneselvans.org}} 
(\texttt{zane.selvans@colorado.edu\footnote{\href{mailto:zane.selvans@colorado.edu}{mailto:zane.selvans@colorado.edu}}})
as part of his Ph.D. dissertation research.

\texttt{SatStress} is released under GNU General Public License (GPL) 
version 3.  For the full text of the license, see: 
\href{http://www.gnu.org/}{\textit{http://www.gnu.org/}}

The project is hosted at Google Code: 
\href{http://code.google.com/p/satstress}{\textit{http://code.google.com/p/satstress}}

(section) 1 Installation

  Hopefully getting \texttt{SatStress} to work on your system is a 
  relatively painless process, however, the software does assume you have 
  basic experience with the Unix shell and programming within a Unix 
  environment (though it should work on Windows too).  In particular, this 
  installation information assumes you already have and are able to use:

  \begin{itemize}
  \setlength{\parskip}{0.6ex}
    \item compilers for both C and Fortran.  Development has been done on Mac 
      OS X (10.5) using the GNU compilers \texttt{gcc} and \texttt{g77}, so
      those should definitely work.  On other systems, with other 
      compilers, your mileage may vary.

    \item the \texttt{make} utility, which manages dependencies between files.

  \end{itemize}

  (section) 1.1 Other Required and Recommended Software

    To get the \texttt{SatStress} package working, you'll need to install 
    some other (free) software first:

    \begin{itemize}
    \setlength{\parskip}{0.6ex}
      \item \textbf{Python 2.5} or later 
        (\href{http://www.python.org}{\textit{http://www.python.org}}).  If
        you're running a recent install of Linux, or Apple's Leopard 
        operating system (OS X 10.5.x), you already have this.  Python is 
        also available for Microsoft Windows, and just about any other 
        platform you can think of.

      \item \textbf{SciPy} 
        (\href{http://www.scipy.org}{\textit{http://www.scipy.org}}), a 
        collection of scientific libraries that extend the capabilities of 
        the Python language.

    \end{itemize}

    In addition, if you want to use \texttt{GridCalc}, you'll need:

    \begin{itemize}
    \setlength{\parskip}{0.6ex}
      \item \textbf{netCDF} 
        (\href{http://www.unidata.ucar.edu/software/netcdf/}{\textit{http://www.unidata.ucar.edu/software/netcdf/}},
        a library of routines for storing, retrieving, and annotating 
        regularly gridded multi-dimensional datasets.  Developed by 
        Unidata\footnote{\href{http://www.unidata.ucar.edu}{http://www.unidata.ucar.edu}}

      \item \textbf{netcdf4-python} 
        (\href{http://code.google.com/p/netcdf4-python/}{\textit{http://code.google.com/p/netcdf4-python/}},
        a Python interface to the netCDF library.

    \end{itemize}

    If you want to actually view \texttt{GridCalc} output, you'll need a 
    netCDF file viewing program.  Many commercial software packages can 
    read netCDF files, such as ESRI ArcGIS and Matlab (from the Mathworks).
    A simple and free reader for OS X is 
    Panoply\footnote{\href{http://www.giss.nasa.gov/tools/panoply/}{http://www.giss.nasa.gov/tools/panoply/}},
    from NASA.  If you want to really be able to interact with the outputs 
    from this model, you should install:

    \begin{itemize}
    \setlength{\parskip}{0.6ex}
      \item \textbf{Matplotlib/Pylab} 
        (\href{http://matplotlib.sourceforge.net/}{\textit{http://matplotlib.sourceforge.net/}}),
        a Matlab-like interactive plotting and analysis package, which uses
        Python as its "shell".

    \end{itemize}

  (section) 1.2 Building and Installing SatStress

    \textbf{RESUME HERE}

    Once you have the required software prerequisites installed, you should
    be able to \texttt{cd} into the directory containing the 
    \texttt{SatStress} module, and simply type \texttt{make all} at the 
    command line.  This will compile the Love number code and run the small
    \texttt{SatStress.test} program embedded within \texttt{SatStress}, 
    just to make sure that everything is in working order (or not).  If 
    you're not using the GNU Fortran 77 compiler \texttt{g77}, you'll need 
    to edit the \texttt{Makefile} for the Love number code:

\begin{alltt}
 ./Love/JohnWahr/Makefile\end{alltt}

(section) 2 Design Overview

  A few notes on the general architecture of the \texttt{SatStress} 
  package.

  (section) 2.1 A Toolkit, not a Program

    The \texttt{SatStress} package is not itself a stand-alone program (or 
    not much of one anyway).  Instead it is a set of tools with which you 
    can build programs that need to know about the stresses on the surface 
    of a satellite, and how they compare to tectonic features, so you can 
    do your own hypothesizing and testing.

  (section) 2.2 Object Oriented

    The package attempts to make use of object oriented 
    programming\footnote{\href{http://en.wikipedia.org/wiki/Object-oriented_programming}{http://en.wikipedia.org/wiki/Object-oriented\_programming}}
    (OOP) in order to maximize the re-usability and extensibility of the 
    code.  Many scientists are more familiar with the imperative 
    programming 
    style\footnote{\href{http://en.wikipedia.org/wiki/Imperative_programming}{http://en.wikipedia.org/wiki/Imperative\_programming}}
    of languages like Fortran and C, but as more data analysis and 
    hypothesis testing takes place inside computers, and as many scientists
    become highly specialized and knowledgeable software engineers (even if
    they don't want to admit it), the advantages of OOP become significant.
    If the object orientation of this module seems odd at first glance, 
    don't despair, it's worth learning.

  (section) 2.3 Written in Python

    Python\footnote{\href{http://www.python.org}{http://www.python.org}} is
    a general purpose, high-level scripting language.  It is an interpreted
    language (as opposed to compiled languages like Fortran or C) and so 
    Python code is very portable, meaning it is usable on a wide variety of
    computing platforms without any alteration.  It is relatively easy to 
    learn and easy to read, and it has a very active development community.
    It also has a large base of friendly, helpful scientific users and an 
    enormous selection of pre-existing libraries designed for scientific 
    applications.  For those tasks which are particularly computationally 
    intensive, Python allows you to extend the language with code written 
    in C and Fortran.  Python is also Free 
    Software\footnote{\href{http://www.gnu.org/philosophy/free-sw.html}{http://www.gnu.org/philosophy/free-sw.html}}.
    If you are a scientist and you write code, Python is a great choice.

  (section) 2.4 Open Source

    Because science today is intimately intertwined with computation, it is
    important for researchers to share the code that their scientific 
    results are based on.  No matter how elegant and accurate your 
    derivation is, if your implementation of the model in code is wrong, 
    your results will be flawed. As our models and hypotheses become more 
    complex, our code becomes vital primary source material, and it needs 
    to be open to peer review.  Opening our source:

    \begin{itemize}
    \setlength{\parskip}{0.6ex}
      \item allows bugs to be found and fixed more quickly

      \item facilitates collaboration and interoperability

      \item reduces duplicated effort

      \item enhances institutional memory

      \item encourages better software design and documentation

    \end{itemize}

    Of course, it also means that other people can use our code to write 
    their own scientific papers, but \textit{that is the fundamental nature
    of science}.  We are all "standing on the shoulders of giants".  Nobody
    re-derives quantum mechanics when they just want to do a little 
    spectroscopy.  Why should we all be re-writing each others code 
    \textit{ad nauseam}?  Opening scientific source code will ultimately 
    increase everyone's productivity.  Additionally, a great deal of 
    science is funded by the public, and our code is a major product of 
    that funding.  It is unethical to make it proprietary.


%%%%%%%%%%%%%%%%%%%%%%%%%%%%%%%%%%%%%%%%%%%%%%%%%%%%%%%%%%%%%%%%%%%%%%%%%%%
%%                                Modules                                %%
%%%%%%%%%%%%%%%%%%%%%%%%%%%%%%%%%%%%%%%%%%%%%%%%%%%%%%%%%%%%%%%%%%%%%%%%%%%

\subsection{Modules}

\begin{itemize}
\setlength{\parskip}{0ex}
\item \textbf{GridCalc}: Calculate stresses on an icy satellite over a rectangular geographic region
on a regularly spaced lat-lon grid.



  \textit{(Section \ref{satstress:GridCalc}, p.~\pageref{satstress:GridCalc})}

\item \textbf{SatStress}: A framework for calculating the surface stresses at a particular place and 
time on a satellite resulting from one or more tidal potentials.



  \textit{(Section \ref{satstress:SatStress}, p.~\pageref{satstress:SatStress})}

\item \textbf{physcon}: A Python dictionary of physical constants in SI units.



  \textit{(Section \ref{satstress:physcon}, p.~\pageref{satstress:physcon})}

\end{itemize}

    \index{satstress \textit{(package)}|)}
